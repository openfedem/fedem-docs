% SPDX-FileCopyrightText: 2023 SAP SE
%
% SPDX-License-Identifier: Apache-2.0
%
% This file is part of FEDEM - https://openfedem.org

% Some useful macros for generating the Fedem Users Guide

\def\IconText#1#2{
 \hskip-25mm
 \begin{minipage}{1.1\textwidth}
  \begin{tabular}{p{0.1\textwidth} p{0.9\textwidth}}
   \includegraphics[width=7mm]{Figures/Icons/#1} &
   \raggedright\vskip-7mm #2
  \end{tabular}
 \end{minipage}}

% Use this instead of \IconText when first paragraph in a section
\def\IconTextFirst#1#2{
 \vskip\parskip
 \hskip\parindent
 \IconText{#1}{#2}}

\def\IconsText#1#2#3{
 \hskip-28.4mm
 \begin{minipage}{1.15\textwidth}
  \begin{tabular}{p{0.03\textwidth} p{0.1\textwidth} p{0.85\textwidth}}
   \includegraphics[width=7mm]{Figures/Icons/#1} &
   \includegraphics[width=7mm]{Figures/Icons/#2} &
   \raggedright\vskip-7mm #3
  \end{tabular}
 \end{minipage}}

\def\IconTextFigure#1#2#3{
 \hskip-20mm
 \begin{minipage}{1.1\textwidth}
  \begin{tabular}{p{0.1\textwidth} p{0.63\textwidth} p{0.25\textwidth}}
   \includegraphics[width=7mm]{Figures/Icons/#1} &
   \raggedright\vspace{-7mm} #2 &
   \vspace{-7mm}\includegraphics[width=0.25\textwidth]{#3}
  \end{tabular}
 \end{minipage}}

\def\GenericNote#1#2#3{
 \hskip-25mm
 \begin{minipage}{1.1\textwidth}
  \begin{tabular}{p{0.1\textwidth} p{0.9\textwidth}}
   \includegraphics[width=8mm]{Figures/#1} &
   \raggedright\vskip-7mm\sl\textbf{#2}: #3
  \end{tabular}
 \end{minipage}}

\def\GenericEnumNote#1#2#3{
 \hskip-28.8mm
 \begin{minipage}{1.1\textwidth}
  \begin{tabular}{p{0.17\textwidth} p{0.83\textwidth}}
   \includegraphics[width=8mm]{Figures/#1} &
   \raggedright\vskip-7mm\sl\textbf{#2}: #3
  \end{tabular}
 \end{minipage}}

% Even more generic, for use within minipages of various size
\def\MiniGenericNote#1#2#3#4#5#6#7{
 \hspace*{#3}
 \begin{minipage}{#4\textwidth}
  \begin{tabular}{p{#5\textwidth} p{#6\textwidth}}
   \includegraphics[width=8mm]{Figures/#1} &
   \raggedright\vskip-7mm\sl\textbf{#2}: #7
  \end{tabular}
 \end{minipage}}

\def\Tip#1{\GenericNote{tip}{TIP}{#1}}
\def\Note#1{\GenericNote{note}{NOTE}{#1}}
\def\Warning#1{\GenericNote{warning}{WARNING}{#1}}
\def\Important#1{\GenericNote{warning}{IMPORTANT}{#1}}
\def\Caution#1{\GenericNote{caution}{CAUTION}{#1}}
\def\EnumTip#1{\GenericEnumNote{tip}{TIP}{#1}}
\def\EnumNote#1{\GenericEnumNote{note}{NOTE}{#1}}
\def\EnumWarning#1{\GenericEnumNote{warning}{WARNING}{#1}}
\def\EnumCaution#1{\GenericEnumNote{caution}{CAUTION}{#1}}

\def\Chapter#1#2{\hypertarget{#2}{\chapter{#1}\label{#2}}}
\def\Section#1#2{\hypertarget{#2}{\section{#1}\label{#2}}}
\def\SubSection#1#2{\hypertarget{#2}{\subsection{#1}\label{#2}}}
\def\SubSubSection#1#2{\hypertarget{#2}{\subsubsection{#1}\label{#2}}}

\def\refAppendix#1#2{\protect\hyperlink{#1}{\sl Appendix~\ref{#1}, ''#2''}}
\def\refChapter#1#2{\protect\hyperlink{#1}{\sl Chapter~\ref{#1}, ''#2''}}
\def\refSection#1#2{\protect\hyperlink{#1}{\sl Section~\ref{#1}, ''#2''}}
\def\refSubSection#1#2#3{\protect\hyperlink{#1}{\sl"#2"} in
                         \protect\hyperlink{#3}{\sl Section~\ref{#3}}}

\def\Registered{\textsuperscript{\textregistered}}

\def\File#1{{\tt#1}}

\def\Bullet#1{
 \begin{picture}(10,10)
  \put(5,5){\circle*{10}}
  \put(2,2){\color{white}\tt#1}
 \end{picture}}
% Use this instead for double-digit bullets
\def\BBullet#1{
 \begin{picture}(10,10)
  \put(5,5){\circle*{10}}
  \put(1,3){\color{white}\tt\footnotesize#1}
 \end{picture}}
% Use this for single-digit bullet inlined in the text
\def\TextBullet#1{
 \begin{picture}(7,7)
  \put(3,3){\circle*{10}}
  \put(0,0){\color{white}\tt#1}
 \end{picture}}

% Define environment bulletlist which uses bullets on the item numbers
%---------------------------------------------------------------------
\newcommand*\circled[1]{%
 \tikz[baseline=(C.base)]
 \node[fill=black,draw,circle,inner sep=1.2pt,line width=0.2mm,](C){
  \color{white}\tt#1};}

\newenvironment{bulletlist}{
 \let\olditem\item
 \renewcommand\item[1][]{
  \stepcounter{enumi}
  \olditem[\circled{\theenumi}]##1}
 \renewcommand\subitem{\olditem[--]}
 \begin{enumerate}
 \setcounter{enumi}{0}}{
 \end{enumerate}}
%---------------------------------------------------------------------
