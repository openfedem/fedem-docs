% SPDX-FileCopyrightText: 2023 SAP SE
%
% SPDX-License-Identifier: Apache-2.0
%
% This file is part of FEDEM - https://openfedem.org

%%%%%%%%%%%%%%%%%%%%%%%%%%%%%%%%%%%%%%%%%%%%%%%%%%%%%%%%%%%%%%%%%%%%%%%%%%%%%%%%
%
% FEDEM Theory Guide.
%
%%%%%%%%%%%%%%%%%%%%%%%%%%%%%%%%%%%%%%%%%%%%%%%%%%%%%%%%%%%%%%%%%%%%%%%%%%%%%%%%

\section{History}

The theory behind the FEDEM solvers was originally developed by
the late Professor Ole Ivar Sivertsen in the late 1970s and through the 1980s.
His work initiated new Ph.D.\ studies and international R\&D projects
that contributed to the development of the first FEDEM software product.

On the basis of Professor Sivertsen's work, a company was established in 1992
by Sintef, Northern Europe's largest R\&D institute based in Trondheim, Norway.
Computer speed reached levels that would allow the theories to produce results,
and visualization technology made it possible to create a user interface.
During this period the FEDEM software was strictly an inhouse code at Sintef.

In 1995 the company Fedem, which later became Fedem Technology,
continued the development of the user interface and made it possible to offer
FEDEM as a commercial product in 1998.

The FEDEM software was continously developed as a product by Fedem Technology
during the 2000s and 2010s while it was also used as an internal tool in
various consultancy projects, until the company was aquired by SAP SE in 2016.
Since then, the FEDEM solvers have been provided as components in the
EPD Connected Products by SAP, until the sunsetting of the Connected Products
in 2023. It was then decided to release Fedem under a open source license
on github, as a service to the existing user community.
