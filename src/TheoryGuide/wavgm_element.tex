% SPDX-FileCopyrightText: 2023 SAP SE
%
% SPDX-License-Identifier: Apache-2.0
%
% This file is part of FEDEM - https://openfedem.org

%%%%%%%%%%%%%%%%%%%%%%%%%%%%%%%%%%%%%%%%%%%%%%%%%%%%%%%%%%%%%%%%%%%%%%%%%%%%%%%%
%
% FEDEM Theory Guide.
%
%%%%%%%%%%%%%%%%%%%%%%%%%%%%%%%%%%%%%%%%%%%%%%%%%%%%%%%%%%%%%%%%%%%%%%%%%%%%%%%%

\section{WAVGM}
\label{s:WAVGM}

WAVGM is an interpolation constraint element\footnote{
This element is known as RBE3 in Nastran.},
which defines the motion at a reference (slave) node as the weighted average of
the motions at a set of other (master) nodes.
The element topology is similar to that of the RGD-element,
see Figure~\ref{fig:WAVGM}, except that for WAVGM node 1 is a slave node
containing the dependent DOFs, whereas all other nodes are masters with
independent DOFs.

\begin{figure}[t]
\begin{center}
\setlength{\unitlength}{1mm}
\begin{picture}(100,75)
\thinlines
\put(15,15){\vector( 1, 0){70}}
\put(15,15){\vector( 0, 1){55}}
\put(15,15){\vector(-1,-1){12}}
\put( 0, 0){$x_{\text{link}}$}
\put(87,14){$y_{\text{link}}$}
\put(13,72){$z_{\text{link}}$}
\thicklines
\put(36,30){\circle*{2}} \put(31,30){$1_s$}
\put(80,20){\circle{2}}\put(82,19){$2_m$}
\put(70,40){\circle{2}}\put(71,36){$3_m$}
\put(60,60){\circle{2}}\put(62,60){$4_m$}
\put(30,50){\circle{2}}\put(29,53){$5_m$}
\qbezier[25](36,30)(58,25)(80,20)
\qbezier[20](36,30)(53,35)(70,40)
\qbezier[20](36,30)(48,45)(60,60)
\qbezier[12](36,30)(33,40)(30,50)
%Nodal dofs
\thinlines
\put(36,30){\vector(-1,-1){4.8}}\put(28,22.5){$u,\theta_x$}
\put(36,30){\vector(1,0){8}}    \put(45,29){$v,\theta_y$}
\put(36,30){\vector(0,1){8}}    \put(34,39){$w,\theta_z$}
\end{picture}
\end{center}
\caption{WAVGM, Multi-node weighted averaged motion element}
\label{fig:WAVGM}
\end{figure}

Unlike the RBAR and RGD elements described in the previous sections,
the WAVGM element does not add stiffness to the link, unless the slave node
already is connected to some of the master nodes via other finite elements.
Thus, the WAVGM element works like a force distributor; forces that are applied
at the reference (slave) node are distributed over the master nodes depending
on the given weighting factors and the relative distance to the reference node.

The manner in which the forces are distributed is analogous to the classical
bolt pattern analysis.
Consider a given force \bm{F} and a moment\bm{M} applied at the reference node
of the WAVGM element.
They are first replaced by an equivalent force $\widetilde{\bm{F}}=\bm{F}$
and moment $\widetilde{\bm{M}}=\bm{M}+\bm{F}\times\bm{e}$ at the weighted center
of gravity of the master nodes, where \bm{e} is the offset vector between the
reference node and the weighted center of gravity.
The force $\widetilde{\bm{F}}$ is then distributed to the master nodes
proportional to the given weighting factors, whereas the moment
$\widetilde{\bm{M}}$ is distributed as forces proportional to their distance
from the weighted center of gravity times their weighting factors.
Alternatively, the moment $\widetilde{\bm{M}}$ may be distributed directly as
moments at the master nodes (provided they are 6-DOF nodes) proportional to
their weighting factors, in the same manner as the force $\widetilde{\bm{F}}$.
This can be used if all master nodes of the element are co-linear, such that
they cannot absorb a moment though a set of forces.

The weighting factors can in principle be different for the various force
component at a given master node.
Thus, the force distribution is carried out on a component-by-component basis.
The resulting expression for the $i$'th force component at master
node number $a$ is then
%
\begin{equation}
\label{equ:WAVGM force distribution}
f_i^a \;=\; \frac{F_i\,\omega_i^a}{\sum_b\omega_i^b}
      \;+\; \frac{\left( M_j - F_k e_i + F_i e_k \right) \omega_i^a r_{ik}^a}
                 {\sum_b\left( {r_{ik}^b}^2 + {r_{ii}^b}^2 \right)\omega_i^b}
      \;-\; \frac{\left( M_k - F_i e_j + F_j e_i \right) \omega_i^a r_{ij}^a}
                 {\sum_b\left( {r_{ii}^b}^2 + {r_{ij}^b}^2 \right)\omega_i^b}
\end{equation}
%
where $(i,j,k)$ forms a cyclic permutation of the components $x,y,z$,
and $r_{ij}^a$ denotes the $j$'th component of the relative position vector
$\bm{r}_i^a$ from the weighted center of gravity to the $a$'th master node,
based on the weighting factors $\omega_i^a$:
%
\begin{equation}
\bm{r}_i^a = \bm{x}^a - \frac{\sum_b\omega_i^b \bm{x}^b}{\sum_b\omega_i^b}
\end{equation}
%
Alternatively, when the master nodes are co-linear such that the moment has to
be distributed directly, the force components at master node $a$ are given by
only the first term of Equation~(\ref{equ:WAVGM force distribution}), whereas
the moment components are
%
\begin{equation}
\label{equ:WAVGM moment distribution}
m_i^a \;=\; \frac{\left( M_i - F_j e_k + F_k e_j \right) \omega_{3+i}^a}
                 {\sum_b\omega_{3+i}^b}
\end{equation}

The above expressions are now used to establish the governing constraint
equations for the WAVGM element, which are used to eliminate the slave node DOFs
in the global system of equations of the link FE model.
Let ${\bf R}_m$ and ${\bf R}_s$ denote vectors that collect force components at
all master DOFs and slave DOFs, respectively, in the element.
Similarly, let ${\bf r}_m$ and ${\bf r}_s$ denote the associated displacement
 vectors.
The master and slave components are then related through
%
\begin{eqnarray}
{\bf r}_s &=& {\bf T}_c   \;{\bf r}_m \label{equ:WAVGM constraints}\\
{\bf R}_m &=& {\bf T}_c^T \:{\bf R}_s
\end{eqnarray}
%
The row of ${\bf T}_c^T$ (column of ${\bf T}_c$) corresponding to a given
slave DOF is then obtained by in inserting a unit value for $F_x,F_y,F_z$,
$M_x,M_y,M_z$, respectively, in turn while letting the other components be zero.

It is clear that for some WAVGM element geometries the denominators of
Equation~(\ref{equ:WAVGM force distribution}) might be small, or even zero.
For instance, for a three-noded element where all nodes lie on a line that is
parallel to the global x-axis, the denominator of the first term is zero when
$i=3$.
The size of the denominator is therefore checked against a threshold value,
and the resulting constraint coefficient is omitted if the denominator is
smaller than this threshold. These checks are performed as follows for the
two terms:
%
\begin{eqnarray}
\sum_b\left( {r_{ik}^b}^2 + {r_{ii}^b}^2 \right)\omega_i^b &>&
\left(\max_b \|r_{ik}^b\|\,\epsilon_{\rm tol} \right)^2 +
\left(\max_b \|r_{ii}^b\|\,\epsilon_{\rm tol} \right)^2 \\[1mm]
\sum_b\left( {r_{ii}^b}^2 + {r_{ij}^b}^2 \right)\omega_i^b &>&
\left(\max_b \|r_{ii}^b\|\,\epsilon_{\rm tol} \right)^2 +
\left(\max_b \|r_{ij}^b\|\,\epsilon_{\rm tol} \right)^2
\end{eqnarray}
%
where $\epsilon_{tol}$ is a relative tolerance parameter that may be set by the
user through the command-line option {\tt -tolWAVGM} of the Fedem Link Reducer
(default value $=10^{-4}$).
Thus, constraint coefficients are added only for those terms satisfying the
above conditions.

It should be emphasized that the constraints given by
Equation~(\ref{equ:WAVGM constraints}) are enforced in strong form in Fedem
(the same is true for the RGD and RBAR elements as well).
This implies that a WAVGM slave node can not be a triad (external node) in Fedem.
Moreover, WAVGM elements where the slave node already is connected to the master
nodes trough other finite elements should be used with caution.
Such element may result in an over-constrained system of equations, such that
the resulting reduced link does not possess the necessary 6 rigid body modes.
This may in turn make the dynamics simulation unstable.
