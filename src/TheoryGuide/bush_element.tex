% SPDX-FileCopyrightText: 2023 SAP SE
%
% SPDX-License-Identifier: Apache-2.0
%
% This file is part of FEDEM - https://openfedem.org

%%%%%%%%%%%%%%%%%%%%%%%%%%%%%%%%%%%%%%%%%%%%%%%%%%%%%%%%%%%%%%%%%%%%%%%%%%%%%%%%
%
% FEDEM Theory Guide.
%
%%%%%%%%%%%%%%%%%%%%%%%%%%%%%%%%%%%%%%%%%%%%%%%%%%%%%%%%%%%%%%%%%%%%%%%%%%%%%%%%

\section{BUSH}
\label{s:BUSH}

BUSH is a 2-node generalized spring element of zero length.
The two element nodes may, or may not be coincident.
The spring element is positioned independently of the two element nodes, and is
connected to these nodes via rigid arms, as shown in Figure~\ref{fig:BUSH}.
Nominal stiffness values for the 3 translational and 3 rotational DOFs are given
in a local element coordinate system.
This local system is either specified explicitly as an element property,
or defined implicitly through the two nodes and a given auxiliary point.
The implicit definition is similar to that of the local coordinate system for a
beam element (see Section~\ref{s:BEAM2}).
Thus, this definition is not applicable if the two element nodes are coincident.
The BUSH element does not contribute to the mass matrix.

\begin{figure}[b]
\begin{center}
\setlength{\unitlength}{1mm}
\begin{picture}(100,75)
\thinlines
\put(15,15){\vector(1,0){70}}
\put(15,15){\vector(0,1){55}}
\put(15,15){\vector(-1,-1){12}}
\put( 0, 0){$x_{\text{link}}$}
\put(87,14){$y_{\text{link}}$}
\put(13,72){$z_{\text{link}}$}
\thicklines
\put(26,20){\circle{2}}\put(22,20){$I$}
\put(70,40){\circle{2}}\put(66,40){$J$}
\put(50,50){\circle*{2}}
%Eccentric arms
\thinlines
\put(50,50){\line(-1,-1){10}}\put(45,42){$e_x$}
\put(40,40){\line(-1, 0){14}}\put(33,37){$e_y$}
\put(26,40){\line( 0,-1){20}}\put(27,30){$e_z$}
\qbezier[8](50,50)(52,52)(55,55)
\qbezier[16](55,55)(62,55)(70,55)
\qbezier[16](70,55)(70,48)(70,40)
\put(71,50){\scriptsize fictitious rigid arm}
%Local system
\thinlines
\put(50,50){\vector( 2, 1){20}}\put(70.5,60){$x_\text{elem}$}
\put(50,50){\vector(-1, 2){4}} \put(42,60){$y_\text{elem}$}
\put(50,50){\vector( 3,-1){10}}\put(60.5,46){$z_\text{elem}$}
\end{picture}
\end{center}
\caption{BUSH, Generalized spring element}
\label{fig:BUSH}
\end{figure}

The BUSH element may also be specified without any stiffness properties,
but only the element topology.
Such elements are automatically created during modeling in Fedem,
for instance when a mechanism joint is attached to a link at a slave node of an
RGD, RBAR or WAVGM element (see the \FedemUG, {\em Section~3.6,
``Attaching and detaching elements''}).
A property-less BUSH element is the created as the connection between this slave
node and an added added external node (triad) at the same location.

The nominal stiffness values for the property-less BUSH element are computed
automatically based on the overall stiffness properties of the link\footnote{
Unless the link is completely rigid, e.g., it consists of a single RGD element.
In that case $k_t = k_r = 2.0 \cdot 10^{11}$ is used instead.}.
For the translational and rotational stiffnesses, $k_t$ and $k_r$, respectively,
the following three alternative procedures are available:
%
\begin{eqnarray}
k_t \;=\; \frac{0.1}{\varepsilon_{\rm sing}}
          \min\left\{{\rm diag}({\bf K}_{\rm tra})\right\} &,&
k_r \;=\; \frac{0.1}{\varepsilon_{\rm sing}}
          \min\left\{{\rm diag}({\bf K}_{\rm rot})\right\} \\
%
k_t \;=\; C_s \frac{{\rm tr}({\bf K}_{\rm tra})}{n_{\rm tra}} \hskip47pt &,&
k_r \;=\; C_s \frac{{\rm tr}({\bf K}_{\rm rot})}{n_{\rm rot}} \\
%
k_t \;=\; C_s \max\left\{{\rm diag}({\bf K}_{\rm tra})\right\} \hskip7pt &,&
k_r \;=\; C_s \max\left\{{\rm diag}({\bf K}_{\rm rot})\right\} \label{eq:defB}
\end{eqnarray}
%
Here, ${\bf K}_{\rm tra}$ and ${\bf K}_{\rm rot}$ are the translational and
rotational parts, respectively, of the fully assembled link stiffness matrix,
and $n_{\rm tra}$ and $n_{\rm rot}$ denote the total number of translational
and rotational DOFs, respectively.
Moreover, diag($\cdot$) denotes the diagonal elements of a given matrix,
whereas tr($\cdot$) is the trace operator (sum of diagonal elements).
Finally, $\varepsilon_{\rm sing}$ denotes the {\em singularity criterion}
used by the Fedem Link Reducer when factoring the link stiffness matrix
(specified by the user through the Link property panel,
see the \FedemUG, {\em Section~4.1.5,``Link properties''}),
and $C_s$ is a user-defined scaling factor that may be specified through the
command-line option {\tt -autoStiffScale} when running the Link Reducer
(default is $10^2$).

The wanted procedure is selected through the command-line option
{\tt -autoStiffMethod}.
The default is to use \eqnref{eq:defB}.
