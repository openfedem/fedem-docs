% SPDX-FileCopyrightText: 2023 SAP SE
%
% SPDX-License-Identifier: Apache-2.0
%
% This file is part of FEDEM - https://openfedem.org

%%%%%%%%%%%%%%%%%%%%%%%%%%%%%%%%%%%%%%%%%%%%%%%%%%%%%%%%%%%%%%%%%%%%%%%%%%%%%%%%
%
% FEDEM Theory Guide.
%
%%%%%%%%%%%%%%%%%%%%%%%%%%%%%%%%%%%%%%%%%%%%%%%%%%%%%%%%%%%%%%%%%%%%%%%%%%%%%%%%

\section{Energy calculations}

Energy calculations are described by the following terms:

\begin{namelist}{$U_{sum}$}

\item[$U_\varepsilon$]
Strain energy,  computed for all links (superelements), springs, 
and the system (the system is the sum of all mechanisms)

\item[$U_{k}$]
Kinetic energy,  computed for all links, 
discrete masses, rotational inertias, and the system

\item[$U_{p}$]
Potential energy,  computed for all links, 
discrete masses, and the system

\item[$U_{i}$]
Input energy,  computed for all external forces, 
springs (contribution from non-constant, stress-free length), and the system

\item[$U_{d}$]
Energy loss,  computed for all links (structural 
damping), axial dampers, joint dampers, joint friction, and the system

\item[$U_{e}$]
External energy, computed for all external ``elements'', such as tires, that are using
$3^\text{rd}$ party software modules. The energy can be one of (or a mix of) strain,
damping, and other energy terms.

\item[$U_{sum}$]
Energy check-sum, computed for the system
\end{namelist}

\noindent
All these terms are can be plotted as functions of time or other 
variables.

\subsection{Strain energy}

The total strain energy for the system is computed as a sum of all the 
element strain energies and spring strain energies.

\subsubsection{Link strain energy}

The strain energy for one link with linearly elastic 
material is given by


\begin{equation}
U_{\varepsilon } = \frac{{1}}{{2}}{\mf v}_{d}^{T} {\mf K}{\mf v}_{d}
\label{eqEC:81}
\end{equation}

\noindent
where ${\mf v}_{d}$ is the deformational displacement of the 
element, and ${\mf K}$ is the element stiffness matrix. The strain 
energy is thus computed on a total form at each converged time-step; in other words, no 
storage of previous strain energy is necessary.

\subsubsection{Spring strain energy}

Even the nonlinear springs have hyper-elastic behavior, and the strain 
energy is computed totally at each converged time-step.
%
\begin{eqnarray}
\mbox{Linear springs :\ \ } && U_{\varepsilon} = \frac{{1}}{{2}}k_{s} u^{2} \\
\mbox{Nonlinear springs:\ \ } && U_{\varepsilon } = \int_{0}^{u} f_{s} \left( 
{\hat u} \right) d{\hat u}  \nonumber
\label{eqEC:82}
\end{eqnarray}
%
\noindent
As the expression for the hyper-elastic, nonlinear spring energy also 
includes the linear spring element, it is used for all spring elements, linear and nonlinear,
and for axial and joint springs. Incremental calculations of the spring strain energy are not
necessary as long as only hyper-elastic nonlinear springs are included in the model.

\subsection{Kinetic energy}

System kinetic energy consists of the contributions from all the links, all 
the lumped masses, and all the discrete rotational inertias.

\subsubsection{Link kinetic energy}

The link kinetic energy is calculated from the superelement 
mass matrix, ${\mf M}$ and the superelement velocity vector, 
$\dot{\mf v}$, as:
%
\begin{equation}
U_{k} = \frac{{1}}{{2}} \dot{\mf v}^{T}{\mf M}
\dot{\mf v} .
\label{eqEC:84}
\end{equation}
%
\noindent
In other words, it is also computed on a total form for each time-step.

\subsubsection{Kinetic energy from discrete masses and inertias}

Discrete masses and rotational inertias contribute to the kinetic energy as:
%
\begin{equation}
U_{k} = \frac{{1}}{{2}}m \dot{\mf u}^{T} 
\dot{\mf u} + \frac{{1}}{{2}}{\tf{\omega}}^{T}{\mf I}_{\omega} \tf\omega ,
\label{eqEC:85}
\end{equation}
%
\noindent
where ${m}$ is the lumped mass and ${\mf I}_\omega $ is the 
rotational inertia matrix for the lumped mass.

\subsection{Potential energy}

As is the case with the kinetic energy, the system's potential energy is the sum off all 
the contributions from the links and the discrete masses. However, the 
discrete rotational inertias do not contribute to the potential energy.

\subsubsection{Link potential energy}

Computation of the potential energy should reflect the changes in potential 
energy rather than the potential energy in relation to the coordinate system 
used for modeling. Choosing a coordinate system which gives large potential energies may hide
other energy contributions. 
By subtracting the potential energy of the masses in their 
initial position ${C_0}$ from the potential energy at the present position ${C_n}$, the initial

potential energy of all masses, superelement masses, and lumped masses is zero.
%
\begin{equation}
U_{p} = m{\mf g}^{T}\left( {\mf x}_{C_n} - {\mf x}_{C_0}  \right)
\label{eqEC:86}
\end{equation}
%
\noindent
Calculations of the potential energy of the links use the displacements of 
the link centroid. This calculation neglects the relative displacement of 
the centroid due to internal deformations of the superelements; however 
this is justified by assuming small deformations. The 
calculation of the potential energy is performed entirely without 
storing energies from previous steps.

\subsection{Input energy}

Input energy to the system consists of contributions from external forces 
and springs that have non-constant, stress-free length.

\subsubsection{Input energy from external forces}

Since the external forces are non-conservative (i.e., can have a time variation 
and co-rotational behavior), the input energy from the external forces must 
be computed in an incremental manner from one time-step to the next:
%
\begin{equation}
 U_{i} = \int_{0}^{t} {{\mf F}^T\!\left( { {\hat t} }\right) \dot{\mf u} \,d{\hat t} 
\approx \sum_{k = 1}^{n} {\overline {{\mf F}}_{k}^{T}\Delta {\mf u}_{k} } } 
 \mbox{\ \ \ \ where\ \ \ \ }
\overline{{\mf F}}_{k}  = \frac{{1}}{{2}}\left( {\mf F}_{k - 1} + {\mf F}_{k}  \right)
\label{eqEC:87}
\end{equation}
%
Subscript $k$ is the time-step index. The summation expression 
above represents a trapezoidal integration scheme.

\subsubsection{Input energy from springs}

The stress-free lengths of the springs (both axial springs and joint springs) 
are subject to possible change. When the spring is 
stressed, this represents an input energy contribution. This input energy 
must be computed incrementally for all the springs:
%
\begin{equation}
 U_{i} = 
\int_{0}^{t} f_{s}({\hat t}) l_{0} \:d{\hat t} \approx \sum_{k = 1}^{n} 
\overline{f}_{s_{k}}  \Delta l_{0}  
 \mbox{\ \ \ \ where\ \ \ \ }
\overline {f}_{s_{k}} = 
\frac{{1}}{{2}}(f_{s_{k-1}} + f_{s_{k}})
\label{eqEC:88}
\end{equation}

\subsection{Energy loss}

Total energy loss consists of the contributions from structural damping 
in the links and discrete dampers (both axial dampers and joint dampers), and 
energy loss from joint friction.

\subsubsection{Energy loss from link structural damping}

The structural damping of the links is composed of mass and stiffness 
proportional damping  (${\mf C} = \alpha_{1}{\mf M}+\alpha_{2}{\mf K}$),
which is used in the following damping energy expression for a link
%
\begin{equation}
U_{d} = \int_{0}^{t} { \dot{\mf v}^{T}{\mf C}
\dot{\mf v} d {\hat t}} 
\approx \sum_{k = 1}^{n} h\alpha _{1}  \bar{\dot{{\mf v}}}_k^T{\mf M}\bar{\dot{{\mf v}}}_k 
+ \sum_{k = 1}^{n} h\alpha _{2}  {\dot{\mf v}{}_d}_{k}^T {\mf K}  {\dot{\mf v}{}_d}_k
\label{eqEC:89}
\end{equation}
%
\noindent
where 
$ \bar{\dot{{\mf v}}}_{k}^{T} = \frac{{1}}{{2}}\left(\dot{\mf v}_{k-1} 
                                           + \dot{\mf v}_{k}  \right)$,
and 
$ {\dot{\mf v}{}_d}_k = \frac{{1}}{{h}}\left( {{\mf v}_d}_{k} 
                                            - {{\mf v}_d}_{k-1} \right)$,
and 
${h}$ represent the time-step size.
%
Note that the deformational velocities, $ {\dot {\mf v}_d} $, are used
when computing the stiffness proportional damping 
energy. This is important to avoid false damping energies from 
rigid-body velocities when using large time-steps.

\subsubsection{Energy loss from discrete dampers}

Energy loss from discrete dampers (axial dampers and joint dampers) is 
computed as
%
\begin{equation}
U_{d} = \int_{0}^{t} f_{d}  {\dot u} \:dt \approx 
\sum_{k = 1}^{n} {\overline{f}{}_d}_k  \Delta u  
\label{eqEC:810}
\end{equation}
%
\noindent
where 
${\overline{f}{}_d}_k = \frac{1}{2}\left( {f_d}_{k-1} + 
{f_d}_k \right)$

\subsubsection{Energy loss from friction}

Friction energy loss is computationally analogous to the damping loss
%
\begin{equation}
U_{d} = \int_{0}^{t} f_{f}\:dt \approx \sum_{k=1}^{n} {\overline{f}{}_f}_k  \Delta u_{k}   
\end{equation}
%
\noindent
where 
${\overline{f}{}_f}_k = \frac{1}{2}\left( {f_f}_{k-1} + 
{f_f}_k \right)$ 

\subsection{External energy}

External energy from 3rd party components such as tires is calculated
is calculated through the incremental work performed at the component interface
from one time-step to the next:
%
\begin{equation}
 U_{e} = \int_{0}^{t} {{\mf F}^T\!\left( { {\hat t} }\right) \dot{\mf u} \:d{\hat t} 
\approx \sum_{k = 1}^{n} {\overline {{\mf F}}_{k}^{T}\Delta {\mf u}_{k} } } 
 \mbox{\ \ \ \ where\ \ \ \ }
\overline{{\mf F}}_{k}  = \frac{{1}}{{2}}\left( {\mf F}_{k - 1} + {\mf F}_{k}  \right)
\label{eqEC:83}
\end{equation}
%
Subscript $k$ is the time-step index. The summation expression 
above represents a trapezoidal integration scheme. The forces $\mf F$ from the
component can have contribution from both stiffness and damping forces, 
and the energy contribution is thus a mixed energy. 


\subsection{Energy check-sum}

An energy check-sum is computed for the system to verify that the 
system energy is preserved:
%
\begin{equation}
U_{\text{sum}} = {U_p}_{\text{system}}  + {U_k}_{\text{system}}  
               + {U_\varepsilon}_{\text{system}}  
               + {U_d}_{\text{system}} - {U_i}_{\text{system}} - {U_e}_{\text{system}}
\label{eqEC:811}
\end{equation}
%
\noindent
The check-sum should remain constant during time integration.
