% SPDX-FileCopyrightText: 2023 SAP SE
%
% SPDX-License-Identifier: Apache-2.0
%
% This file is part of FEDEM - https://openfedem.org

%%%%%%%%%%%%%%%%%%%%%%%%%%%%%%%%%%%%%%%%%%%%%%%%%%%%%%%%%%%%%%%%%%%%%%%%%%%%%%%%
%
% FEDEM Theory Guide.
%
%%%%%%%%%%%%%%%%%%%%%%%%%%%%%%%%%%%%%%%%%%%%%%%%%%%%%%%%%%%%%%%%%%%%%%%%%%%%%%%%

\chapter{Fundamentals}
\label{c:Fundamentals}

%%%%%%%%%%%%%%%%%%%%%%%%%%%%%%%%%%%%%%%%%%%%%%%%%%%%%%%%%%%%%%%%%%%%%%%%%%%%%%%%
\section{Notation}
\label{s:Notation}
%%%%%%%%%%%%%%%%%%%%%%%%%%%%%%%%%%%%%%%%%%%%%%%%%%%%%%%%%%%%%%%%%%%%%%%%%%%%%%%%

Matrices and matrix vectors (one column matrices), are denoted in text as
upright bold symbols, such as $\mf K$ and $\mf v$.
Tensors are denoted in italicized bold symbols, such as ${\tf a}=a_i{\tf i}_i$.

Given two coordinate systems with the three orthonormal base vectors
$({\tf I}_1 {\tf I}_2 {\tf I}_3)$ and $({\tf i}_1 {\tf i}_2 {\tf i}_3)$,
system ${\tf I}_i$ is called the \textit{global system}, whereas system
${\tf i}_i$ is called the \textit{local system}.

Using \textit{tensor notation}, a vector (first order tensor) $\tf a$
can be represented as
%
\begin{equation}
\eqalign{
{\tf a} \;=\; & a_1{\tf I}_1 + a_2{\tf I}_2 + a_3{\tf I}_3 \;=\;
\sum_{i=1}^3 a_i{\tf I}_i \;=\; a_i{\tf I}_i \cr =\; &
{\tilde a}_1{\tf i}_1 + {\tilde a}_2{\tf i}_2 + {\tilde a}_3{\tf i}_3 \;=\;
\sum_{i=1}^3 {\tilde a}_i{\tf i}_i \;=\; {\tilde a}_i{\tf i}_i}
\end{equation}
%
with respect to the two coordinate systems.

Using \textit{matrix notation}, this same vector will be represented as $\mf a$
and $\tilde{\mf a}$ in the global and local coordinate systems respectively:
%
\begin{equation}
{\mf a} \;= \left[\begin{array}{c}
 a_1 \\ a_2 \\ a_3
\end{array}\right]
\quad\mbox{and}\quad
\tilde{\mf a} \;= \left[\begin{array}{c}
\tilde a_1 \\ \tilde a_2 \\ \tilde a_3
\end{array}\right]
\end{equation}

%%%%%%%%%%%%%%%%%%%%%%%%%%%%%%%%%%%%%%%%%%%%%%%%%%%%%%%%%%%%%%%%%%%%%%%%%%%%%%%%
\section{Rigid-body motion}
\label{s:Rigid-body motion}
%%%%%%%%%%%%%%%%%%%%%%%%%%%%%%%%%%%%%%%%%%%%%%%%%%%%%%%%%%%%%%%%%%%%%%%%%%%%%%%%

The position and orientation in space of a rigid body, $\cal S$, can be
represented by the position and orientation of a fixed coordinate system.
This rigid body can be a link or a superelement.
This section addresses the rigid-body motion of the superelement.

The vector $\mf s$ represents the position of the fixed coordinate system
in relation to the global coordinate system.
The fixed coordinate system is defined by the three orthonormal base vectors
$({\tf i}_1,{\tf i}_2,{\tf i}_3)$, which comprise the rotation matrix:
%
\begin{eqnarray}
\makebox[28mm]{Position:\hfill} {\mf s} &=&
\left[\begin{array}{c} s_x\\ s_y\\ s_z \end{array}\right]
\label{eq:posVec} \\
\makebox[30mm]{Orientation:} {\mf R}_S &=&
\left[\begin{array}{ccc} {\mf i}_1 & {\mf i}_2 & {\mf i}_3 \end{array}\right] =
\left[\begin{array}{ccc}
i_{1x} & i_{2x} & i_{3x} \\
i_{1y} & i_{2y} & i_{3y} \\
i_{1z} & i_{2z} & i_{3z}
\end{array}\right]
\label{eq:rotMat}
\end{eqnarray}

It is frequently necessary to describe the motion within the rigid body $\cal S$
in, for example, joint attachment points.
Such points have a position relative to the rigid body coordinate system.
The joint itself has an orientation relative to the rigid body's orientation.
Defining the point $\cal N$ (a FE node, for example) within the rigid body with
coordinates $\tilde{\mf a}$ relative to the body's fixed coordinate system,
a local coordinate system, $\tilde{\mf R}_N$, relative to the body-fixed
system is represented by
%
\begin{eqnarray}
\makebox[28mm]{Position:\hfill} \tilde{\mf a} &=&
\left[\begin{array}{c}
\tilde{a}_x \\ \tilde{a}_y \\ \tilde{a}_z \end{array}\right] \\
\makebox[30mm]{Orientation:} \tilde{\mf R}_N &=&
\left[\begin{array}{ccc} \tilde{\mf i}_1 & \tilde{\mf i}_2 & \tilde{\mf i}_3
\end{array}\right] = \left[\begin{array}{ccc}
\tilde{i}_{1x} & \tilde{i}_{2x} & \tilde{i}_{3x} \\
\tilde{i}_{1y} & \tilde{i}_{2y} & \tilde{i}_{3y} \\
\tilde{i}_{1z} & \tilde{i}_{2z} & \tilde{i}_{3z}
\end{array}\right]
\end{eqnarray}

The point $\cal N$ has a motion in the global system produced as a result of the
motion of the body-fixed coordinate system.
With $\mf a$ being the coordinates of the point relative to the global system,
we have
%
\begin{equation}
{\mf a} \;=\; {\mf R}_S\tilde{\mf a} + {\mf s}
\end{equation}
%
The orientation of the point in the local coordinate system, $\tilde{\mf R}_N$,
will then be as follows in the global system:
%
\begin{equation}
{\mf R}_N \;=\; {\mf R}_S \tilde{\mf R}_N
\end{equation}

The above equations can be combined by using $4\times4$ transformation matrices
containing the position and rotation as defined by \eqsref{eq:posVec}{eq:rotMat}
%
\begin{equation}
{\mf P}_S \;=\;
\left[\begin{array}{cc} {\mf R}_S & {\mf s}\\ {\mf 0} & 1 \end{array}\right] =
\left[\begin{array}{cccc}
i_{1x} & i_{2x} & i_{3x} & s_x \\
i_{1y} & i_{2y} & i_{3y} & s_y \\
i_{1z} & i_{2z} & i_{3z} & s_z \\
0 & 0 & 0 & 1
\end{array}\right]
\label{eq:posDef}
\end{equation}
%
and
%
\begin{equation}
\tilde{\mf P}_N \;=\;
\left[\begin{array}{cc}
\tilde{\mf R}_N & \tilde{\mf a} \\
{\mf 0} & 1 \end{array}
\right] = \left[\begin{array}{cccc}
\tilde{i}_{1x} & \tilde{i}_{2x} & \tilde{i}_{3x} & \tilde{a}_x \\
\tilde{i}_{1y} & \tilde{i}_{2y} & \tilde{i}_{3y} & \tilde{a}_y \\
\tilde{i}_{1z} & \tilde{i}_{2z} & \tilde{i}_{3z} & \tilde{a}_z \\
0 & 0 & 0 & 1
\end{array}\right]
\end{equation}

The position and orientation of point $\cal N$ in the global system
can then be expressed as
%
\begin{equation}
{\mf P}_N \;=\; {\mf P}_S\tilde{\mf P}_N \;=\;
\left[\begin{array}{cc} {\mf R}_N & {\mf a} \\ {\mf 0} & 1 \end{array}\right]
\label{eq:posUpdate}
\end{equation}
%
\remark{In computer implementations, the $4\times4$ transformation matrices
are represented as $3\times4$ matrices obtained by deleting the fourth of the
$4\times4$ matrix as in.
The matrix multiplication of \eqnref{eq:posUpdate} is then modified by taking
into account the implicitly defined $[0\;0\;0\;1]$ fourth row of the matrix.}

%%%%%%%%%%%%%%%%%%%%%%%%%%%%%%%%%%%%%%%%%%%%%%%%%%%%%%%%%%%%%%%%%%%%%%%%%%%%%%%%
\section{Finite rotations}
\label{s:Finite rotations}
%%%%%%%%%%%%%%%%%%%%%%%%%%%%%%%%%%%%%%%%%%%%%%%%%%%%%%%%%%%%%%%%%%%%%%%%%%%%%%%%

\subsection{Spin of a matrix}

The cross product of vectors is written as
%
\begin{equation}
{\tf c} \;=\; {\tf a}\times{\tf b}
\end{equation}
%
when using tensor notation.
Using matrix notation this can be written
%
\begin{equation}
{\mf c} \;=\; \widehat{\mf a}\,{\mf b}
\end{equation}
%
where
%
\begin{equation}
\widehat{\mf a} \;=
\left[\begin{array}{ccc}
   0 &  a_z & -a_y \\
-a_z &    0 &  a_x \\
 a_y & -a_x &    0
\end{array}\right]
\quad\text{when}\quad
{\mf a} \;= \left[\begin{array}{c}
a_x \\ a_y \\ a_z
\end{array}\right]
\label{eqF:SpinDef}
\end{equation}
%
The above equations define the \textit{Spin} of a vector.
This notation is frequently used in the matrix expressions in the following.

\subsection{Rotation of a vector}

Any rotation of a vector from its original direction $\mf a$ to its rotated
direction ${\mf a}^\prime$ can be described by the general relationship
%
\begin{equation}
{\mf a}^\prime \;=\; {\mf R}{\mf a}
\end{equation}
%
where $\mf R$ is a {\em orthonormal} $3\times3$ matrix.
A matrix is orthonormal when each row and column represented as a vector is
orthogonal to all the other rows and columns of the matrix, and has a unit length.
Since $\mf R$ is orthonormal, the inverse relationship becomes
%
\begin{equation}
{\mf a} \;=\; {\mf R}^{-1}{\mf a}^\prime \;=\; {\mf R}^T{\mf a}^\prime
\end{equation}
%
A rotation matrix $\mf R$ can be established in a number of ways,
referred to as different {\em parameterizations} of the rotation.
The rotation matrix is also often called rotation {\em tensor},
as it has properties of a second order tensor.

\subsection{Rodriguez parameterization}

Rodriguez parameterization states that any combination of rotations can be
represented by a single rotation $\theta$ about a single rotational axis
defined by a unit vector $\tf n$.
The vector is defined as
%
\begin{equation}
{\tf\theta} \;=\; \theta{\mf n}
\end{equation}
%
and the rotation matrix can be written as
\begin{equation}
{\mf R} \;=\; {\mf R}({\tf\theta}) \;=\;
{\mf I} + \frac{\sin\theta}{\theta}\widehat{\tf\theta} +
\frac{1}{2}\left(\frac{\sin\frac{\theta}{2}}{\frac{\theta}{2}}\right)^2
\widehat{\tf\theta}^2
\label{eq:RodriguesDef}
\end{equation}

\subsection{Variation of Rodriguez parameterization}

The variation of the instantaneous rotation axis $\tf\omega$ with respect to the
finite rotations $\tf\theta$ of the Rodriguez parameterization can be written as
%
\begin{equation}
\delta{\mf \omega} \;=\;
\frac{\partial\tf\omega}{\partial\tf\theta} \delta{\tf\theta} \;=\;
{\mf H}({\tf\theta}) \delta{\tf\theta}
\label{eq:DomegaDthetaDef}
\end{equation}
%
where
%
\begin{equation}
{\mf H}({\tf \theta}) \;=\;
\frac{1}{\theta^2}{\tf\theta}{\tf\theta}^T + \frac{\sin\theta}{\theta}
\left({\mf I}-\frac{1}{\theta^2}{\tf\theta}{\tf\theta}^T\right) +
\frac{1-\cos\theta}{\theta^2}
{\mf\widehat{\tf\theta}}
\label{eq:DomegaDtheta}
\end{equation}
%
\remark{Attention should be paid to the $0/0$ terms of \eqnref{eq:DomegaDtheta}
when calculating ${\mf H}({\theta})$ for very small values of $\theta$.}

\subsection{Euler angles parameterization}
\label{subs:Euler angles}

The Euler angle parameterization actually refers to several parameterizations
of a finite rotation.
The total rotation is defined by a series of rotations about subsequent follower axes.
To establish the relationship for the follower axes we first establish
a relationship for subsequent rotations about rigid (global) axes.

\subsubsection{Sequential rotations about rigid axes}

A vector ${\mf a}_0$ is first rotated an angle $\theta_x$ about the global
$X$-axis to obtain vector ${\mf a}_1$.
The vector ${\mf a}_1$ is then rotated an angle $\theta_y$ about the global
$Y$-axis to obtain vector ${\mf a}_2$.
Finally, the vector ${\mf a}_2$ is rotated an angle $\theta_z$ about the global
$Z$-axis to obtain vector ${\mf a}_3$.
This yields the equations:
%
\begin{eqnarray}
{\mf a}_1 &=& {\mf R}_x{\mf a}_0 \\
{\mf a}_2 &=& {\mf R}_y{\mf a}_1 \;=\;
{\mf R}_y{\mf R}_x{\mf a}_0 \;=\; {\mf R}_{Rxy}{\mf a}_0
\label{eqF:RigAxisRot} \\
{\mf a}_3 &=& {\mf R}_z{\mf a}_2 \;=\;
{\mf R}_z{\mf R}_y{\mf R}_x{\mf a}_0 \;=\; {\mf R}_{Rxyz}{\mf a}_0
\label{eqF:RigidXYZ}
\end{eqnarray}
%
where ${\mf R}_x$, ${\mf R}_y$, and ${\mf R}_z$ are established
from \eqnref{eq:RodriguesDef} by inserting ${\tf\theta}^T = [\theta_x\;0\;\;0]$,
${\tf\theta}^T = [0\;\;\theta_y\;0]$ and ${\tf\theta}^T = [0\;\;0\;\;\theta_x]$,
respectively.

\subsubsection{Sequential rotations about follower axes}

When rotating a vector ${\mf a}_0$ with rotation matrix ${\mf R}_x$,
a new coordinate system consisting of the global axes rotated with matrix
${\mf R}_x$, can be obtained.

\remark{The columns of ${\mf R}_x$ form the basis vectors of the new coordinate system.}

\noindent
As vector ${\mf a}_0$ rotates with the coordinate system to form vector
${\mf a}_1$, the vector has not changed in the co-rotated coordinate system:
%
\begin{equation}
\tilde{\mf a}_1 \;=\; {\mf a}_0
\end{equation}
%
where superimposed $\tilde{}$ is used to designate a vector in the local
(co-rotated) coordinate system of ${\mf R}_x$.

A second rotation of the vector ${\mf a}_1$, an angle $\theta_y$ about the new
local $y$-axis, is established in the co-rotated coordinate system
%
\begin{equation}
\tilde{\mf a}_2 \;=\; {\mf R}_y\tilde{\mf a}_1 \;=\; {\mf R}_y{\mf a}_0
\end{equation}
%
where matrix ${\mf R}_y$ is identical to the matrix established
in \eqnref{eqF:RigAxisRot}.
Transforming the vector $\tilde{\mf a}_2$ back to the global coordinate system
can be achieved by the relationship
%
\begin{equation}
{\mf a}_2 \;=\; {\mf R}_x\tilde{\mf a}_2 \;=\;
{\mf R}_x{\mf R}_y{\mf a}_0 \;=\; {\mf R}_{Fxy}{\mf a}_0
\end{equation}

A third rotation about the new $z$-axis of the second updated coordinate system
is established in the local coordinate system
%
\begin{equation}
\tilde{\tilde{\mf a}}_3 \;=\; {\mf R}_z\tilde{\tilde{\mf a}}_2 \;=\;
{\mf R}_z{\mf a}_0
\quad\text{since}\quad
\tilde{\tilde{\mf a}}_2 \;=\; \tilde{\mf a}_1 \;=\; {\mf a}_0
\end{equation}
%
where superimposed $\tilde{\tilde{}}$ is used to designate a vector in the
local (co-rotated) coordinate system of ${\mf R}_{Fxy}$.

Transforming $\tilde{\tilde{\mf a}}_3$ back to the global coordinate system is
performed in steps that reverse the intermediate local coordinate systems
%
\begin{eqnarray}
\tilde{\mf a}_3 &=& {\mf R}_y\tilde{\tilde{\mf a}}_3 \;=\;
{\mf R}_y{\mf R}_z{\mf a}_0 \\
{\mf a}_3 &=& {\mf R}_x\tilde{\mf a}_3 \;=\;
{\mf R}_x{\mf R}_y{\mf R}_z{\mf a}_0 \;=\; {\mf R}_{Fxyz}{\mf a}_0
\label{eqF:FollowerXYZ}
\end{eqnarray}

Comparing \eqsref{eqF:RigidXYZ}{eqF:FollowerXYZ},
the sequential rotations about follower axes are obtained by reversing the
order of the matrix multiplication in relation to rigid axis rotation and
forming the resultant rotation matrix:
%
\begin{eqnarray}
\text{Rigid (fixed) axis $X$-$Y$-$Z$ rotation:} \hskip5mm {\mf R}_{Rxyz} &=&
{\mf R}_z{\mf R}_y{\mf R}_x \\
\text{Follower axis $X$-$Y$-$Z$ rotation:} \hskip12mm {\mf R}_{Fxyz} &=&
{\mf R}_x{\mf R}_y{\mf R}_z
\label{eqF:EulerXYZ}
\end{eqnarray}

\subsubsection{Euler angles}

To define a number of Euler angle parameterizations,
the axis and the sequence in which to perform the rotations are determined.
The most common parameterizations use all axes in the sequences $X$-$Y$-$Z$,
$Y$-$Z$-$X$, and $Z$-$X$-$Y$, and their reverse-sequenced alternatives.
Two axis combinations such as $X$-$Y$-$X$ are also possible.
These parameterizations are easily established by substitutions into the
expression~\eqref{eqF:EulerXYZ}.

This gives the 12 different parameterizations of Table~\ref{tab:EulerAngles}.
The Euler angle parameterization used in Fedem is $Z$-$Y$-$X$.
%
\begin{table}[b]
\caption{Euler Angle Parameterizations}
\begin{center}
\begin{tabular}{llll}
\noalign{\hrule\smallskip}
Three axis combinations: & $X$-$Y$-$Z$, & $Y$-$Z$-$X$, & $Z$-$X$-$Y$, \\
                         & $Z$-$Y$-$X$, & $X$-$Z$-$Y$, & $Y$-$X$-$Z$  \\
\noalign{\smallskip\hrule\smallskip}
Two axis combinations:   & $X$-$Y$-$X$, & $Y$-$X$-$Y$, \\
                         & $Y$-$Z$-$Y$, & $Z$-$Y$-$Z$, \\
                         & $Z$-$X$-$Z$, & $X$-$Z$-$X$  \\
\noalign{\smallskip\hrule}
\end{tabular}
\end{center}
\label{tab:EulerAngles}
\end{table}

\subsection{Euler angles extraction}
\label{subs:Euler angles extraction}

Section~\ref{subs:Euler angles} describes how to obtain an incremental rotation
matrix, $\mf R$, given three Euler angles $\theta_x$, $\theta_y$ and $\theta_z$.
In some cases we would also like to perform the inverse transformation, that is,
given an (incremental) rotation matrix, find the corresponding Euler angles.

Assuming we use the Euler $Z$-$Y$-$X$ parameterization, we start by creating the
rotation matrix that rotates a vector an angle $\theta_z=\alpha$
about the $Z$-axis:
%
\begin{equation}
{\mf R}_z \;=\;
\left[\begin{array}{ccc}
\cos\alpha & -\sin\alpha & 0 \\
\sin\alpha &  \cos\alpha & 0 \\
0 & 0 & 1
\end{array}\right]
\end{equation}
%
Next, create the rotation matrix to rotate a vector an angle $\theta_y=\beta$
about the $Y$-axis:
%
\begin{equation}
{\mf R}_y \;=\;
\left[\begin{array}{ccc}
\cos\beta & 0 & \sin\alpha \\
0 & 1 & 0 \\
-\sin\beta & 0 & \cos\beta
\end{array}\right]
\end{equation}
%
Then, create the rotation matrix rotating a vector an angle $\theta_x=\gamma$
about the $X$-axis:
%
\begin{equation}
{\mf R}_x \;=\;
\left[\begin{array}{ccc}
1 & 0 & 0 \\
0 & \cos\gamma & -\sin\gamma \\
0 & \sin\gamma & \cos\gamma
\end{array}\right]
\end{equation}
%
Multiplying these matrices together we get the single rotation matrix:
%
\begin{equation}
\eqalign{{\mf R} &\;=\; {\mf R}_z {\mf R}_y {\mf R}_x \;=\cr&
\footnotesize
\left[\begin{array}{ccc}
\cos\alpha\cos\beta &
\cos\alpha\sin\beta\sin\gamma - \sin\alpha\cos\gamma &
\cos\alpha\sin\beta\cos\gamma + \sin\alpha\sin\gamma \\
\sin\alpha\cos\beta &
\sin\alpha\sin\beta\sin\gamma + \cos\alpha\cos\gamma &
\sin\alpha\sin\beta\cos\gamma - \cos\alpha\sin\gamma \\
-\sin\beta & \cos\beta\sin\gamma & \cos\beta\cos\gamma
\end{array}\right]}
\end{equation}
%
Letting $r_{ij}$ denote the individual elements of this matrix,
we can now derive the Euler angles as follows:
%
\begin{eqnarray}
\label{eq:EulerZ}
\theta_z \;=\; \alpha &=& \arctan\frac{r_{21}}{r_{11}} \\
\label{eq:EulerY}
\theta_y \;=\; \beta  &=& \arctan\frac{-r_{31}}{\sqrt{r_{11}^2 + r_{21}^2}} \\
\label{eq:EulerX}
\theta_x \;=\; \gamma &=& \arctan\frac{r_{32}}{r_{33}}
\end{eqnarray}
%
\remark{In a computer implementation of the above equations, one should use the
intrinsic math function {\tt atan2(y,x)} when evaluating $\arctan\frac{y}{x}$,
such that the sign of the computed angles are correctly set according to which
quadrant the arguments $x$ and $y$ implies.}
