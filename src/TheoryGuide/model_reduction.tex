% SPDX-FileCopyrightText: 2023 SAP SE
%
% SPDX-License-Identifier: Apache-2.0
%
% This file is part of FEDEM - https://openfedem.org

%%%%%%%%%%%%%%%%%%%%%%%%%%%%%%%%%%%%%%%%%%%%%%%%%%%%%%%%%%%%%%%%%%%%%%%%%%%%%%%%
%
% FEDEM Theory Guide.
%
%%%%%%%%%%%%%%%%%%%%%%%%%%%%%%%%%%%%%%%%%%%%%%%%%%%%%%%%%%%%%%%%%%%%%%%%%%%%%%%%

\chapter{Model Reduction}
\label{chap:CMS}

Dynamic analysis is generally more expensive than static analysis, because the solution involves
repeated computations of the same form, whereas static analysis requires only a single computation.
The introduction of Component Mode Synthesis (CMS) model reduction and superelement techniques,
however, reduces the cost of dynamic computations.

The CMS reduction technique reduces the cost of the analysis by decreasing the number DOFs
used. The fundamental assumption is that the low-frequency modes of vibration are the most
important; in other words, the higher the frequency mode, the less significant it is in the
dynamic analysis. It is also true that for any numerical model, and especially an FE model,
the high-frequency modes are less accurate in comparison to underlying differential equation
solutions.

The form of applied loads is important when deciding upon the required complexity of the FE
models. If a harmonic load is applied and the steady-state response is required, a peak response
is achieved if the load has a frequency component close to a resonant frequency. In this case,
the accuracy of the FE model must be sufficient to represent the resonant frequencies over
the entire range of the load's frequencies so that no significant resonant frequencies are
missed.

If the loads are not periodic \mdash{}as is the case, for example, in deployment and latching
simulations\mdash then there is no simple relationship between the structural resonant frequencies
and the frequency content of the loads. However, there is an effective frequency within the
loads defined by the rate of change from one state to another, that is, the rise time of the
input. If the force changes rapidly, it has a short rise time and the input has a high-frequency
content. This requires a relatively large number of DOFs in the model. If the force input changes
relatively slowly (it has a long rise time), it will have a lower frequency content, and fewer
modes are therefore required in the analysis.

The number of nodes in a mesh basically controls the number of DOFs in the model. When a dynamic
model is constructed, the mesh used can reflect the results. For instance, if only the displacement
or the first few resonant frequencies are required, a coarse mesh can be used. This mesh can
be much coarser than the mesh used in an equivalent static stress analysis. If, however, dynamic
stresses are required, a finer mesh density of the kind used for static stressing is necessary.
In this case, the mesh can be graduated in regions of stress concentration in exactly the same
manner as in static stress analysis.

The most expensive dynamic computation is performed for impact and wave propagation models.
Here, the short rise times of the forcing function require a relatively large number of DOFs.
In addition, the large stress and displacement changes at the front of the wave need a fine
mesh to model the wave front accurately. However, the wave front moves through the complete
model as the wave propagates, so a uniformly fine mesh is necessary throughout the model to
accurately address wave propagation problems.

There are various methods of condensing static models into smaller dynamic models automatically,
but for the various reasons outlined above, these methods must be used with care because they
are only applicable in certain cases, depending upon the required response. In some cases,
typically when calculating dynamic stresses, the degree to which a model can be condensed depends
partly upon the reduction process used, and the method of recovering stresses after the displacements
have been determined. These two factors must be considered together when determining the degree
of condensation to be employed within the analysis.

\section{Review of model reduction methods}

All reduction methods are based on the same principle: to reduce the effective number of DOFs
in order to reduce computational costs. However, because these methods are approximate and
different approximations can be used, the range of dynamic reduction methods is extensive.
In addition, each of these methods has sub-variants.

Most of the dynamic model reduction techniques available can be classified into
one of the following four categories:
%
\begin{itemize}
\item \textit{Modal Reduction} \mdash only a small number of the possible modes
of vibration\mdash{}typically those with lowest eigenfrequency\mdash{}are used
in the dynamic analysis.
%
\item \textit{Static Condensing (Guyan reduction)} \mdash this method of reducing the effective
number of DOFs is based entirely upon static considerations. Fedem uses this classic FE reduction
technique when no additional component modes are specified. In static reduction, the nodes
(and corresponding DOFs) are split into master/external and slave/internal nodes. Only master/external
nodes are retained after the static reduction.
%
\item \textit{Dynamic Condensing} \mdash the method of reducing the effective number of DOFs
is based upon both static and dynamic considerations. The CMS reduction implemented in Fedem
is one example of dynamic condensing.
%
\item \textit{Dynamic Substructuring} \mdash the complete structure is formed by establishing
a reduced set of equations for a set of components, or substructures, that comprises the total
structure or mechanism. One of the techniques above is used for each substructure.
\end{itemize}

All reduction (or condensing) methods can be written in terms of a coordinate transformation
of the form

\begin{equation}
{\mf v} \;=\; {\mf H}\,{\mf q}
\end{equation}

\noindent
where $\mf v$ is the full set of displacements of size $n$, $\mf q$ is the reduced or condensed
set of displacements of size $m$ where $m < n$, and $\mf H$ is a transformation matrix relating
the two sets of coordinates. The various reduction techniques can then be viewed as different
methods for constructing a suitable transformation matrix.

There are many ways to define the transformation matrix $\mf H$, but they are not all equally
accurate for dynamic analysis. Various factors must be considered, and most methods of condensing
the equations result in a compromise in the requirements. The first and most important considerations
are that dynamic behavior is largely controlled by the lowest static and component vibration
modes, and that any condensation method should allow the lower modes to be recovered with a
smaller error than the higher ones.

Secondly, the higher vibration modes still contribute to the dynamic response.
However, if the maximum frequency content of the acting loads is significantly
lower than a particular resonant frequency, the contribution of that high
resonance to the overall response will be in terms of a constant (static) value.
This is illustrated in Figure~\ref{fig:freq_response}.
%
\begin{figure}[t]
\center{
\setlength{\unitlength}{3pt}
\begin{picture}(90,55)(-10,-8)
\thicklines
% Axis system
\put(-10,0){\vector(1,0){90}}
\put( 0,-5){\vector(0,1){45}}
\put(-5,45){\textsl{Response}}
\put(65,-9){\textsl{Frequency}}
%The actual friction function
\thinlines
%Mode 1
\qbezier(   0, 8)(14, 8)(18.5,32)
\qbezier(18.5,32)(20,45)(21.5,32)
\qbezier(21.5,32)(25, 3)(67  , 1)
%Mode 2
\qbezier( 0, 6)(35  , 6  )(43,25)
\qbezier(43,25)(47.5,36  )(51,25)
\qbezier(51,25)(56  ,10.5)(67, 6.5)
%
\put(24,-1){\line(0,1){40}}
\put(23,-4){$\omega$}
\put( 6,33){\textsl{Mode 1}}
\put(52,28){\textsl{Mode 2}}
\end{picture}
}
\caption{Individual modal response}
\label{fig:freq_response}
\end{figure}
%
Mode 2 here has a substantially constant response up to the frequency $\omega$,
and its contribution to the overall response is defined almost entirely by its
stiffness over this range. This is not the case for mode 1, which requires dynamic behavior
because it varies considerably over the frequency range 0 to $\omega$. If a frequency of $2\omega$
or higher is necessary, mode 2 must also be modeled dynamically.

In general, the transformation matrix, $\mf H$, must be constructed in such
a way that the high-frequency modes can be defined by their static response.
Once the transformation has been defined, the equation of motion can be
condensed from the full set of $n$ equations to the smaller set, $m$,
by means of the transformation matrix $\mf H$.

In static analysis, a solution achieved by means of substructuring is exact. If the full structure
is analyzed directly and then as a series of substructures, it is found that apart from round-off
error the two solutions are identical. There is no approximation involved in static substructuring.

This is not the case with dynamic analysis. When the equations of any individual substructure
are reduced in size, information regarding the dynamic response \mdash{}that is, information
regarding the higher resonant frequencies and mode shapes\mdash is lost. The solution can be
organized to preserve the stiffness but not the mass characteristics associated with the high-frequency
modes. Thus, a static substructure solution is exact, but a dynamic substructure solution is
only approximate.

\subsection{Modal reduction}

In general, modal reduction is only carried out when the set of modes used in dynamic analysis
is reduced. In this case, each column in the transformation matrix, $\mf H$, is an eigenvector
of the system. The transformation to the reduced set of equations has the added computational
advantage that the reduced stiffness and mass matrices are both diagonal. For the assumption
of either proportional damping or modal damping, the reduced damping matrix is also diagonal.
This is used to solve the forced response of the equations, as the reduced equations are a
set of $m$ uncoupled single-DOF equations. Proportional damping is described in Section~\ref{sec:PropDamping}.

In modal reduction, the contribution of all the modes included in the transformation is exact.
The response, however, is not exact, even for a low-frequency input, because the static contributions
of the high-frequency modes have not been included. This is not important if the excitation
is near one of the low resonance frequencies, because the response of this mode will then dominate.
However, at a distance from a resonance it can lead to significant error.

\subsection{Static condensing (Guyan reduction)}

A common form of dynamic reduction is static condensing or Guyan reduction. In this form of
reduction, the stiffness properties of the structure are preserved at the expense of its dynamic
properties. The nodes (and corresponding DOFs) are split into master/external and slave/internal
nodes; only master DOFs are retained after reduction. This form of reduction is widely used,
being simple and relatively inexpensive to carry out; however, it should be used with some
caution.

For static condensing, a set of master/external DOFs are defined that become the $\mf q$~displacements
in the transformation equations. The remaining displacements are called the slave/internal
DOFs (and nodes), and these are eventually eliminated from the solution process. There are
many more slave than master DOFs. The transformation matrix, $\mf H$, is constructed by applying
a unit displacement to each master DOF in turn, while setting all the other master DOFs to
zero. The resulting slave displacements are used to define a column in $\mf H$. This is repeated
for each master in turn to construct the complete transformation matrix.

In Fedem, the master DOFs are defined implicitly by the selection of nodes for use as connection
points between component FE models; these points are called Triads.

This process of constructing the transformation matrix preserves rigid body inertias and the
static behavior of the structure. However, it only approximates the dynamic behavior, since
the process implicitly assumes that inertia forces on the slave freedoms are zero. This has
the effect of redistributing the mass of slave freedoms onto the masters (Triads), thereby
altering the distribution of structural mass. It is this assumption and its consequences that
introduce the errors into the Guyan reduction process. Errors can be minimized, but not eliminated,
by selecting an optimum set of master DOFs. The selection process and resulting problems are
the subjects of the following sections.

\subsubsection{Choice of master DOFs}

Within a static/Guyan reduction, the slave freedoms should be those that do not contribute
much to the inertia forces in the dynamic response. Inertia forces are the product of mass
and acceleration; slave freedoms, therefore, should be those points of low mass or low acceleration.
Expressed differently, the master freedoms should be defined at points where the inertia forces
are greatest; these correspond to points of high mass and low stiffness.

\subsubsection{Problems of static reduction}

It is very difficult to quantify the errors associated with static reduction. Generally, the
error in a given resonant frequency or mode shape increases with eigenvalue number: the lower
eigenvalues are predicted more accurately than the higher ones. However, this rule is not absolute,
and occasionally some higher modes are predicted more accurately than lower ones. Any predicted
mode shape can only be a linear combination of the columns of the transformation matrix, $\mf
H$. If $\mf H$ is deficient in the pattern of displacement required by a given mode shape,
that mode will not be predicted accurately. If this is the case, the eigenvalues of the condensed
system will be greater than those of the original uncondensed system, resulting in a stiffer
system than that of the original.

The fewer the number of master freedoms (Triads), the greater the potential error. To minimize
error, the user must choose a sufficient number of masters \mdash{}preferably between two and
ten times the number of required modes. However, even with a factor of ten it is very difficult
to predict more than general consequences of the reduction. A choice of too many masters can
increase computational costs.

The original finite element equation system is heavily banded, and its sparse nature enables
efficient handling of large sets of equations. However, the reduced, transformed set is fully
populated, so that although fewer equations are involved, they are much more densely packed
and therefore more costly to solve. If the original set of equations were of size~$n$ and average
bandwidth~$b$, then the number of master freedoms,~$m$, should satisfy $m^{2} < bn$. If not,
it is less expensive computationally to work with the original set of equations without any
reduction.

\subsection{Dynamic condensing}

Fedem uses a dynamic condensing technique that combines modal reduction and static condensing.
The disadvantage of modal reduction (apart from the cost of determining the eigenvectors) is
that it does not preserve the static behavior of the structure, because the stiffness contribution
from the modes that have been ignored is never reinstated
(refer to Figure~\ref{fig:freq_response}).
Conversely, static condensing preserves static stiffness at the expense of dynamic effects.
In dynamic condensing, the transformation matrix, $\mf H$, is largely based on the eigenvectors
associated with the lowest frequencies, as is the case with modal reduction; but these eigenvectors
are augmented by the static modes associated with master freedoms, as in static condensation.

In dynamic condensing, static modes must be manipulated so that the stiffness associated with
lower modes (stiffness accounted for by the presence of these modes) is not included twice.
In Fedem, static modes are complemented by "fixed interface normal modes," and these normal
modes (eigenvectors) are never included in the preserved static modes. Fixed interface modes
are computed by eigenvector analysis of the structure with all the nodal DOFS (triad DOFS)
fixed. A fixed DOF means defining its value as zero and (usually) removing it from the equation
system.

\subsection{Dynamic substructuring}

Another method of improving the computational efficiency of a dynamic analysis is to use dynamic
substructuring. However, its use also compromises accuracy. The structure or mechanism is divided
into a series of smaller structures or substructures; in Fedem, these are called Links. The
division into Links in Fedem is based on physical identification of the various mechanism components.

The transformation matrix in Fedem is defined in such a way as to preserve the original freedoms
and takes the following form for each Link:
%
\begin{namelist}{${\mf v}_1$\ }
\item[{${\mf v}_e$}] represents the freedoms that connect to other Links or applied springs,
dampers, and loads.
%
\item[{${\mf v}_i$}] represents the internal freedoms that are internal to a single Link.
%
\item[{$\mf q$}] represents the generalized freedoms, internal to the reduced Link, that are
included to augment and improve the dynamic response (fixed interface normal/component modes).
\end{namelist}
%
It is important to note that with this definition of $\mf H$ the connection freedoms are preserved
across the transformation, making assembly of the complete mechanism relatively simple. However,
this approach does place restrictions upon the method of transformation.

The simplest way of forming the subcomponent is to base the transformation upon Guyan reduction.
Here the master freedoms are not chosen arbitrarily, but are defined at the connection nodes
(Triads) between Links and the nodes at which springs, dampers, and loads are applied. There
are no augmenting freedoms $\mf q$, and the transformation matrix is identical to that of a
Guyan reduction with the connection nodes/freedoms selected as the master freedoms.

\subsubsection{Internal component modes in dynamic substructuring}

The accuracy of the dynamic behavior of a substructure can be improved considerably by including
extra terms in the augmenting freedoms, $\mf q$, of the transformation matrix. These terms
are usually based on some of the system eigenvectors, thus the process becomes very similar
to that of the dynamic condensing method described above. There are many ways of including
these terms, all of which lead to different variations of the dynamic substructure method.

It is difficult to say which is the best method, as they all produce similar results and their
relative accuracy is dependent upon the specific problem being analyzed. The various forms
of dynamic substructuring can be classified in one of two categories, depending on the way
in which the eigenvectors are formed. The most direct method is to calculate the substructure
eigenvectors with the connection nodes/freedoms fixed; this is the method employed in Fedem.
The alternative method is to calculate the eigenvectors of the substructure with the connection
freedoms free.

\subsubsection{Fixed boundary dynamic substructure}

If the eigenvectors of the substructure are calculated while the connection
freedoms are held fixed, the coordinate transformation matrix $\mf H$ contains
the first few eigenvectors of the fixed boundary substructure
\mdash{}the $\tf\Phi$ term in \eqnref{eqCMS:Hdef}.
The number of augmenting freedoms is then equal to the number of these fixed
interface modes.
The static modes can be taken directly from the Guyan reduction process with the
connection freedoms as masters \mdash{}the $\mf B$ term in \eqnref{eqCMS:Hdef}.
This means that the behavior of the substructure is defined by the static
behavior of its connection freedoms and the internal dynamic behavior
of its augmenting freedoms.
The representation of the dynamic behavior of the substructure still includes
some approximations, but these become progressively less significant as more
eigenvectors are added to the augmenting set.

This combined approach, which is the method used in Fedem,
is referred to as the Component Mode Synthesis (CMS) method.
The user can also add extra Triads, which need not be connection nodes,
for the purpose of improving the dynamic behavior of the substructure.

\subsubsection{Free boundary dynamic substructure}

The alternative to the fixed boundary substructure is the free boundary form. In this case
the connection freedoms, ${\mf v}_1$, are left free, and the transformation matrix is defined
in terms of the eigenvectors. If the substructure has no natural supports, the eigenvectors
will include the rigid body modes to represent the rigid body mass of the structure. The free
boundary substructure generally requires more calculations to form the transformation matrix.
This form of dynamic substructure ensures that the mass description of the component is correct;
the stiffness description, however, will only be approximate. To recover correct stiffness
behavior, at least for the connection freedoms, the dynamic modes can be augmented by static
modes\footnote{There are some indications that free boundary dynamic substructure methods
are more accurate than fixed boundary methods, depending upon the manner of implementation
for each method and the type of problem being solved.}.

\subsection{Summary}

Fedem uses a state-of-the-art dynamic condensing method, Component Mode Synthesis (CMS), which
combines static and fixed-interface normal modes. Fedem  can also increase accuracy by including
an additional distribution of master DOFs (Triads). The CMS method is also well suited to flexible
mechanism analysis because it preserves the effective masses and inertias.

Depending on complexity and modeling detail, the finite element model of a substructure can
consist of a large or small number of FE nodes. These nodes are divided into internal and external
nodes. The external nodes are defined during mechanism modeling as the connection points for
joints, springs, dampers, external loads, control input, points of interest, and so on. These
nodes are retained as so-called "supernodes" during the reduction of the substructure to a
superelement.

% SPDX-FileCopyrightText: 2023 SAP SE
%
% SPDX-License-Identifier: Apache-2.0
%
% This file is part of FEDEM - https://openfedem.org

%%%%%%%%%%%%%%%%%%%%%%%%%%%%%%%%%%%%%%%%%%%%%%%%%%%%%%%%%%%%%%%%%%%%%%%%%%%%%%%%
%
% FEDEM Theory Guide.
%
%%%%%%%%%%%%%%%%%%%%%%%%%%%%%%%%%%%%%%%%%%%%%%%%%%%%%%%%%%%%%%%%%%%%%%%%%%%%%%%%

\section{Component mode synthesis reduction}
\label{sec:CMS reduction}

For each substructure modeled, the status codes for the substructure DOFs are set to 1 for
internal DOFs and 2 for external DOFs. With this control information, the system mass and stiffness
matrices are assembled from the corresponding element matrices. The first $n_1$ DOFs of the
system matrices are internal substructure DOFs, and the next $n_2$ DOFs are external.

In symbolic form, the substructure matrices for mass and stiffness can be divided into submatrices,
with index $i$ for internal and $e$ for external as follows:
%
\begin{equation}
{\mf M} \;=
\left[\begin{array}{cc}
{\mf M}_{ii} & {\mf M}_{ie} \\
{\mf M}_{ei} & {\mf M}_{ee} \\
\end{array} \right]
\quad\text{and}\quad
{\mf K} \;=
\left[\begin{array}{cc}
{\mf K}_{ii} & {\mf K}_{ie} \\
{\mf K}_{ei} & {\mf K}_{ee} \\
\end{array} \right]
\end{equation}
%
where ${\mf M}_{xx}$ are mass and ${\mf K}_{xx}$ are stiffness submatrices.

The stiffness relation for the substructure may now be expressed as:
%
\begin{equation}
\left[\begin{array}{cc}
{\mf K}_{ii} & {\mf K}_{ie} \\
{\mf K}_{ei} & {\mf K}_{ee} \\
\end{array}\right] \left[\begin{array}{cc}
{\mf v}_i \\
{\mf v}_e \\
\end{array}\right] = \left[\begin{array}{cc}
{\mf Q}_i \\
{\mf Q}_e \\
\end{array}\right]
\label{eqCMS:41}
\end{equation}
%
where ${\mf v}_x$ is the displacement vector and ${\mf Q}_x$ is the load vector.

\subsection{Static modes}
\label{subsec:Static modes}

The first line of \eqnref{eqCMS:41} may be written:
%
\begin{equation}
\eqalign{
{\mf v}_i \;=\; & {\mf K}_{ii}^{-1} {\mf Q}_i
                - {\mf K}_{ii}^{-1} {\mf K}_{ie} {\mf v}_e \cr =\; &
{\mf v}_i^i + {\mf v}_i^e}
\label{eqCMS:42}
\end{equation}
%
where ${\mf v}_i^i$ represents internal displacements with fixed external DOFs,
and ${\mf v}_i^e$ represents internal displacements as a function of external
displacements.
Examination of each term provides
%
\begin{equation}
{\mf v}_i^i \;=\; {\mf K}_{ii}^{-1} {\mf Q}_i
\label{eqCMS:43}
\end{equation}
%
and
%
\begin{equation}
{\mf v}_i^e \;=\; -{\mf K}_{ii}^{-1}{\mf K}_{ie}{\mf v}_e \;=\; {\mf B}{\mf v}_e
\label{eqCMS:44}
\end{equation}
%
where
%
\begin{equation}
{\mf B} \;=\; -{\mf K}_{ii}^{-1} {\mf K}_{ie}
\label{eqCMS:45}
\end{equation}

The set of nodal displacements $[\;{\mf v}_i^e \;\; {\mf v}_e\;]$ represents the
"exact" solution of static loads and boundary conditions acting on the external
nodes ("exact" in the sense that the model reduction has not in any way changed
the solution in relation to the solution of the full system).

\subsection{Constrained dynamic modes}
\label{secCMS:FixeIntMode}

The substructure matrices can be reduced by CMS transformation to produce a set of modes called
{\it Craig Bampton modes}. This is done by eliminating the internal DOFs as system DOFs, and
replacing them with a limited number of substructure vibration modes called generalized DOFs.
The CMS transformation starts with an eigenvalue analysis of the substructure system matrices
with the external DOFs fixed at zero. The equation for free, undamped vibration of the substructure's
internal DOFs is written:
%
\begin{equation}
{\mf M}_{ii} \ddot{\mf v}_i^i + {\mf K}_{ii} {\mf v}_i^i \;=\; {\mf 0}
\label{eqCMS:46}
\end{equation}

Considering simple harmonic motion, the displacement ${\mf v}_i^i$ may be expressed as:
%
\begin{equation}
{\mf v}_i^i \;=\; {\bs\phi}\,\sin\omega t
\label{eqCMS:47}
\end{equation}
%
where the eigenvector ${\bs\phi}$ is defined by the eigenvalue problem.
%
\begin{equation}
\left({\mf K}_{ii} -\omega^2{\mf M}_{ii}\right) {\bs\phi} \;=\; {\mf 0}
\label{eqCMS:48}
\end{equation}

In a substructure with $n$ active DOFs of which $p$ are external DOFs,
the internal displacements $v_i^i$ can be expressed as:
%
\begin{equation}
{\mf v}_i^i \;=\; \sum_{k=1}^S {\bs\phi}_k \, y_k \;=\; {\bs\Phi}{\mf y}
\label{eqCMS:49}
\end{equation}
%
where $s<n-p$, and
%
\begin{equation}
{\bs\Phi} \;=\; \left[ \begin{array}{cccc}
{\bs\phi}_1 & {\bs\phi}_2 & \cdots & {\bs\phi}_s
\end{array}\right]
\end{equation}
%
is the eigenvector matrix and has dimensions $(n-p)\times s$.

\subsection{Reduced system}
\label{subsec:Reduced system}

The superelement displacements are now expressed by the external DOFs
${\mf v}_e$ and by the new generalized DOFs ${\mf y}$:
%
\begin{equation}
{\mf v} \;= \left[\begin{array}{c}
{\mf v}_e \\ {\mf v}_i
\end{array}\right] = \left[\begin{array}{cc}
{\mf I} & {\mf 0} \\
{\mf B} & {\tf\Phi}
\end{array}\right]
\left[\begin{array}{c}
{\mf v}_e \\ {\mf y}
\end{array}\right] =\; {\mf H} \left[\begin{array}{c}
{\mf v}_e \\ {\mf y}
\end{array}\right]
\label{eqCMS:Hdef}
\end{equation}

Usually only a few of the lowest modes of vibration need to be included to
achieve good results, and this may reduce the size of the problem substantially.
If all eigenmodes are included, $s=n-p$, the CMS transformation is exact.

\iftoggle{publicedition}{}{% The following is for the in-house edition only.

The substructure dynamic equation of motion may be written:
%
\begin{equation}
\label{eqCMS:PartitionedModelEq}
\eqalign{
\left[\begin{array}{cc}
{\mf M}_{ee} & {\mf M}_{ei} \\
{\mf M}_{ie} & {\mf M}_{ii}
\end{array}\right] \left[\begin{array}{cc}
\ddot{\mf v}_e \\
\ddot{\mf v}_i
\end{array}\right] \;+\; &
\left[\begin{array}{cc}
{\mf C}_{ee} & {\mf C}_{ei} \\
{\mf C}_{ie} & {\mf C}_{ii}
\end{array}\right] \left[\begin{array}{c}
\dot{\mf v}_e \\
\dot{\mf v}_i
\end{array}\right] \cr +\; &
\left[\begin{array}{cc}
{\mf K}_{ee} & {\mf K}_{ei} \\
{\mf K}_{ie} & {\mf K}_{ii}
\end{array}\right] \left[\begin{array}{c}
{\mf v}_e \\
{\mf v}_i
\end{array}\right] \;=\;
\left[\begin{array}{c}
{\mf Q}_e \\
{\mf Q}_i
\end{array}\right]}
\end{equation}
%
where ${\mf C}_{xx}$ represents damping.

Combining \eqnref{eqCMS:Hdef} and its first and second time derivatives with
\eqnref{eqCMS:PartitionedModelEq}, and pre-multiplying with ${\mf H}^T$ produces
%
\begin{equation}
\label{eqCMS:ReducedModelEq}
\eqalign{
\left[\begin{array}{cc}
{\mf m}_{11} & {\mf m}_{12} \\
{\mf m}_{21} & {\mf m}_{22}
\end{array}\right]
\left[\begin{array}{c}
\ddot{\mf v}_e \\
\ddot{\mf y}
\end{array}\right] \;+\; &
\left[\begin{array}{cc}
{\mf c}_{11} & {\mf c}_{12} \\
{\mf c}_{21} & {\mf c}_{22}
\end{array}\right] \left[\begin{array}{c}
\dot{\mf v}_e \\
\dot{\mf y}
\end{array}\right] \cr +\; &
\left[\begin{array}{cc}
{\mf k}_{11} & {\mf k}_{12} \\
{\mf k}_{21} & {\mf k}_{22}
\end{array}\right] \left[\begin{array}{c}
{\mf v}_e \\
{\mf y}
\end{array}\right] \;=\;
\left[\begin{array}{c}
{\mf q}_1 \\
{\mf q}_2
\end{array}\right]}
\end{equation}
%
where
%
\begin{eqnarray}
{\mf m}_{11} &=& {\mf M}_{ee} +
{\mf B}^T {\mf M}_{ie} + {\mf M}_{ei} {\mf B} +
{\mf B}^T {\mf M}_{ii} {\mf B}
\label{eqCMS:Msubs}
\\
{\mf m}_{12} &=& {\mf m}_{21}^T \;=\;
{\mf M}_{ei} {\bs\Phi} + {\mf B}^T
{\mf M}_{ii} {\bs\Phi}
\nonumber \\
{\mf m}_{22} &=& {\bs\Phi}^T
{\mf M}_{ii} {\bs\Phi}
\nonumber \\[5mm]
%
{\mf c}_{11} &=& {\mf C}_{ee} +
{\mf B}^T {\mf C}_{ie} + {\mf C}_{ei} {\mf B} +
{\mf B}^T {\mf C}_{ii} {\mf B}
\label{eqCMS:Csubs}\\
{\mf c}_{12} &=&  {\mf c}_{21}^T \;=\;
{\mf C}_{ei} {\bs\Phi} + {\mf B}^T
{\mf C}_{ii} {\bs\Phi}
\nonumber \\
{\mf c}_{22} &=& {\bs\Phi}^T
{\mf C}_{ii} {\bs \Phi }
\nonumber \\[5mm]
%
{\mf k}_{11} &=& {\mf K}_{ee} +
{\mf K}_{ie}^T {\mf B}
\label{eqCMS:Ksubs}\\
{\mf k}_{12} &=& {\mf k}_{21}^T \;=\; {\mf 0}
\nonumber \\
{\mf k}_{22} &=& {\bs \Phi}^T {\mf K}_{ii} {\bs\Phi}
\nonumber \\[5mm]
%
{\mf q}_1 &=& {\mf Q}_e + {\mf B}^T {\mf Q}_i
\label{eqCMS:Qsubs}\\
{\mf q}_2 &=& {\bs\Phi}^T {\mf Q}_i
\nonumber
\end{eqnarray}

The matrix ${\mf m}_{22}$ from \eqnref{eqCMS:Msubs} is diagonal and the
eigenmodes are normalized, so that
%
\begin{equation}
{\mf m}_{22} \;=\; {\mf I} \;=\; {\bs\Phi}^T {\mf M}_{ii} {\bs\Phi}
\label{eqCMS:424}
\end{equation}
%
which is equal to the unit matrix.
Notice that ${\mf m}_{11}$, ${\mf m}_{12}$, and ${\mf m}_{21}$ are not diagonal.

The matrix ${\mf k}_{11}$ from \eqnref{eqCMS:Ksubs} is reduced by the expression
%
\begin{equation}
{\mf k}_{11} \;=\; {\mf K}_{ee} +
{\mf B}^T {\mf K}_{ie} + {\mf K}_{ei} {\mf B} +
{\mf B}^T {\mf K}_{ii} {\mf B}
\label{eqCMS:425}
\end{equation}
%
Expanding the last two terms by \eqnref{eqCMS:45} shows that the terms are
equal, but with opposite signs; therefore, they cancel out of the equation.
The similarity of these terms is due to the symmetric properties of the
stiffness matrix, which produce
%
\begin{equation}
{\mf K}_{ie}^T \;=\; {\mf K}_{ei}
\label{eqCMS:426}
\end{equation}
%
and
%
\begin{equation}
\left({\mf K}_{ii}^{-1}\right)^T \;=\; {\mf K}_{ii}^{-1}
\label{eqCMS:427}
\end{equation}

For the same reasons, the matrices ${\mf k}_{12} = {\mf k}_{21}^T$
are reduced from the equation
%
\begin{equation}
\label{eqCMS:428}
\eqalign{
{\mf k}_{12} \;=\;& {\mf k}_{21}^T \;=\; {\mf K}_{ei} {\bs\Phi} +
{\mf B}^T {\mf K}_{ii} {\bs\Phi} \cr =\; &
{\mf K}_{ei} {\bs\Phi} +
\left(-{\mf K}_{ii}^{-1}{\mf K}_{ie}\right)^T {\mf K}_{ii} {\bs\Phi} \cr =\; &
{\mf K}_{ei} {\bs\Phi} -
{\mf K}_{ie}^T \left({\mf K}_{ii}^{-1}\right)^T
{\mf K}_{ii}{\bs\Phi} = {\mf 0}}
\end{equation}

It may be shown that the stiffness matrix ${\mf k}_{22}$ is diagonal and of a form in which
%
\begin{equation}
{\mf k}_{22} = \left\lceil\begin{array}{cccc}
\omega_1^2 & \omega_2^2 & \cdots & \omega_{n-p}^2
\end{array}\right\rfloor
\label{eqCMS:429}
\end{equation}
%
$\omega_1^2, \omega_2^2, \ldots\omega_{n-p}^2$ are the eigenvalues corresponding
to the eigenmodes of eigenvector matrix $\bs\Phi$.

The substructure matrices reduced by CMS transformation, as shown above,
are the superelement matrices in a standardized form (Craig-Bampton modes).
These matrices will form the basis for the mechanism simulation formulation.

\subsection{Gravitational forces}

Gravitational forces are calculated from unit gravitational acceleration vectors in the substructure's
local $x$-, $y$-, and $z$-directions and denoted ${\mf U}_x$, ${\mf U}_y$, and ${\mf U}_z$,
respectively. This also means that for all DOFs in ${\mf U}_x$ corresponding to $x$-translation,
the acceleration component of ${\mf U}_x$ is set equal to 1, whereas those corresponding to
$y$- and $z$-translation are equal to 0. In addition, for ${\mf U}_y$ and ${\mf U}_z$ the acceleration
components in the $y$- and $z$-directions, respectively, are set equal to 1, while the accelerations
in the other directions are~0.

The gravitational forces ${\mf G}_x$ corresponding to ${\mf U}_x$ are calculate
from (see \eqnref{eqCMS:ReducedModelEq}):
%
\begin{equation}
\left[\begin{array}{cc}
{\mf G}_{xi} \\ {\mf G}_{xe}
\end{array}\right] =
\left[\begin{array}{cc}
{\mf M}_{ii} & {\mf M}_{ie} \\
{\mf M}_{ei} & {\mf M}_{ee}
\end{array}\right]
\left[\begin{array}{cc}
{\mf U}_{xi} \\ {\mf U}_{xe} \\
\end{array} \right] =
\left[\begin{array}{cc}
{\mf M}_{ii}{\mf U}_{xi} + {\mf M}_{ie}{\mf U}_{xe} \\
{\mf M}_{ei}{\mf U}_{xi} + {\mf M}_{ee}{\mf U}_{xe}
\end{array}\right]
\label{eqK35:1}
\end{equation}
%
The forces from unit gravitational acceleration in the $x$-direction are reduced
by CMS-transformation to ${\mf g}_x$ (see \eqnref{eqCMS:PartitionedModelEq}):
%
\begin{equation}
\left[\begin{array}{cc}
{\mf g}_{xe} \\ {\mf g}_{xg}
\end{array}\right]  =
\left[\begin{array}{cc}
{\mf I} & {\mf B}^T \\
{\mf 0} & {\bs\Phi}^T
\end{array}\right]
\left[\begin{array}{cc}
{\mf G}_{xe} \\ {\mf G}_{xi}
\end{array}\right] =
\left[\begin{array}{cc}
{\mf G}_{xe} + {\mf B}^T {\mf G}_{xi} \\
{\bs\Phi}^T {\mf G}_{xi}
\end{array}\right]
\label{eqK35:2}
\end{equation}
%
Expanding by using \eqnref{eqK35:1} results in:
%
\begin{eqnarray}
{\mf g}_{xe} &=&  {\mf M}_{ie}^T {\mf U}_{xi} +
{\mf M}_{ee} {\mf U}_{xe} + {\mf B}^T {\mf M}_{ii}{\mf U}_{xi} +
{\mf B}^T  {\mf M}_{ie} {\mf U}_{xe}
\label{eqK35:3} \\
%
{\mf g}_{xg} &=& {\bs\Phi}^T {\mf M}_{ii}{\mf U}_{xi} +
{\bs\Phi}^T {\mf M}_{ie}{\mf U}_{xe}
\label{eqK35:5}
\end{eqnarray}
%
taking into account the symmetry property ${\mf M}_{ei} = {\mf M}_{ie}^T$.

The corresponding reduced gravitational forces from unit gravitation in the
$y$-direction are easily found by changing indices:
%
\begin{eqnarray}
{\mf g}_{ye} &=& {\mf M}_{ie}^T {\mf U}_{yi} +
{\mf M}_{ee} {\mf U}_{ye} + {\mf B}^T {\mf M}_{ii}{\mf U}_{yi} +
{\mf B}^T {\mf M}_{ie} {\mf U}_{ye}
\label{eqK35:6} \\
%
{\mf g}_{yg} &=& {\bs\Phi}^T {\mf M}_{ii}{\mf U}_{yi} +
{\bs \Phi}^T {\mf M}_{ie}{\mf U}_{ye}
\label{eqK35:7}
\end{eqnarray}
%
and similarly in the $z$-direction:
%
\begin{eqnarray}
{\mf g}_{ze} &=& {\mf M}_{ie}^T {\mf U}_{zi} +
{\mf M}_{ee} {\mf U}_{ze} + {\mf B}^T {\mf M}_{ii}{\mf U}_{zi} +
{\mf B}^T  {\mf M}_{ie} {\mf U}_{ze}
\label{eqK35:8} \\
%
{\mf g}_{zg} &=& {\bs\Phi}^T {\mf M}_{ii}{\mf U}_{zi} +
{\bs \Phi}^T {\mf M}_{ie}{\mf U}_{ze}
\label{eqK35:9}
\end{eqnarray}

The superelement gravitational forces ${\mf g}$ are now calculated from
(see \meqsref{eqK35:3}{eqK35:9}):
%
\begin{equation}
{\mf g} \;=\; a_{g_x}
\left[\begin{array}{cc}
{\mf g}_{xe} \\ {\mf g}_{xg}
\end{array}\right] + a_{g_y}
\left[\begin{array}{cc}
{\mf g}_{ye} \\ {\mf g}_{yg}
\end{array}
\right] + a_{g_z}
\left[\begin{array}{cc}
{\mf g}_{ze} \\ {\mf g}_{zg}
\end{array}\right]
\label{eqK35:11}
\end{equation}
%
where ${\mf a}_g = [\; a_{g_x} \;\; a_{g_y} \;\; a_{g_z}\;]^T$ is the
gravitational acceleration, measured in the superelement coordinate system.

These gravitational forces must now be added to the system force vector,
but only after they have been transformed to the local supernode coordinate
system, if there is one.

} % End in-house edition only

\iftoggle{publicedition}{}{% in-house edition only
% SPDX-FileCopyrightText: 2023 SAP SE
%
% SPDX-License-Identifier: Apache-2.0
%
% This file is part of FEDEM - https://openfedem.org

%%%%%%%%%%%%%%%%%%%%%%%%%%%%%%%%%%%%%%%%%%%%%%%%%%%%%%%%%%%%%%%%%%%%%%%%%%%%%%%%
%
% FEDEM Theory Guide.
%
%%%%%%%%%%%%%%%%%%%%%%%%%%%%%%%%%%%%%%%%%%%%%%%%%%%%%%%%%%%%%%%%%%%%%%%%%%%%%%%%

\def\CGtiny{{\mbox{\tiny CG}}}
\def\CRtiny{{\mbox{\tiny CR}}}

\section{Inertia forces and high-speed rotation}
\label{s:Inertia forces and high-speed rotation}

When performing FE-based dynamics analysis with high-speed rotation, the inertia
forces can be poorly represented if superelements with very few external DOFs
have motion with high-speed rotation components.

This problem is usually only encountered with rigid body analysis, in which the
number of external DOFs for each superelement is too limited to calculate the
stresses with any accuracy.
However, such a model is often desirable, as it can provide fairly accurate
motion and joint force characteristics while being very efficient numerically.
The cause of the problem is that model reduction (as described earlier in
Chapter~\ref{chap:CMS}) invariably reduces the interpolation order of
displacements, velocities and accelerations over the superelement,
thereby producing a poor representation of the centripetal acceleration field
for the superelement or element mesh.

There are two remedies for this problem:
%
\begin{enumerate}
\item Using an adequate number of external nodes to represent the centripetal
acceleration field.
This generally means selecting nodes that effectively span the complete
superelement and surround the center of gravity.
\item Using an automatic algorithm to correct the inertia forces with respect to
rigid body resultants.
\end{enumerate}

Remedy 1 is not always desirable, as it mandates the use of more DOFs than are
strictly necessary to achieve satisfactory result.
In addition, it takes some experience to determine an adequate selection of
external nodes.

In order to eliminate the guesswork from remedy 1 above, a procedure to obtain
accurate rigid body inertia forces while still allowing a minimal number of
external DOFs has been developed.
The procedure uses the rigid body mass properties for each superelement to check
consistency between the nodal inertia forces and rigid body resultant forces.
If they are not consistent, the nodal inertia forces are corrected to ensure
correct rigid body resultants.

It is important to note that if a fairly large number of external nodes are
chosen in order to obtain good stress results for a particular superelement,
a good representation of the acceleration field is also ensured and the inertia
force corrections outlined in this section are reduced (approaches zero).
In this way the additional correction forces will not affect the stress results.

\subsection{Rigid body mass properties}
\label{subs:Rigid body mass properties}

The rigid body mass properties for a structure can be calculated from the
FE~representation of the structure---either the full mass matrix $\mf M$,
or the reduced mass matrix ${\mf M}_r$ given by \eqnref{eqCMS:ReducedModelEq}.
It can be shown that both $\mf M$ and ${\mf M}_r$ yield the same mass property
matrix, ${\mf I}_a$, about a given point $a$.

\subsubsection{Inertia matrix about an arbitrary point}

For a given structure, an $n_{\text{\tiny DOFs}}\times 6$ matrix for the rigid
body displacement vector can be formed, in which each column consists of a rigid
body displacement vector of unit magnitude:
%
\begin{equation}
{\mf R}_a = \left[ \begin{array}{*6c}
{\mf v}_{x} & {\mf v}_y & {\mf v}_z &
{\mf v}_{\omega_x} & {\mf v}_{\omega_y} & {\mf v}_{\omega_z}
\end{array}\right]
\end{equation}
%
Using this matrix in a congruence transformation of the mass matrix yields
%
\begin{equation}
\eqalign{
{\mf I}_a =\;& {\mf R}_a^T{\mf M}_r {\mf R}_a \cr =\;&
\left[\begin{array}{*6c}
m & 0 & 0 & 0 & mz_\CGtiny & -my_\CGtiny \\
0 & m & 0 & -mz_\CGtiny & 0 & mx_\CGtiny \\
0 & 0 & m & my_\CGtiny & -mx_\CGtiny & 0 \\
0 & -mz_\CGtiny & my_\CGtiny & I_{xx_a} & I_{xy_a} & I_{xz_a} \\
mz_\CGtiny & 0 & -mx_\CGtiny & I_{yx_a} & I_{yy_a} & I_{yz_a} \\
-my_\CGtiny & mx_\CGtiny & 0 & I_{zx_a} & I_{zy_a} & I_{zz_a}
\end{array}\right]
}\label{eq:Ia}
\end{equation}

\subsubsection{Inertia matrix about center of gravity}

From the off-diagonal submatrices in \eqnref{eq:Ia},
the position of the Center of Gravity (CG) can be established.
Performing the transformation about the CG yields
%
\begin{equation}
{\mf I}_\CGtiny = {\mf R}_\CGtiny^T {\mf M}_r {\mf R}_\CGtiny =
\left[\begin{array}{*6c}
m & 0 & 0 & 0 & 0 & 0 \\
0 & m & 0 & 0 & 0 & 0 \\
0 & 0 & m & 0 & 0 & 0 \\
0 & 0 & 0 & I_{xx} & I_{xy} & I_{xz} \\
0 & 0 & 0 & I_{yx} & I_{yy} & I_{yz} \\
0 & 0 & 0 & I_{zx} & I_{zy} & I_{zz}
\end{array}\right]
\label{eq:Icg}
\end{equation}

\begin{figure}[b]
\center{
\setlength{\unitlength}{1mm}
\begin{picture}(55,70)(15,2)
\thinlines
% Global coordinate system
\thicklines
\put( 37,53){CG}
\put( 40,50){\vector(3, 1){20}}\put(63,57){${\tf n}_1$}
\put( 40,50){\vector(2,-1){15}}\put(59,41){${\tf n}_2$}
\put( 40,50){\vector(0,-1){25}}\put(43,25){${\tf n}_3$}
\thinlines
\put( 46,52){\line(2,-1){4}}
\put( 44,48){\line(3, 1){6}}
\thicklines
\put( 40,50){\vector(1,0){10}}\put(51,49){${\tf u}_\CGtiny$}
\put( 19,3){\vector(3, 1){40}}
\put( 19,3){\vector(3, 1){38}}\put(62,17){${\tf\omega}_\CGtiny$}
\thinlines
\put( 40,50){\line(0,-1){40}}\put(39,6){CR}
\put( 40,15){\line(3, 1){4.5}}
\put( 44.5,11.5){\line(0, 1){5}}
% The ``potato''
\thicklines
\qbezier(27,40)( 5,46)(25,52)
\qbezier(25,52)(32,54)(37,62)
\qbezier(37,62)(46,74)(55,60)
\qbezier(55,60)(63,50)(53,44)
\qbezier(53,44)(40,36)(27,40)
%\graphpaper[2](0,0)(70,70)
\end{picture}
}
\caption{Rotation axis of screw-like motion}
\label{fig:screw-like motion}
\end{figure}

\subsection{Extracting rigid body velocity}
\label{subs:Extracting rigid body velocity}

The element velocity vector is composed of both rigid body displacements,
${\mf v}_r$, and deformational displacements, ${\mf v}_d$, where the
deformational displacements are orthogonal to rigid body displacements, i.e.
%
\begin{equation}
\dot{\mf v} \;=\; \dot{\mf v}_r + \dot{\mf v}_d \;=\;
{\mf R}_\CGtiny \dot{\mf a} + \dot{\mf v}_d
\quad\mbox{where}\quad
{\mf R}_\CGtiny^T \dot{\mf v}_d = {\mf 0}
\end{equation}
%
The rigid body velocities can thus be established as
%
\begin{equation}
\dot{\mf a} \;=\; \left({\mf R}_\CGtiny^T {\mf R}_\CGtiny\right)^{-1}
{\mf R}_\CGtiny^T \dot{\mf v}
\quad\mbox{where}\quad
\dot{\mf a} \;= \left[\begin{array}{c}
\dot{u}_\CGtiny \\ \dot{v}_\CGtiny \\ \dot{w}_\CGtiny \\
\omega_{x} \\ \omega_{y} \\ \omega_{z}
\end{array}\right] = \left[\begin{array}{c}
\dot{\mf u}_\CGtiny \\
{\tf\omega}_\CGtiny
\end{array}\right]
\label{eq:rigid velocity}
\end{equation}
%
which, again, can be used to establish the axis of rotation relative to CG
(see Figure~\ref{fig:screw-like motion})
%
\begin{equation}
{\tf n}_1 \;=\; \frac{\tf\omega}{\|\tf\omega\|},\quad
{\tf n}_3 \;=\; \frac{\dot{\tf u}_\CGtiny\times{\tf n}_1}{\|\dot{\tf u}_\CGtiny
\times{\tf n}_1\|},\quad
{\tf n}_2 \;=\; {\tf n}_3\times{\tf n}_1
\end{equation}
%
The axis of rotation can then be established as
%
\begin{equation}
{\tf r}_\CRtiny \;=\; r_\CRtiny{\tf n}_3 \quad\mbox{where}\quad
r_\CRtiny \;=\; \frac{\dot{\tf u}_\CGtiny \cdot {\tf n}_2}{\|{\tf\omega}\|}
\end{equation}

\subsection{Rigid body inertia forces}
\label{subs:Rigid body inertia forces}

From the rigid body velocities established in \eqnref{eq:rigid velocity}
and the rigid body properties in \eqnref{eq:Icg},
the forces about the Center of Rotation (CR) can be established as follows
%
\begin{eqnarray}
\mbox{Force:~~}  {\tf f}_{RB} &=&
m\,{\tf\omega}\times\left({\tf\omega}\times{\tf r}_\CRtiny\right) \\
\mbox{Torque:~~~}{\tf t}_{RB} &=&
\dot{\tf L}_\CRtiny = \dot{\tf L}_\CGtiny =
\tf\omega\times\left({\tf I}_\CGtiny\cdot\tf\omega\right)
\end{eqnarray}

\subsection{FE Resultant inertia forces}
\label{subs:FE Resultant inertia forces}

Having established the rigid body velocity vector, the associated FE
acceleration vector for this $\ddot{\mf v}_{\omega_\CRtiny}$ can be established.

The virtual work equation about the CR is
%
\begin{eqnarray}
\mbox{Force:~~~} {\mf f}_{FE} &=&
\delta{\mf v}_f^T {\mf M} \ddot{\mf v}_{\omega_\CRtiny} \label{eq:Ffe} \\
\mbox{Torque:~~~}{\mf t}_{FE} &=&
\delta{\mf v}_t^T {\mf M} \ddot{\mf v}_{\omega_\CRtiny}
\end{eqnarray}
%
where $\delta{\mf v}_f$ contains virtual displacement vectors for unit
translations in $X$, $Y$ and $Z$, whereas $\delta{\mf v}_t$ contains virtual
displacement vectors for rotations $X$, $Y$ and $Z$ about CR.

\subsection{Correction to FE inertia forces}
\label{subs:Correction to FE inertia forces}

For models with few DOFs, there is often a discrepancy between the forces
resulting from the FE mass matrix and the rigid body force resultants.
The correction of the FE resultant forces should then be
%
\begin{eqnarray}
\mbox{Force:~~~} {\mf f}_C &=& {\mf f}_{RB} - {\mf f}_{FE} \label{eq:Fc} \\
\mbox{Torque:~~~}{\mf t}_C &=& {\mf t}_{RB} - {\mf t}_{FE} \label{eq:Tc}
\end{eqnarray}

Forces and torques must now be added at the available nodes in such a way that
the resultants after the correction are equal to those of \eqsref{eq:Fc}{eq:Tc}.
The solution to this is not unique, but from an empirical or heuristic
viewpoint, the following procedure has been adopted:
%
\begin{enumerate}
\item The resultant force correction is evenly distributed to each node $i$
on the superelement
%
\begin{equation}
{\mf f}_{C_i} \;=\; \frac{1}{n_{\mbox{\tiny nodes}}}{\mf f}_C
\label{eq:Fci}
\end{equation}
%
This, of course, alters the torque about CR, which requires additional
compensation when correcting the resultant torque.
%
\item The new FE resultant torque, which includes the forces above, becomes
%
\begin{equation}
{\mf t}_{FE_2} \;=\; \delta{\mf v}_t^T \left(
{\mf M}\ddot{\mf v}_{\omega_\CRtiny} + {\mf f}_{\mbox{\tiny corr}}
\right)
\end{equation}
%
when combining the forces from \eqsref{eq:Ffe}{eq:Fci}.
The resultant corrected torque is
%
\begin{equation}
{\mf t}_{C_2} \;=\; {\mf t}_{RB} - {\mf t}_{FE_2}
\end{equation}
%
and the accompanying nodal correction torque is
%
\begin{equation}
{\mf t}_{C_i} \;=\; \frac{1}{n_{\mbox{\tiny nodes}}}{\mf t}_{C_2}
\label{eq:Tci}
\end{equation}
%
\item Combining the \eqsref{eq:Fci}{eq:Tci}
into the superelement correction vector results in
%
\begin{equation}
{\mf F}^{I_c} \;= \left[\begin{array}{c}
{\mf f}^{C_1} \\ {\mf t}_{C_1} \\ \vdots \\ {\mf f}_{C_n} \\ {\mf t}_{C_n}
\end{array}\right]
\end{equation}
%
\end{enumerate}

\remark{Models with high-speed rotation components are sometimes integrated more
accurately using standard Newmark rather than Newmark with numerical damping.
Using a low damping factor for the HHT-$alpha$ method improves accuracy.
See Section~\ref{s:Newmark time integration}
and~\ref{s:Newmark integration with numerical damping} with remarks.}

}
