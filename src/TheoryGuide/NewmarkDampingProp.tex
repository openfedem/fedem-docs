% SPDX-FileCopyrightText: 2023 SAP SE
%
% SPDX-License-Identifier: Apache-2.0
%
% This file is part of FEDEM - https://openfedem.org

%%%%%%%%%%%%%%%%%%%%%%%%%%%%%%%%%%%%%%%%%%%%%%%%%%%%%%%%%%%%%%%%%%%%%%%%%%%%%%%%
%
% FEDEM Theory Guide.
%
%%%%%%%%%%%%%%%%%%%%%%%%%%%%%%%%%%%%%%%%%%%%%%%%%%%%%%%%%%%%%%%%%%%%%%%%%%%%%%%%

\subsection{Numerical characteristics of the HHT-$\alpha$ method}

The Hilber--Hughes--Taylor $\alpha$-method introduces numerical damping for the
higher frequencies, which is very beneficial for the numerical stability of the
time integration.
However, ``high frequency'' is relative to the time step size being used.
When using a time step with a sampling rate of 300 Hz, 50-60 Hz is a fairly high
frequency compared to the sampling frequency, and thus has a fairly significant
numerical damping.
With a time step of 800 Hz, the numerical damping at 50-60 Hz is much smaller.
The numerical properties of the HHT-$\alpha$ method is described in more detail
in~\cite{GeradinRixen}.

The damping ratio of the HHT-$\alpha$ method is a function of the parameter
$\omega h$, see Figure~\ref{fig:HHTdamping}.
Reduction in the time step size, $h$, will move the damping ratio up in the
frequency range with the inverse factor of the time step reduction, e.g.,
reducing the time step size with a factor of 0.1 will move the numerical
damping up in the frequency range with a factor of 10.

\begin{figure}[p]
\center{
\setlength{\unitlength}{1mm}
\begin{picture}(107,72)(-15,-20)
\thinlines
% Axis system
\put( 0,-10){\line(1,0){90}}
\put( 0, 50){\line(1,0){90}}
\put( 0,-10){\line(0,1){60}}
\put(90,-10){\line(0,1){60}}
\multiput( 0,-11)(20, 0){5}{\line(0,1){2}}
\multiput( 0, 49)(20, 0){5}{\line(0,1){2}}
\multiput(-1,-10)( 0,10){7}{\line(1,0){2}}
\multiput(89,-10)( 0,10){7}{\line(1,0){2}}
%
\put( 25,-20){\textsl{Frequency parameter, $\omega h$}}
\put(-15,  5){\rotatebox{90}{\textsl{Relative damping ratio}}}
\put( -2,-15){$0.0$}
\put( 18,-15){$0.5$}
\put( 38,-15){$1.0$}
\put( 58,-15){$1.5$}
\put( 78,-15){$2.0$}
\put(-12,-11){$-0.005$}
\put(-10, -1){$0.000$}
\put(-10,  9){$0.005$}
\put(-10, 19){$0.010$}
\put(-10, 29){$0.015$}
\put(-10, 39){$0.020$}
\put(-10, 49){$0.025$}
%
\thicklines
\qbezier     (5,45)(10,45)(15,45)\put(17,44){\text{HHT $(\alpha = 0.10)$}}
\qbezier[ 10](5,40)(10,40)(15,40)\put(17,39){\text{HHT $(\alpha = 0.05)$}}
\qbezier[  5](5,35)(10,35)(15,35)\put(17,34){\text{Newmark $(\alpha = 0.0)$}}
\qbezier[ 43](0.0,0.0) (42.5,0.0) (80.0,0.0)
\qbezier[100](0.0,0.0) (26.6,0.0) (80.0,24.6)
\qbezier     (0.0,0.0) (26.6,0.0) (80.0,43.4)
\end{picture}
}
\caption{Numerical damping ratio for HHT and Newmark algorithm.}
\label{fig:HHTdamping}
\end{figure}

The frequency of the response has an error which also is dependent on the
frequency parameter $\omega h$.
This is shown in Figure~\ref{fig:HHTfrequency}.
Note, however, that the error is largely independent of the amount of numerical
damping introduced through the parameter $\alpha$.
In order to capture the frequency with sufficient accuracy we use a sufficiently
small time step.

\begin{figure}[p]
\center{
\setlength{\unitlength}{1mm}
\begin{picture}(107,80)(-15,-10)
\thinlines
% Axis system
\put( 0,  0){\line(1,0){90}}
\put( 0, 65){\line(1,0){90}}
\put( 0,  0){\line(0,1){65}}
\put(90,  0){\line(0,1){65}}
\multiput( 0, -1)(20, 0){5}{\line(0,1){2}}
\multiput( 0, 64)(20, 0){5}{\line(0,1){2}}
\multiput(-1,  0)( 0,10){7}{\line(1,0){2}}
\multiput(89,  0)( 0,10){7}{\line(1,0){2}}
%
\put( 25,-10){\textsl{Frequency parameter, $\omega h$}}
\put(-15, 10){\rotatebox{90}{\textsl{Relative periodicity error}}}
\put( -2, -5){$0.0$}
\put( 18, -5){$0.5$}
\put( 38, -5){$1.0$}
\put( 58, -5){$1.5$}
\put( 78, -5){$2.0$}
\put(-10, -1){$0.00$}
\put(-10,  9){$0.05$}
\put(-10, 19){$0.10$}
\put(-10, 29){$0.15$}
\put(-10, 39){$0.20$}
\put(-10, 49){$0.25$}
\put(-10, 59){$0.30$}
%
\thicklines
\qbezier     (5,55)(10,55)(15,55)\put(17,54){\text{HHT $(\alpha = 0.10)$}}
\qbezier[ 10](5,50)(10,50)(15,50)\put(17,49){\text{HHT $(\alpha = 0.05)$}}
\qbezier[  5](5,45)(10,45)(15,45)\put(17,44){\text{Newmark $(\alpha = 0.0)$}}
\qbezier[ 43](0.0,0.0) (28.7,0.0) (80.0,54.0)
\qbezier[100](0.0,0.0) (28.7,0.0) (80.0,58.0)
\qbezier     (0.0,0.0) (28.7,0.0) (80.0,61.6)
\end{picture}
}
\caption{Relative periodicity error for HHT and Newmark algorithm.}
\label{fig:HHTfrequency}
\end{figure}

\remark{
As a rule of thumb, one should use a time step of one tenth of the highest
response frequency one wants capture with a reasonable degree of accuracy.}
