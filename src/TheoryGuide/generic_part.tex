% SPDX-FileCopyrightText: 2023 SAP SE
%
% SPDX-License-Identifier: Apache-2.0
%
% This file is part of FEDEM - https://openfedem.org

%%%%%%%%%%%%%%%%%%%%%%%%%%%%%%%%%%%%%%%%%%%%%%%%%%%%%%%%%%%%%%%%%%%%%%%%%%%%%%%%
%
% FEDEM Theory Guide.
%
%%%%%%%%%%%%%%%%%%%%%%%%%%%%%%%%%%%%%%%%%%%%%%%%%%%%%%%%%%%%%%%%%%%%%%%%%%%%%%%%

\section{Generic part element}
\label{sec:GenericPart}

A generic part consists of a rigid spider attached to
a Center of Gravity (CG) node.
Each spider leg spans the distance from the CG node to
one of the other element nodes.
At the end of each leg there is a linear spring (equivalent to the BUSH element
described in Section~\ref{s:BUSH}) with some stiffness $k_t$ against translation
in all directions, and stiffness $k_r$ against rotation in all three directions.
For two co-located nodes ($i$ and $j$) the overall stiffness
of one such spring element is then given by
%
\begin{equation}
\label{eq:leg stiffness}
\left[\begin{array}{l}
{\mf f}_i \\ {\mf m}_i \\ {\mf f}_j \\ {\mf m}_j
\end{array}\right] =
\left[\begin{array}{rrrr}
 k_t {\mf I} &      {\mf 0} & -k_t {\mf I} &      {\mf 0} \\
     {\mf 0} &  k_r {\mf I} &      {\mf 0} & -k_r {\mf I} \\
-k_t {\mf I} &      {\mf 0} &  k_t {\mf I} &      {\mf 0} \\
     {\mf 0} & -k_r {\mf I} &      {\mf 0} &  k_r {\mf I}
\end{array}\right]
\left[\begin{array}{l}
{\mf v}_i \\ {\tf \theta}_i \\ {\mf v}_j \\ {\tf \theta}_j
\end{array}\right]
\end{equation}

With node $j$ being rigidly attached to the CG node,
one can establish the virtual displacement relation
%
\begin{equation}
\left[\begin{array}{c}
{\mf v}_j \\ {\tf \theta}_j
\end{array}\right] =
\left[\begin{array}{rr}
{\mf 1} & -\widehat{{\mf e}} \\
{\mf 0} & {\mf 1}
\end{array}\right]
\left[\begin{array}{c}
{\mf v}_\textit{CG} \\ {\tf \theta}_\textit{CG}
\end{array}\right]
\quad\text{where}\quad {\mf e} =
\left[\begin{array}{c}
x_j - x_\textit{CG} \\
y_j - y_\textit{CG} \\
z_j - z_\textit{CG}
\end{array}\right]
\end{equation}

Using the kinematic relationship above in a virtual work expression, one
can establish the stiffness matrix for the spider leg between element node $i$
and the CG node as
%
\begin{equation}
\label{eq:spider stiffness}
\left[\begin{array}{l}
{\mf f}_i \\ {\mf m}_i \\ {\mf f}_\textit{CG} \\ {\mf m}_\text{CG}
\end{array}\right] =
\left[\begin{array}{cccc}
 {\mf k}_t & {\mf 0}   & -{\mf k}_t &  {\mf k}_t \hat{{\mf e}} \\
 {\mf 0}   & {\mf k}_r &  {\mf 0}   & -{\mf k}_r \\
-{\mf k}_t & {\mf 0}   &  {\mf k}_t & -{\mf k}_t \hat{{\mf e}} \\
 \hat{{\mf e}}^T {\mf k}_t & -{\mf k}_r &
-\hat{{\mf e}}^T {\mf k}_t &
 ({\mf k}_r + \hat{{\mf e}}^T{\mf k}_T \hat{{\mf e}})
\end{array}\right]
\left[\begin{array}{l}
{\mf v}_i \\ {\tf \theta}_i \\ {\mf v}_\textit{CG} \\ {\tf \theta}_\text{CG}
\end{array}\right]
\end{equation}
%
where ${\mf k}_t = k_t{\mf I}$ and ${\mf k}_r = k_r{\mf I}$.

The full stiffness matrix for the generic part is then formed by assembling the
nodal contributions from all such spider legs.
Rewriting \eqnref{eq:spider stiffness} with the more compact notation
%
\begin{equation}
\left[\begin{array}{c}
{\mf f}^i_1 \\ {\mf f}^i_2
\end{array}\right] =
\left[\begin{array}{cc}
 {\mf k}^i_{11} &  {\mf k}^i_{12} \\
 {\mf k}^i_{21} &  {\mf k}^i_{22}
\end{array}\right]
\left[\begin{array}{c}
{\mf v}^i_1 \\ {\mf v}^i_2
\end{array}\right]
\end{equation}
%
the assembly of \eqnref{eq:spider stiffness} for all nodes from $1$ to $n$
(with $n$ being the CG node) gives the following generic part stiffness matrix
%
\begin{equation}
\label{eq:generic part stiffness matrix}
\left[\begin{array}{c}
{\mf f}^1_1 \\ \vdots \\ \;\;\;{\mf f}^{n-1}_1 \\
\sum\limits_{j=1}^{n-1} {\mf f}^j_2
\end{array}\right] =
\left[\begin{array}{cccc}
{\mf k}^1_{11} & \hdots & {\mf 0}           & {\mf k}^1_{12} \\
\vdots         & \ddots &                   & \vdots         \\
{\mf 0}        &        & {\mf k}^{n-1}_{11} & \;\;\;{\mf k}^{n-1}_{12} \\
{\mf k}^1_{21} & \hdots & {\mf k}^{n-1}_{21} &
\sum\limits_{j=1}^{n-1} {\mf k}^j_{22}
\end{array}\right]
\left[\begin{array}{l}
{\mf v}^1_1 \\ \;\vdots \\ {\mf v}^{n-1}_1 \!\!\\[2mm] {\mf v}^n_2 \\[2mm]
\end{array}\right]
\end{equation}

The mass properties of the generic part element are assumed concentrated at the
CG node, both with respect to translation and rotation.
This gives the mass matrix defined in the relationship
%
\begin{equation}
\label{eq:generic part mass matrix}
\left[\begin{array}{l}
{\mf f}^1_1 \\ \;\vdots \\ {\mf f}^{n-1}_1 \!\!\\ {\mf f}^n_2
\end{array}\right] =
\left[\begin{array}{cccc}
{\mf 0} & \hdots & {\mf 0} & {\mf 0} \\
\vdots  & \ddots &         & \vdots  \\
{\mf 0} &        & {\mf 0} & {\mf 0} \\
{\mf 0} & \hdots & {\mf 0} & {\mf m}^n_{22}
\end{array}\right]
\left[\begin{array}{l}
\ddot{\mf v}^1_1 \\ \;\vdots \\ \ddot{\mf v}^{n-1}_1 \!\!\\ \ddot{\mf v}^n_2
\end{array}\right]
\end{equation}
%
where
%
\begin{equation}
\label{eq:m22}
{\mf m}^n_{22} =
\left[\begin{array}{cccccc}
m & 0 & 0 & 0 & 0 & 0 \\
0 & m & 0 & 0 & 0 & 0 \\
0 & 0 & m & 0 & 0 & 0 \\
0 & 0 & 0 & I_{xx} & I_{xy} & I_{xz} \\
0 & 0 & 0 & I_{xy} & I_{yy} & I_{yz} \\
0 & 0 & 0 & I_{xz} & I_{yz} & I_{zz}
\end{array}\right]
\end{equation}

When using the generic part element with the element matrices defined by
\eqsref{eq:generic part stiffness matrix}{eq:generic part mass matrix}
in a dynamics simulation, one might get artificial oscillations in the CG node
DOFs, depending on the actual magnitude of the characteristic mass and
stiffness being used.
This problem might be avoided by eliminating the CG node DOFs through static
condensation of
\eqsref{eq:generic part stiffness matrix}{eq:generic part mass matrix},
before the time integration is started.
This is similar to what is being done for the internal DOFs of the finite
element parts in the model reduction process,
see Sections~\ref{subsec:Static modes} and~\ref{subsec:Reduced system}.

Assuming that all ${\mf f}^j_2$ in \eqnref{eq:generic part stiffness matrix}
are always zero (since no distributed loads are associated with a generic part),
the last line of \eqnref{eq:generic part stiffness matrix} yields
%
\begin{equation}
\label{eq:generic part B-matrix}
{\mf v}^n_2 = {\mf B}
\left[\begin{array}{l}
{\mf v}^1_1 \\ \;\vdots \\ {\mf v}^{n-1}_1
\end{array}\right] \quad\text{where}\quad
{\mf B} =
\left[ \sum\limits_{j=1}^{n-1} {\mf k}^j_{22} \right]^{-1}
\left[\begin{array}{cccc}
{\mf k}^1_{21} & \hdots & {\mf k}^{n-1}_{21}
\end{array}\right]
\end{equation}
%
Combining \eqnref{eq:generic part B-matrix} and its associated second-derivative
with the first $n-1$ lines of
\eqsref{eq:generic part stiffness matrix}{eq:generic part mass matrix},
respectively, while pre-multiplying with ${\mf B}^T$ produces the following
element matrices for the generic part
%
\begin{eqnarray}
{\mf k} &=&
\left[\begin{array}{cccc}
{\mf k}^1_{21} & \hdots & {\mf k}^{n-1}_{21}
\end{array}\right]^T {\mf B} +
\left[\begin{array}{ccc}
{\mf k}^1_{11} & \hdots & {\mf 0} \\
\vdots         & \ddots & \vdots  \\
{\mf 0}        & \hdots & {\mf k}^{n-1}_{11}
\end{array}\right] \\
{\mf m} &=& {\mf B}^Ti {\mf m}_{22}^n {\mf B}
\end{eqnarray}

The stiffness coefficients $k_t$ and $k_r$ that is used in
\eqnref{eq:leg stiffness} may be specified by the user for each part
(see the \FedemUG, {\em Section~4.1.5 ``Link properties''}).
However, it is also possible to let the coefficients be automatically computed,
in such a way that the generic part behaves like an ``almost'' rigid element.

The automatic ``rigid'' stiffness is computed from the mass properties of the
part and a given target eigenfrequency:
%
\begin{eqnarray}
k_t &=& \left(2\pi f_{\rm rig}\right)^2 \,m \\
k_r &=& \left(2\pi f_{\rm rig}\right)^2 \,\frac{1}{3}\sum_{i,j = x,y,z} I_{ij}
\end{eqnarray}
%
where $m$ and $I_{ij}$ are components of the mass matrix~\eqref{eq:m22},
and $f_{\rm rig}$ is the desired target eigenfrequency of the part (in [Hz]).
This target value may be set by the user through the command-line option
{\tt -targetFrequencyRigid} for the Dynamics Solver.
The default value is 10000 Hz.
