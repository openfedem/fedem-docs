% SPDX-FileCopyrightText: 2023 SAP SE
%
% SPDX-License-Identifier: Apache-2.0
%
% This file is part of FEDEM - https://openfedem.org

%%%%%%%%%%%%%%%%%%%%%%%%%%%%%%%%%%%%%%%%%%%%%%%%%%%%%%%%%%%%%%%%%%%%%%%%%%%%%%%%
%
% FEDEM Theory Guide.
%
%%%%%%%%%%%%%%%%%%%%%%%%%%%%%%%%%%%%%%%%%%%%%%%%%%%%%%%%%%%%%%%%%%%%%%%%%%%%%%%%

\chapter{Introduction}

\pagenumbering{arabic}

Fedem, an abbreviation for Finite Element Dynamics in Elastic Mechanisms,
is a code for effective modeling, simulation and visualization of finite element
assemblies and control systems. The code is based on a non-linear finite element
formulation, which predicts the dynamic response of elastic mechanisms
experiencing non-linear effects such as large rigid-body rotations.
The elastic and rigid-body motions of a mechanism are solved together
with control systems. Fedem represents in this respect
\textit{Multidisciplinary Mechanical Analysis}.

This theory guide is intended to provide users and others with insight into
the mathematical and physical basis of the numerical simulation code.

% SPDX-FileCopyrightText: 2023 SAP SE
%
% SPDX-License-Identifier: Apache-2.0
%
% This file is part of FEDEM - https://openfedem.org

%%%%%%%%%%%%%%%%%%%%%%%%%%%%%%%%%%%%%%%%%%%%%%%%%%%%%%%%%%%%%%%%%%%%%%%%%%%%%%%%
%
% FEDEM Theory Guide.
%
%%%%%%%%%%%%%%%%%%%%%%%%%%%%%%%%%%%%%%%%%%%%%%%%%%%%%%%%%%%%%%%%%%%%%%%%%%%%%%%%

\section{History}

The theory behind the FEDEM solvers was originally developed by
the late Professor Ole Ivar Sivertsen in the late 1970s and through the 1980s.
His work initiated new Ph.D.\ studies and international R\&D projects
that contributed to the development of the first FEDEM software product.

On the basis of Professor Sivertsen's work, a company was established in 1992
by Sintef, Northern Europe's largest R\&D institute based in Trondheim, Norway.
Computer speed reached levels that would allow the theories to produce results,
and visualization technology made it possible to create a user interface.
During this period the FEDEM software was strictly an inhouse code at Sintef.

In 1995 the company Fedem, which later became Fedem Technology,
continued the development of the user interface and made it possible to offer
FEDEM as a commercial product in 1998.

The FEDEM software was continously developed as a product by Fedem Technology
during the 2000s and 2010s while it was also used as an internal tool in
various consultancy projects, until the company was aquired by SAP SE in 2016.
Since then, the FEDEM solvers have been provided as components in the
EPD Connected Products by SAP, until the sunsetting of the Connected Products
in 2023. It was then decided to release Fedem under a open source license
on github, as a service to the existing user community.


\section{System Simulation Methods}

One method of System Simulation is the so-called Multi Body System Simulation, with which 
the real physical behavior of the product under investigation can be reduced 
to a few - for instance overall behavior relevant - characteristics and then 
can be simulated as a numerical model on the computer. Each body is usually
treated as completely rigid.

The Finite Element Method is used for instance 
for a vehicle to calculate fatigue, stiffness, dynamical behavior of the car 
body, of chassis components, of engines.  
The size and the shape of the Finite Elements are chosen according to the
required accuracy of the results. For many years the Finite Element Method has 
been a successful tool for product analysis with respect to 
functionality and safety.

New simulation tools are now commercially available, the so-called 
Multi Discipline Simulation tools, which combine Multi Body System 
Simulation, the Finite Element Method and Control Engineering. These 
computer programs make a much more detailed modeling possible and 
correspondingly yield much more accurate Multi Body System Simulation results.

\section{Terminology and Definitions}

Most of the terms and names used within Fedem is quite standard within Finite Element technology
and dynamic simulation. However, over the years certain terms have evolved among users and
developers of the software, to a point where the terminology has become an integral part of
the software product. 

\begin{itemize}

\item[\bf DOF]
Degree of Freedom. For the mechanism models of Fedem, a DOF is usually a translational
or rotational degree of freedom. Regarding a control system a DOF is defined more broadly as
simply an unknown of the equation system to be solved.

\item[\bf Triad]
Numerically a triad is simply a node with displacement DOFs. The term has
been coined within the modeling frame to better to describe the modeling entity that eventually
results in a node in the computational model. Because of its frequent use within the modeling
vocabulary it has also come to be used synonymously with nodes within the numerical/theoretical
vocabulary of Fedem.

\item[\bf Link]
Regarding numerical algorithms a link is a superelement. The element is linear within
its corotated coordinate system. This term also has its background from the modeling
side of Fedem and has since become a part of the theoretical vocabulary.

\item[\bf Model]
A model is defined as a more or less simplified representation 
of a system. The model must represent the properties of a system being studied,
as accurate as possible.

\item[\bf Simulation]
Imitation of certain properties of a (mechanical) system in a computational model.

\item[\bf Kinematic simulation]
Kinematic simulation refers to calculations of motion in a system 
with no reference to forces and torques necessary to achieve this motion.

\item[\bf Dynamic simulation]
Dynamic simulation refers to the calculations of motion in a system 
where both constraint forces and forces necessary to drive the system are 
taken into account.

\item[\bf Multi discipline dynamic simulation]
The multi discipline dynamic simulation concept refers to a 
simulation model where the overall system is modeled as a mechanism, the 
parts are modeled as Finite Element substructures, and combined with a 
possible control system, all is integrated in the same simulation model.
A control system may include controllers, actuators and sensors (to model feedback loops).

\item[\bf System simulation]
The term system simulation incorporates the three terms 
\textit{kinematic simulation}, \textit{dynamic simulation} and 
\textit{multi discipline dynamic simulation} described above.
\end{itemize}
