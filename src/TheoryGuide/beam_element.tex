% SPDX-FileCopyrightText: 2023 SAP SE
%
% SPDX-License-Identifier: Apache-2.0
%
% This file is part of FEDEM - https://openfedem.org

%%%%%%%%%%%%%%%%%%%%%%%%%%%%%%%%%%%%%%%%%%%%%%%%%%%%%%%%%%%%%%%%%%%%%%%%%%%%%%%%
%
% FEDEM Theory Guide.
%
%%%%%%%%%%%%%%%%%%%%%%%%%%%%%%%%%%%%%%%%%%%%%%%%%%%%%%%%%%%%%%%%%%%%%%%%%%%%%%%%

\section{BEAM2}
\label{s:BEAM2}

The BEAM2 element is based on Euler-Bernoulli's beam theory with quadratic shape
functions and continuous first derivatives.
The deformations account for are bending, shear, axial and St.\ Venant torsion.
The element is straight with uniform cross section and material properties.
The cross section does not need to be symmetrical.
The 2 nodal points of the element, one at each end, may be offset in relation
to the principle axis.

Figure~\ref{fig:BEAM2} shows an arbitrary beam element referred to in a link
coordinate system $(X, Y, Z)$ and a local system $(x, y, z)$.
The local $x$-axis coincides with the beam axis through the center of gravity of
the cross sections, and is positive in the direction from point $I$ to $J$.
The local $y$- and $z$-axes coincide with the principal axes of the cross section.
An auxiliary point $K$ defines together with the beam axis the local $xz$-plane.
The local $z$-axis is positive in the direction from the element toward point $K$.
The end points $I$ and $J$ are connected, via fictitious rigid eccentricities,
to the nodal points $II$ and $JJ$, respectively, at which nodal parameters are defined.

\begin{figure}[t]
\begin{center}
\setlength{\unitlength}{1mm}
\begin{picture}(100,75)
\thinlines
\put(15,15){\vector( 1, 0){70}}
\put(15,15){\vector( 0, 1){55}}
\put(15,15){\vector(-1,-1){12}}
\put( 0, 0){$x_{\text{link}}$}
\put(87,14){$y_{\text{link}}$}
\put(13,72){$z_{\text{link}}$}
\thicklines
\qbezier(30,30)(32,30.5) (34,31)
\qbezier(30,30)(30,32)   (30,34)
\qbezier(30,34)(31,34.25)(32,34.5)
\qbezier(34,31)(33,32.75)(32,34.5)
\put(31.5,32.375){\circle*{1.5}}\put(31,27){$I$}
\put(34,31)  {\line(3,2){35}}
\put(32,34.5){\line(3,2){35}}
\put(30,34)  {\line(3,2){35}}
\qbezier(65,57.33)(66,57.58)(67,57.83)
\qbezier(69,54.33)(68,56.08)(67,57.83)
\put(66.5,55.708){\circle*{1.5}}\put(66,60){$J$}
\thinlines
\put(66.5,55.708){\vector(3,2){15}}\put(83,65){$x_{\text{elem}}$}
\put(31.5,32.375){\vector(-1,4){5}}\put(25,55){$y_{\text{elem}}$}
\put(31.5,32.375){\vector(4,1){40}}\put(73,42){$z_{\text{elem}}$}
\put(66.5,45){\circle*{1.5}}\put(68,45){$K$}
\put(66.5,55.708){\line(0,-1){14.5}}
%Eccentricity at node I to II
\put(31.5,32.375){\line(-1,-1){10}}
\put(21.5,22.375){\line( 1, 0){25}}
\put(46.5,22.375){\line( 0, 1){ 5}}
\put(46.5,27.375){\circle{1.5}}\put(48,26){$II$}
\put(21,27){$e_x$}
\put(32,19){$e_y$}
\put(48,23){$e_z$}
%Eccentricity at node j to JJ
\put(66.5,55.708){\line(-1,-1){5}}
\put(61.5,50.708){\line( 1, 0){15}}
\put(76.5,50.708){\line( 0, 1){ 5}}
\put(76.5,55.708){\circle{1.5}}\put(78,54){$JJ$}
\put(78,50){\scriptsize{fictitious rigid arm}}
%Nodal dofs
%\thinlines
%\put(70,40){\vector(1,0){6}}    \put(79,39){$v$}
%\put(70,40){\vector(0,1){6}}    \put(70,49){$w$}
%\put(70,40){\vector(-1,-1){3.6}}\put(64,37){$u$}
\end{picture}
\end{center}
\caption{BEAM2, Beam element}
\label{fig:BEAM2}
\end{figure}

The BEAM2 element may optionally be equipped with pin flags in end $I$ and/or $J$.
They are used to remove connections between the associated grid point and
selected DOFs of the beam defined in the local element coordinate system.
Thus, they work like inserting a local hinge in the beam for the selected DOFs.
The beam must have stiffness associated with the DOFs that are released in this
manner, e.g.\ if local DOF 4 is released in one end, the beam must have a
nonzero torsional stiffness.

\subsection{Spot weld element}
\label{subs:Spot weld element}

The BEAM2 element described above is also used to represent spot welds\footnote{
Known as CWELD elements in Nastran.} in Fedem.
A circular massive cross section is assumed for a spot weld beam.
An explicit local $z$-axis definition is therefore not needed for such elements.

Instead of the rigid arms, a spot weld BEAM2 element is equipped with a WAVGM
element in each end (see Section~\ref{s:WAVGM} below), in order to distribute
the forces transferred by the beam over a group of nodes in the welded surfaces.
The end points of the beam, which are connected to the reference node of the
WAVGM elements, are then defined by the projection of a specified point
onto each of the 2 welded surfaces.
The WAVGM elements may span an arbitrary number of nodes, but typically they
are connected to all surface nodes of the finite element that is intersected by
the spot weld beam.

Since the welded surfaces typically are quite close, the actual length
of the spot weld beam will be relatively small (or maybe zero).
However, it is possible to equip the spot weld beam with a separate effective
length property to be used instead of its actual length when calculating the
element stiffness.
