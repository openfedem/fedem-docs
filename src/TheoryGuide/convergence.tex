% SPDX-FileCopyrightText: 2023 SAP SE
%
% SPDX-License-Identifier: Apache-2.0
%
% This file is part of FEDEM - https://openfedem.org

%%%%%%%%%%%%%%%%%%%%%%%%%%%%%%%%%%%%%%%%%%%%%%%%%%%%%%%%%%%%%%%%%%%%%%%%%%%%%%%%
%
% FEDEM Theory Guide.
%
%%%%%%%%%%%%%%%%%%%%%%%%%%%%%%%%%%%%%%%%%%%%%%%%%%%%%%%%%%%%%%%%%%%%%%%%%%%%%%%%

\subsection{Convergence criteria}
\label{subs:Convergence criteria}

Some criteria for terminating the equilibrium iterations governed by
\meqsref{eq:NewtonDynEq}{eq:317}, must be established.
They are typically based on some {\it norms} of the vector quantities involved.
When computing the norms, we have the problem of dealing with different units.
There are rotational and translational DOFs (typically measured in radians and
meters), as well as generalized DOFs associated with the component modes.

Changing the modeling units of length and translation from [m] to [mm] changes
the size of the translations with 3 orders of magnitude, whereas the rotations
remain the same.
Such a change of units should not affect the relative contribution of rotation
and translations to the norm of a displacement vector.

\subsubsection{Scaled vector norms}

In order to make the vector norms unit independent, we choose to use rotations
as the ``base'' displacement quantity since they are always measured in radians.
With a rigid rotation of unit magnitude as guiding displacement, an appropriate
scaling of the translations is then assumed to be $\frac{1}{L_\text{model}}$,
where $L_\text{model}$ is the largest dimension of the model itself.

Based on this, we define the scaled vector norm as
%
\begin{equation}
% Corrected by KMO 19.08.04 (see TT #2074)
\left\| {\mf v} \right\|_\text{scaled} \;=\;
\sqrt{\frac{\sum_i(w_i v_i)^2}{\sum_i w_i^2}}
\end{equation}
%
\begin{equation*}
\text{where}\quad\left\{\begin{array}{ll}
w_i = 1 & \text{if $i$ is rotational DOF} \\
w_i = \frac{1}{L_\text{model}} & \text{if $i$ is translational DOF} \\
w_i = 1 & \text{if $i$ is generalized DOF}
\end{array}\right.
\end{equation*}
%
Convergence criterias are now defined as
%
\begin{equation}
\left\| {\mf v} \right\|_\text{scaled} \leq \varepsilon_\textit{tol}
\end{equation}
%
where the vector ${\mf v}$ can be one of the following
%
\clearpage
\begin{namelist}{XXX}
%
\item[$\ls{i}\dr_k$] displacement correction
defined by \eqsref{eq:NewtonDynEq}{eq:NewmarkDispCorr}
%
\item[$\ls{i}\ddr_k$] velocity correction
defined by \eqsref{eq:NewtonDynEq}{eq:NewmarkVelAndAccCorr}\footnote{
For the scaled vector norm of the velocity correction we also have the option of
``relaxing'' the convergence criterion when the overall velocity of the
mechanism is large, i.e.,
\begin{equation*}
\varepsilon_{tol} \;=\; \varepsilon_{abs} +
\varepsilon_{vel}\left\|\dot{\mf v}\right\|
\end{equation*}}
%
\item[$\ls{i}\dddr_k$] acceleration correction
defined by \eqsref{eq:NewtonDynEq}{eq:NewmarkVelAndAccCorr}
%
\item[$\ls{i}\hat{\mf R}_k$] residual vector (unbalanced forces)
defined by \eqsref{eq:NewtonDynEq}{eq:residual}
%
\end{namelist}
%
The scaling factors are inverted for the force residual vector.

\remark{Velocity and acceleration corrections are computed from the displacement
correction as time step dependent scalings proportional to
$\frac{1}{h}$ and $\frac{1}{h^2}$ respectively,
as given by \eqnref{eq:NewmarkVelAndAccCorr}.
For small time steps, any noise in the displacement corrections gets greatly
magnified in the velocity and acceleration terms, and can lead to convergence
problems if the convergence criterion involves testing on the
velocity and acceleration corrections.}

\subsubsection{DOF type selective infinite norms}

The infinite norm of a vector is defined as
%
\begin{equation}
\left\|{\mf v}\right\|_{\infty} \;=\; \lim_{n\to\infty} \sqrt[n]{\sum_i |v_i^n|}
 \;=\; \max_{\forall i} |v_i|
\end{equation}
%
Because of the different DOF types (rotation, translation and component modes),
we define three different infinite norms of a mixed vector ${\mf v}$ as
%
\begin{equation}
\begin{array}{lcll}
\left\| {\mf v} \right\|_{\infty,\text{tran}} &=& \max|v_i| &
:\;\forall\; i \text{~being a translational DOF} \\
\left\| {\mf v} \right\|_{\infty,\text{rot}} &=& \max|v_i| &
:\;\forall\; i \text{~being a rotational DOF} \\
\left\| {\mf v} \right\|_{\infty,\text{gen}} &=& \max|v_i| &
:\;\forall\; i \text{~being a component mode DOF}
\end{array}
\end{equation}
%
Convergence criterias based on these norms are now defined as
%
\begin{equation}
\left\|{\mf v}\right\|_{\infty,\text{tran}} \leq \varepsilon_\textit{tol},\quad
\left\|{\mf v}\right\|_{\infty,\text{rot}} \leq \varepsilon_\textit{tol},\quad
\left\|{\mf v}\right\|_{\infty,\text{gen}} \leq \varepsilon_\textit{tol}
\end{equation}
%
where ${\mf v}$ can be the correction to the displacements $\ls{i}\dr_k$,
velocity $\ls{i}\ddr_k$, acceleration $\ls{i}\dddr_k$,
as well as the residual vector $\ls{i}\hat{\mf R}_k$.

\subsubsection{Energy norms}

The product of residual vector $\ls{i}\hat{\mf R}_k$ and displacement
correction $\ls{i}\dr_k$ form an incremental energy term.
The advantage of using this quantity in convergence testing is the automatic
same unit of energy for all DOFs regardless of translation/force,
rotation/torque, or generalized DOF/generalize force.

We define energy norm analogous to the infinite and scaled vector norms
respectively, as the max DOF energy and average energy
%
\begin{eqnarray}
\text{Max DOF energy:} &
E_{\max} =& \max_{j=1,n}
\left| \ls{i}\delta r_{kj} \ls{i}\hat{R}_{kj} \right| \\
\text{Average energy:} &
E_\text{avg} \;=& \frac{1}{n} \sum_{j=1,n}
\left| \ls{i}\delta r_{kj} \ls{i}\hat{R}_{kj} \right|
\end{eqnarray}
%
\remark{Structures close to a buckling state will often be close to force
equilibrium (low residual norm), while the displacements are largely
undetermined (high displacement correction norm).
Similarly, structures with very stiff members often are close to the correct
position (low displacement correction norm), whereas the unbalanced forces are
high (high residual norm).
Both of these cases can lead to convergence problems when the convergence
criterion is based on displacements and/or force equilibrium.
In these cases the energy norms often give a more ``balanced'' picture and can
improve stability in passing problem areas in the dynamics simulation.}
