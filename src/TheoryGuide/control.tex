% SPDX-FileCopyrightText: 2023 SAP SE
%
% SPDX-License-Identifier: Apache-2.0
%
% This file is part of FEDEM - https://openfedem.org

%%%%%%%%%%%%%%%%%%%%%%%%%%%%%%%%%%%%%%%%%%%%%%%%%%%%%%%%%%%%%%%%%%%%%%%%%%%%%%%%
%
% FEDEM Theory Guide.
%
%%%%%%%%%%%%%%%%%%%%%%%%%%%%%%%%%%%%%%%%%%%%%%%%%%%%%%%%%%%%%%%%%%%%%%%%%%%%%%%%

\chapter{Control System}
\label{c:Control System}

Mechanisms are often connected to or acted upon by items such as sensors,
controllers, and actuators; therefore, a need exists for multidisciplinary
mechanism and/or finite-element control simulation.
Consequently, a control system has been developed for simulation in a composite
system.

\section{Problem statement}

Elements have been designed to model these sensors, controllers, actuators,
and so on, which are referred to as control elements or control blocks.
Control elements are actually functions, also referred to as control equations,
that describe the modeled element.
The control elements can be connected together with input and output blocks to
perform more complex operations.
Once connected the elements are referred to as a control module.

Separate numerical methods are used for the structural and control calculations,
so that only minor changes are imposed on the FEM part.

The structural equation constitutes a $2^{\text{nd}}$ order system for the
linear case and can be written as:
%
\begin{equation}
{\mf M} \ddot{\mf r} + {\mf C} \ddot{\mf r} + {\mf Kr} \;=\; {\mf Q}(t)
\label{eqC:91}
\end{equation}
%
where ${\mf r}$ is the vector of displacement, ${\mf M}$, ${\mf C}$ and
${\mf K}$ are matrices for mass, damping, and stiffness, respectively, and
${\mf Q}(t)$ is a vector of time-dependent forces acting on the structure.

The control equations can usually be written in the following form:
%
\begin{eqnarray}
\dot{\mf x} &=& {\mf f}(t,{\mf u},{\mf x},{\mf z}) \\
    {\mf O} &=& {\mf g}(t,{\mf u},{\mf x},{\mf z})
\label{eqC:92}
\end{eqnarray}
%
where $\mf u$, $\mf x$ and $\mf z$ are vectors of inputs, state variables,
and algebraic variables, respectively.
Most control elements are not explicitly time-dependent.
Exceptions are elements with inherent clock functions, such as those in a
sample and hold element or a multiplexing unit.

The structural and control calculations are coupled when some elements from the
displacement vector enter into input vector ${\mf u}$ and, in response, some of
the forces in ${\mf Q}(t)$ are taken from the control variables $\mf x$
or $\mf z$.
Any remaining inputs of $\mf u$ can be time functions,
such as a controller reference.

The system for solving the structural calculations is based on Newmark's
established $\beta$-method with (as a general rule) $\gamma=\frac{1}{2}$ and
$\beta=\frac{1}{4}$ (see~\cite{Newmark}).
These parameters correspond to the trapezoidal rule.

The dimensions of the discrete structural calculations are usually much higher
than those of the control calculations, and therefore represent the heaviest
portion of the computations.

In terms of accuracy, the structural calculations can be expected to limit the
time step more than the control calculations.
However, cases may exist in which the coupling between the structural and
control calculations limits the time step more than either of the two calculations.

The control system module integrates the variables in the
differential-algebraic system one step forward from a given state.
The solution for the control system part is iterated until there is convergence
on each invocation; therefore, these iterations form an inner loop of the
dynamic integration iterations.

\section{Control variables}

The system is divided into three types of variables: inputs, state,
and algebraic variables.
The user defines inputs, while the system automatically determines state
and algebraic variables.
The inputs are either external time functions such as a controller reference,
or sensor inputs that are not changed in the control system.

Control elements are identified by type and have a number of inputs,
internal states, and outputs.
The internal states are not connected to other modules.
In addition, there are a number of parameters the user needs to provide.
Figure~\ref{figC:91} shows a module of type $nn$, with $i$ inputs, $j$
internal states, $k$ outputs, and $m$ parameters.

The user needs to bear two things in mind concerning this system.
First, some parameters are used to carry values from one time step to another.
These parameters appear in parentheses in the library and are not provided by the user.
Second, the distinction between inputs and outputs cannot be made for some
elements before the configuration is set up.
This is known as the causality problem and may occur in algebraic relations
(see~\cite{Iversen:STF48}).
However, this problem is resolved by the system during initialization.

\begin{figure}[t]
\center
\setlength{\unitlength}{1mm}
\begin{picture}(67,40)
\thicklines
\put( 6,15){\line(10,0){9}}
\put( 6,25){\line(10,0){9}}
\put( 5,15){\circle{2}}
\put(10,18){$\vdots$}
\put( 5,25){\circle{2}}
\put( 1,14){$i$}
\put( 1,24){$1$}
\put(15,10){\framebox(30,20){General}}
\put(54,15){\line(-1,0){9}}
\put(54,25){\line(-1,0){9}}
\put(55,15){\circle{2}}
\put(50,18){$\vdots$}
\put(55,25){\circle{2}}
\put(57,14){$i+j+k$}
\put(57,24){$i+j+1$}
\put(20,35){\vector(0,-1){5}}
\put(19,36){$1$}
\put(28,32){$\cdots$}
\put(40,35){\vector(0,-1){5}}
\put(39,36){$m$}
\put(20, 6){\circle{2}}
\put(20, 7){\line(0,1){3}}
\put(17, 0){$i+1$}
\put(28, 5){$\cdots$}
\put(40, 6){\circle{2}}
\put(40, 7){\line(0,1){3}}
\put(37, 0){$i+j$}
\end{picture}
\caption{The general control module}
\label{figC:91}
\end{figure}

To configure a control scheme, the user draws a block scheme of modules from
the library, as shown in Figure~\ref{figC:92}.

\begin{figure}[t]
\center
\setlength{\unitlength}{1mm}
\begin{picture}(120,40)
\thicklines
\put( 15,27){\line(10,0){5}}
\put( 14,27){\circle{2}}
\put( 13,29){$1$}
\put( 15,19){\line(10,0){5}}
\put( 14,19){\circle{2}}
\put( 20,15){\framebox(15,16){$0$}}
\put( 45,15){\framebox(15,16){$k$}}
\put( 70,15){\framebox(30,16){$\frac{k}{\omega_c^2 + 2\xi\omega_0 s + s^2}$}}
\put( 40,23){\circle{2}}
\put( 39,25){$2$}
\put( 65,23){\circle{2}}
\put( 64,25){$3$}
\put(105,23){\circle{2}}
\put(104,25){$5$}
\put( 85,10){\circle{2}}
\put( 87, 9){$4$}
\put( 85,11){\line(0,1){4}}
\put( 35,23){\line(1,0){4}}
\put( 41,23){\line(1,0){4}}
\put( 60,23){\line(1,0){4}}
\put( 66,23){\line(1,0){4}}
\put(100,23){\line(1,0){4}}
\put(106,23){\line(1,0){4}}
\put(105,22){\line(0,-1){17}}
\put(105, 5){\line(-1,0){91}}
\put( 14, 5){\vector(0,1){13}}
\put( 52,35){\vector(0,-1){4}}
\put( 51,36){$k$}
\put( 75,35){\vector(0,-1){4}}
\put( 74,36){$k$}
\put( 85,35){\vector(0,-1){4}}
\put( 84,36){$\omega_0$}
\put( 95,35){\vector(0,-1){4}}
\put( 94,36){$\xi$}
\end{picture}
\caption{Configuration example}
\label{figC:92}
\end{figure}

\section{Control system tasks}

The control system module manages the three main activities:
Initialization, steady state computation, and integration.

\subsection{Initialization}

During initialization the task is to label the state and algebraic variables.
For the algebraic variables of \eqnref{eqC:92}, initialization determines which
variable to allocate to the function value.
Each module must be prepared for this task, which is managed during the
initialization process.
Algebraic loops or causality problems may require several iterations.

\subsection{Steady state}

With the different types of variables collected in one vector,
%
\begin{equation}
{\mf y} \;= \left[\begin{array}{c}
{\mf u} \\ {\mf x} \\ {\mf z}
\end{array}\right]
\label{eqC:93}
\end{equation}
%
the set of equations may be written:
%
\begin{equation}
{\mf F}(t,{\mf y},\dot{\mf y}) \;=
\left[\begin{array}{c}
-{\mf u} + {\mf s}(t) \\
-\dot{\mf x} + {\mf f}(t,{\mf y}) \\
{\mf g}(t,{\mf y}) \\
\end{array}\right] =
\left[\begin{array}{c}
{\mf 0} \\ {\mf 0} \\ {\mf 0}
\end{array}\right]
\label{eqC:94}
\end{equation}
%
Steady state is found by setting the derivative to zero.
Applying Newton iteration to \eqnref{eqC:94} produces the following system
which is solved at iteration number $k$:
%
\begin{equation}
\left[\begin{array}{ccc}
{\mf I} &  {\mf 0}   &  {\mf 0}   \\
{\mf 0} & -{\mf f}_x & -{\mf f}_z \\
{\mf 0} & -{\mf g}_x & -{\mf g}_z
\end{array}\right]\left[\begin{array}{c}
\Delta{\mf u}^k \\
\Delta{\mf x}^k \\
\Delta{\mf z}^k
\end{array}\right] = \left[\begin{array}{c}
{\mf 0} \\
{\mf f}(t,{\mf u},{\mf x}^k,{\mf z}^k) \\
{\mf g}(t,{\mf u},{\mf x}^k,{\mf z}^k)
\end{array}\right]
\label{eqC:95}
\end{equation}
%
A standard equation solver is used for the solution of this vector equation.

\subsection{Time integration}

Integration is performed using two numerical methods to estimate local error.

A starting point of \eqnref{eqC:94} is established for the development of the
numerical equations.
All the input components can be treated as time functions, even though some of
them are determined by the structure calculations.
The inputs are not changed in the control calculations.
Because they are labeled, the variables in vector ${\mf y}$ do not have to be be
ordered according to type; this is for convenience of development notation only.

Since Newmark's $\beta$ method is of second order, it naturally follows that a
method of second order is chosen for the control part also.
Therefore, Lobatto IIIC, an implicit, second-order Runge-Kutta method,
is selected.
Backward Euler, which is first-order and also implicit,
is used for local error estimation.
Local extrapolation is used, meaning that the integration proceeds with the
result from the higher-order method.

The general $m$-level Runge-Kutta (RK) method for the solution of
\eqnref{eqC:94} can be written:
%
\begin{eqnarray}
F(t + c_i h, Y_i, \dot{Y}_i) &=& 0 \\
\label{eqC:96}
Y_i &=& y_n + h\sum_{j=1}^m a_{ij} \dot{Y}_i \\
\label{eqC:97}
y_{n+1} &=& y_n + h\sum_{i=1}^m b_i \dot{Y}_i
\label{eqC:98}
\end{eqnarray}
%
where $h$ is the time step.
To visualize a particular method, the matrix ${\mf A}$ and the vectors ${\mf b}$
and ${\mf c}$ formed by the ${a}$, ${b}$ and ${c}$ coefficients are usually put
in a tableau as shown in Table~\ref{tabC:91}.
For backward Euler and Lobatto IIIC, we have the following Butcher tableau
given by Table~\ref{tabC:92}.

\begin{table}
\begin{center}
\begin{tabular} {c|c}
{\mf c} & {\mf A}     \\ \hline
        & {\mf b}$^T$ \\
\end{tabular}
\end{center}
\caption{Butcher tableau}
\label{tabC:91}
\end{table}

\begin{table}
\begin{center}
\begin{tabular}{c|r}
1 & 1 \\ \hline
  & 1 \\
\end{tabular} \hskip1.5cm
\begin{tabular}{c|rr}
0 & 1/2 & -1/2 \\
1 & 1/2 &  1/2 \\ \hline
  & 1/2 &  1/2 \\
\end{tabular}
\end{center}
\caption{Backward Euler and Lobatto IIIC}
\label{tabC:92}
\end{table}
%
For an RK method satisfying
%
\begin{equation}
b_i \;=\; a_{mi},\;\;\mbox{for}\;\; i=1,\dots,m
\label{eqC:99}
\end{equation}
%
a recording of the response in the variables is obtained as in the two methods above.
%
\begin{equation}
y_{n+1} \;=\; Y_m
\label{eqC:910}
\end{equation}
%
From \eqsref{eqC:99}{eqC:910} an explicit expression may be derived for Y:
%
\begin{equation}
\dot{Y}_i \;=\;
\frac{1}{h}\sum_{j=1}^m d_{ij} \left(Y_j - y_n \right),
\;\;\mbox{where}\;\;
d_{ij} = D = A^{-1}
\label{eqC:911}
\end{equation}
%
The \meqsref{eqC:96}{eqC:911} can then be replaced by:
%
\begin{eqnarray}
F\left(t + c_i h Y_i \frac{1}{h}\sum_{j=1}^m d_{ij}
\left(Y_j-y_n\right)\right) &=& 0,
\;\;\mbox{for}\;\; i = 1,\dots,m \\
\label{eqC:912}
y_{n+1} &=& Y_m
\label{eqC:913}
\end{eqnarray}
%
For Backward Euler this gives:
%
\begin{equation}
F\left(t_{n+1}, \tilde{y}_{n+1} ,\frac{1}{h}\left(
\tilde{y}_{n+1} - y_n\right)\right) \;=\; 0
\label{eqC:914}
\end{equation}
%
and for Lobatto IIIC:
%
\begin{eqnarray}
F\left(t_n,Y_1, \frac{1}{h}\left(Y_1 + Y_2 - 2y_n \right)\right) &=& 0
\label{eqC:915} \\
%
F\left(t_{n+1}, Y_2 ,\frac{1}{h}\left(-Y_1 + y_2 \right)\right) &=& 0
\label{eqC:916}
\end{eqnarray}
%
The local error estimate is then:
%
\begin{equation}
l_{n+1} = \left\| y_{n+1} - \tilde{y}_{n+1}\right\|
\label{eqC:918}
\end{equation}
%
To find the solution of the implicit numerical equations~\eqref{eqC:914},
\eqref{eqC:915} and~\eqref{eqC:916}, Newton iteration is used as in steady state.
Starting values for the iterations are:
%
\begin{eqnarray}
\tilde{y}_{n+1}^0 &=& y_n + \frac{h_{n+1}}{h_n} \left(y_n - y_{n-1}\right)
\label{eqC:919} \\
%
Y_{1,n+1}^0 &=& y_n
\label{eqC:920} \\
%
Y_{2,n+1}^0 &=& \tilde{y}_{n+1}
\label{eqC:921}
\end{eqnarray}
%
The Jacobian, which is needed in the Newton matrices, is generated numerically.
This is done by excitation of each variable in turn and recording of the
response in the other variables.

\section{Control element library}

The control elements represent basic algebraic or differential functions;
time- dependent sample and hold functions; basic linear transfer functions;
the most frequently used controllers; and algebraic elements with
discontinuities in one of their derivatives, such as the logical switch,
limitation, and dead zone.

In the case of the sample and hold element, the user meets no restriction with
respect to relations between the sample period and the numerical time step.
However, to maintain the order of the method, the control system has to adjust
the time step to collect the appropriate sample points.

Each sample represents a discontinuous change in the variable.
The discontinuity is said to be of order $q$ when it occurs in the $q$'th
derivative of one of the variables and the lower-order derivatives are continuous.
Gear and {\O}sterby~\cite{GearOsterby} have shown that the accuracy of the
result will drop below the order $p$ when $q \le p$, unless we the integration
points hit the discrete discontinuity points.
With $p = 2$, actions must be taken for discontinuities in the first derivative
of the right-hand side of \eqnref{eqC:92}.

The discontinuity points of regular sampling is foreseen easily, however, this is
not the case for other element types, such as the logical switch or dead zone.
In these cases the discontinuity point must be found by interpolation.
For this purpose, an interpolant is used that is optimal for Lobatto IIIC.
For development of interpolants for RK-methods refer to~\cite{RungeKutta}
for an example.

%%%%%%%%%%%%%%%%%%%%%%%%%%%%%%%%%%%%%%%%%%%%%%%%%%%%%%%%%%%%%%%%%%%%%%%%%%%%%%%%
\subsection{Basic Elements}

\subsubsection{Comparator\hfill type 1}
\begin{minipage}{61mm}
\setlength{\unitlength}{1mm}
\begin{picture}(60,30)
\thicklines
\put( 6, 5){\line(1,0){9}}
\put( 5, 5){\circle{2}}
\put( 1, 4){2}
\put(10, 1){$-$}
\put( 6,15){\line(1,0){9}}
\put( 5,15){\circle{2}}
\put( 1,14){1}
\put(10,11){$+$}
\put(15, 0){\framebox(30,20){0}}
\put(54,10){\line(-1,0){9}}
\put(55,10){\circle{2}}
\put(57, 9){3}
\put(30,25){\vector(0,-1){5}}
\put(29,26){$K$}
\end{picture}
\end{minipage}\hfill
\begin{minipage}{55mm}
$\displaystyle
y_3 = K(y_1 - y_2)
$
\end{minipage}

\bigskip
\subsubsection{Adder\hfill type 2}
\begin{minipage}{61mm}
\setlength{\unitlength}{1mm}
\begin{picture}(60,30)
\thicklines
\put( 6, 5){\line(1,0){9}}
\put( 5, 5){\circle{2}}
\put( 1, 4){2}
\put(10, 1){$+$}
\put( 6,15){\line(1,0){9}}
\put( 5,15){\circle{2}}
\put( 1,14){1}
\put(10,11){$+$}
\put(15, 0){\framebox(30,20){0}}
\put(54,10){\line(-1,0){9}}
\put(55,10){\circle{2}}
\put(57, 9){3}
\put(30,25){\vector(0,-1){5}}
\put(29,26){$K$}
\end{picture}
\end{minipage}\hfill
\begin{minipage}{55mm}
$\displaystyle
y_3 = K(y_1 + y_2)
$
\end{minipage}

\bigskip
\subsubsection{Amplifier\hfill type 3}
\begin{minipage}{61mm}
\setlength{\unitlength}{1mm}
\begin{picture}(60,30)
\thicklines
\put( 6,10){\line(1,0){9}}
\put( 5,10){\circle{2}}
\put( 1, 9){1}
\put(15, 0){\framebox(30,20){$K$}}
\put(54,10){\line(-1,0){9}}
\put(55,10){\circle{2}}
\put(57, 9){2}
\put(30,25){\vector(0,-1){5}}
\put(29,26){$K$}
\end{picture}
\end{minipage}\hfill
\begin{minipage}{55mm}
$\displaystyle
y_2 = K y_1
$
\end{minipage}

\bigskip
\subsubsection{Integrator\hfill type 4}
\begin{minipage}{61mm}
\setlength{\unitlength}{1mm}
\begin{picture}(60,30)
\thicklines
\put( 6,10){\line(1,0){9}}
\put( 5,10){\circle{2}}
\put( 1, 9){1}
\put(15, 0){\framebox(30,20){$\displaystyle\frac{K}{s}$}}
\put(54,10){\line(-1,0){9}}
\put(55,10){\circle{2}}
\put(57, 9){2}
\put(30,25){\vector(0,-1){5}}
\put(29,26){$K$}
\end{picture}
\end{minipage}\hfill
\begin{minipage}{55mm}
$\displaystyle
\dot{y}_2 = K y_1
$\\[3mm]
The resulting output is then \\
$\displaystyle
y_2 = K\int_0^t y_1 d\tau
$
\end{minipage}

\bigskip
\subsubsection{Limited derivator\hfill type 5}
\begin{minipage}{61mm}
\setlength{\unitlength}{1mm}
\begin{picture}(60,35)
\thicklines
\put( 6,15){\line(10,0){9}}
\put( 5,15){\circle{2}}
\put( 1,14){1}
\put(15, 5){\framebox(30,20){$\displaystyle\frac{s}{1+T_u s}$}}
\put(54,15){\line(-1,0){9}}
\put(55,15){\circle{2}}
\put(57,14){3}
\put(30,30){\vector(0,-1){5}}
\put(29,31){$T_u$}
\put(30, 1){\circle{2}}
\put(30, 2){\line(0,1){3}}
\put(33, 0){2}
\end{picture}
\end{minipage}\hfill
\begin{minipage}{55mm}
$\displaystyle
\dot{y}_2 = y_3
$\\[1mm]
$\displaystyle
y_3 = \frac{1}{T_u}(y_1 - y_2)
$\\[2mm]
$y_2$ is an internal variable.
\end{minipage}

\bigskip
\subsubsection{Multiplier\hfill type 6}
\begin{minipage}{61mm}
\setlength{\unitlength}{1mm}
\begin{picture}(60,20)
\thicklines
\put( 6, 5){\line(1,0){9}}
\put( 5, 5){\circle{2}}
\put( 1, 4){2}
\put( 6,15){\line(1,0){9}}
\put( 5,15){\circle{2}}
\put( 1,14){1}
\put(15, 0){\framebox(30,20){$\otimes$}}
\put(54,10){\line(-1,0){9}}
\put(55,10){\circle{2}}
\put(57, 9){3}
\end{picture}
\end{minipage}\hfill
\begin{minipage}{55mm}
$\displaystyle
y_3 = y_1 y_2
$
\end{minipage}

\bigskip
\subsubsection{Power\hfill type 7}
\begin{minipage}{61mm}
\setlength{\unitlength}{1mm}
\begin{picture}(60,30)
\thicklines
\put( 6,10){\line(1,0){9}}
\put( 5,10){\circle{2}}
\put( 1, 9){1}
\put(15, 0){\framebox(30,20){$p$}}
\put(54,10){\line(-1,0){9}}
\put(55,10){\circle{2}}
\put(57, 9){2}
\put(30,25){\vector(0,-1){5}}
\put(29,26){$K$}
\end{picture}
\end{minipage}\hfill
\begin{minipage}{55mm}
$\displaystyle
y_2 = y_1^p
$
\end{minipage}

%%%%%%%%%%%%%%%%%%%%%%%%%%%%%%%%%%%%%%%%%%%%%%%%%%%%%%%%%%%%%%%%%%%%%%%%%%%%%%%%
\subsection{Time dependent elements}

\subsubsection{Time delay\hfill type 11}
\begin{minipage}{61mm}
\setlength{\unitlength}{1mm}
\begin{picture}(60,30)
\thicklines
\put( 6,10){\line(1,0){9}}
\put( 5,10){\circle{2}}
\put( 1, 9){1}
\put(30,25){\vector(0,-1){5}}
\put(29,26){$T$}
\put(15, 0){\framebox(30,20){$e^{-T s}$}}
\put(54,10){\line(-1,0){9}}
\put(55,10){\circle{2}}
\put(57, 9){2}
\end{picture}
\end{minipage}\hfill
\begin{minipage}{55mm}
$\displaystyle
y_2(t) = y_1(t - T)
$
\end{minipage}

\bigskip
\subsubsection{Sample and hold\hfill type 12}
\begin{minipage}{61mm}
\setlength{\unitlength}{1mm}
\begin{picture}(60,30)
\thicklines
\put( 6,10){\line(1,0){9}}
\put( 5,10){\circle{2}}
\put( 1, 9){1}
\put(20,25){\vector(0,-1){5}}
\put(19,26){$T$}
\put(30,25){\vector(0,-1){5}}
\put(29,26){$m$}
\put(40,25){\vector(0,-1){5}}
\put(39,26){$d$}
\put(15, 0){\framebox(30,20){$\frac{1}{s}(1-e^{-Ts})$}}
\put(54,10){\line(-1,0){9}}
\put(55,10){\circle{2}}
\put(57, 9){2}
\end{picture}
\end{minipage}\hfill
\begin{minipage}{55mm}
$$\eqalign{
y_2(t) =\:& y_1(t^*) \cr
t^* =\:& (\frac{t- d\, T}{m\, T})m\, T
}\hskip2cm{}$$
\small
\makebox[3mm][l]{$T$} - basic sample period \\
\makebox[3mm][l]{$m$} - multiplicity of basic sample period \\
\makebox[3mm][l]{$d$} - phase delay in basic sample periods
\end{minipage}

%%%%%%%%%%%%%%%%%%%%%%%%%%%%%%%%%%%%%%%%%%%%%%%%%%%%%%%%%%%%%%%%%%%%%%%%%%%%%%%%
\subsection{Piecewise Continuous Elements}

\subsubsection{Logic switch\hfill type 21}
\begin{minipage}{61mm}
\setlength{\unitlength}{1mm}
\begin{picture}(60,30)
\thicklines
\put( 6,10){\line(1,0){9}}
\put( 5,10){\circle{2}}
\put( 1, 9){1}
\put(20,25){\vector(0,-1){5}}
\put(19,26){$L$}
\put(30,25){\vector(0,-1){5}}
\put(29,26){$Y_{\rm on}$}
\put(40,25){\vector(0,-1){5}}
\put(39,26){$U$}
\put(15, 0){\framebox(30,20){}}
% The picture in the middle
\thinlines
\put(17,10){\vector(1,0){26}}
\put(25, 2){\vector(0,1){16}}
\put(20, 5){\line(1,0){15}}
\put(37, 4){{\scriptsize $L$}}
\put(35, 5){\line(0,1){10}}
\put(30,14){{\scriptsize $U$}}
\put(34,16){{\scriptsize $Y_{on}$}}
\put(35,15){\line(1,0){5}}
%
\thicklines
\put(54,10){\line(-1,0){9}}
\put(55,10){\circle{2}}
\put(57, 9){2}
\end{picture}
\end{minipage}\hfill
\begin{minipage}{55mm}
$\displaystyle
y_2 = \left\{\begin{array}{ccc}
L & \text{for} & y_1 < Y_{\rm on} \\
U & \text{for} & y_1 \geq Y_{\rm on}
\end{array}\right.
$\\[3mm]
\makebox[5mm][l]{$L$} - lower value \\
\makebox[5mm][l]{$U$} - upper value \\
\makebox[5mm][l]{$Y_{\rm on}$} - switch coordinate
\end{minipage}

\bigskip
\subsubsection{Limitation\hfill type 22}
\begin{minipage}{61mm}
\setlength{\unitlength}{1mm}
\begin{picture}(60,30)
\thicklines
\put( 6,10){\line(1,0){9}}
\put( 5,10){\circle{2}}
\put( 1, 9){1}
\put(20,25){\vector(0,-1){5}}
\put(19,26){$L$}
\put(40,25){\vector(0,-1){5}}
\put(39,26){$U$}
\put(15, 0){\framebox(30,20){}}
% The picture in the middle
\thinlines
\put(17,10){\vector(1,0){26}}
\put(30, 2){\vector(0,1){16}}
\put(20, 5){\line(1,0){5}}
\put(26, 4){{\scriptsize $L$}}
\put(25, 5){\line(1,1){10}}
\put(32,14){{\scriptsize $U$}}
\put(35,15){\line(1,0){5}}
%
\thicklines
\put(54,10){\line(-1,0){9}}
\put(55,10){\circle{2}}
\put(57, 9){2}
\end{picture}
\end{minipage}\hfill
\begin{minipage}{55mm}
$\displaystyle
y_2 = \max\left\{L,\min\left\{U,y_1\right\}\right\}
$\\[3mm]
\makebox[4mm][l]{$L$}- lower limit \\
\makebox[4mm][l]{$U$}- upper limit
\end{minipage}

\bigskip
\subsubsection{Dead Zone\hfill type 23}
\begin{minipage}{61mm}
\setlength{\unitlength}{1mm}
\begin{picture}(60,30)
\thicklines
\put( 6,10){\line(1,0){9}}
\put( 5,10){\circle{2}}
\put( 1, 9){1}
\put(20,25){\vector(0,-1){5}}
\put(19,26){$L$}
\put(40,25){\vector(0,-1){5}}
\put(39,26){$R$}
\put(15, 0){\framebox(30,20){}}
% The picture in the middle
\thinlines
\put(17,10){\vector(1,0){26}}
\put(30, 2){\vector(0,1){16}}
\put(18, 5){\line(1,1){5}}
\put(22,11){{\scriptsize $L$}}
\put(23,10){\line(1,0){10}}
\put(32, 7){{\scriptsize $R$}}
\put(33,10){\line(1,1){5}}
%
\thicklines
\put(54,10){\line(-1,0){9}}
\put(55,10){\circle{2}}
\put(57, 9){2}
\end{picture}
\end{minipage}\hfill
\begin{minipage}{55mm}
$\displaystyle
y_2 = \left\{\begin{array}{ccl}
y_1 - L & \text{for} & y_1 < L \\
0       & \text{for} & L \leq y_1 \leq R \\
y_1 - R & \text{for} & y_1 > Y_0
\end{array}\right.
$\\[3mm]
\makebox[4mm][l]{$L$} - left limit \\
\makebox[4mm][l]{$U$} - right limit
\end{minipage}

\bigskip
\subsubsection{Hysteresis\hfill type 24}
\begin{minipage}{61mm}
\setlength{\unitlength}{1mm}
\begin{picture}(60,30)
\thicklines
\put( 6,10){\line(1,0){9}}
\put( 5,10){\circle{2}}
\put( 1, 9){1}
\put(20,25){\vector(0,-1){5}}
\put(19,26){$L$}
\put(30,25){\vector(0,-1){5}}
\put(29,26){$R$}
\put(40,25){\vector(0,-1){5}}
\put(39,26){$\alpha$}
\put(15, 0){\framebox(30,20){}}
% The picture in the middle
\thinlines
\put(17,10){\vector(1,0){26}}
\put(30, 2){\vector(0,1){16}}
\put(18, 5){\line(1,1){12}}
\put(22,11){{\scriptsize $L$}}
\put(30,17){\line(1,0){10}}
\put(18, 5){\line(1,0){10}}
\put(32, 7){{\scriptsize $R$}}
\put(28, 5){\line(1,1){12}}
%
\thicklines
\put(54,10){\line(-1,0){9}}
\put(55,10){\circle{2}}
\put(57, 9){2}
\end{picture}
\end{minipage}\hfill
\begin{minipage}{55mm}
$\displaystyle
y_2 = \left\{\begin{array}{cl}
\alpha(y_1 - R) & \text{on right slope} \\
\alpha(y_1 - L) & \text{on left slope} \\
y_1 & \text{otherwise}
\end{array}\right.
$\\[3mm]
\makebox[4mm][l]{$L$} - left limit \\
\makebox[4mm][l]{$R$} - right limit
\end{minipage}

%%%%%%%%%%%%%%%%%%%%%%%%%%%%%%%%%%%%%%%%%%%%%%%%%%%%%%%%%%%%%%%%%%%%%%%%%%%%%%%%
\subsection{Compensator Elements}

\subsubsection{PI\hfill type 31}
\begin{minipage}{61mm}
\setlength{\unitlength}{1mm}
\begin{picture}(60,35)
\thicklines
\put( 6,15){\line(10,0){9}}
\put( 5,15){\circle{2}}
\put( 1,14){1}
\put(15, 5){\framebox(30,20){$\displaystyle K_p\frac{1+T_i s}{T_i s}$}}
\put(54,15){\line(-1,0){9}}
\put(55,15){\circle{2}}
\put(57,14){3}
\put(20,30){\vector(0,-1){5}}
\put(19,31){$K_p$}
\put(40,30){\vector(0,-1){5}}
\put(39,31){$T_i$}
\put(30, 1){\circle{2}}
\put(30, 2){\line(0,1){3}}
\put(33, 0){2}
\end{picture}
\end{minipage}\hfill
\begin{minipage}{55mm}
$\displaystyle
\dot{y}_2 = y_1
$\\
$\displaystyle
y_3 = K_p\left(y_1 + \frac{1}{T_i}y_2\right)
$\\[2mm]
where $y_2$ is an internal variable. \\[3mm]
The resulting output is then \\
$\displaystyle
y_3 = K_p\left(y_1 + \frac{1}{T_i}\int_0^t y_1 d\tau\right)
$
\end{minipage}

\bigskip
\subsubsection{P + lim.\ I\hfill type 32}
\begin{minipage}{61mm}
\setlength{\unitlength}{1mm}
\begin{picture}(60,35)
\thicklines
\put( 6,15){\line(10,0){9}}
\put( 5,15){\circle{2}}
\put( 1,14){1}
\put(15, 5){\framebox(30,20){$\displaystyle K_p\:\frac{T_{fi}}{T_i}\:\frac{1+T_i s}{1+T_{fi}
s}$}}
\put(54,15){\line(-1,0){9}}
\put(55,15){\circle{2}}
\put(57,14){4}
\put(20,30){\vector(0,-1){5}}
\put(19,31){$K_p$}
\put(30,30){\vector(0,-1){5}}
\put(29,31){$T_i$}
\put(40,30){\vector(0,-1){5}}
\put(39,31){$T_{fi}$}
\put(20, 1){\circle{2}}
\put(20, 2){\line(0,1){3}}
\put(23, 0){2}
\put(40, 1){\circle{2}}
\put(40, 2){\line(0,1){3}}
\put(43, 0){3}
\end{picture}
\end{minipage}\hfill
\begin{minipage}{55mm}
$\displaystyle
\dot{y}_2 = y_1 \qquad T_{fi} > T_i
$\\
$\displaystyle
\dot{y}_3 = y_4
$\\
$\displaystyle
y_4 = K_p\left(y_1 + \frac{1}{T_i}y_2\right) - \frac{1}{T_{fi}}y_3
$\\[2mm]
$y_2$ and $y_3$ are internal variables.
\end{minipage}

\bigskip
\subsubsection{PD\hfill type 33}
\begin{minipage}{61mm}
\setlength{\unitlength}{1mm}
\begin{picture}(60,35)
\thicklines
\put( 6,15){\line(10,0){9}}
\put( 5,15){\circle{2}}
\put( 1,14){1}
\put(15, 5){\framebox(30,20){$\displaystyle K_p( 1 + T_d s)$}}
\put(54,15){\line(-1,0){9}}
\put(55,15){\circle{2}}
\put(57,14){4}
\put(20,30){\vector(0,-1){5}}
\put(19,31){$K_p$}
\put(40,30){\vector(0,-1){5}}
\put(39,31){$T_d$}
\put(20, 1){\circle{2}}
\put(20, 2){\line(0,1){3}}
\put(23, 0){2}
\put(40, 1){\circle{2}}
\put(40, 2){\line(0,1){3}}
\put(43, 0){3}
\end{picture}
\end{minipage}\hfill
\begin{minipage}{55mm}
$\displaystyle
\dot{y}_2 = y_1
$\\
$\displaystyle
\dot{y}_3 = y_4
$\\
$\displaystyle
y_3 = K_p\left(T_d y_1 + y_2\right)
$\\[2mm]
where $y_2$ and $y_3$ are \\ internal variables. \\[3mm]
The resulting output is then \\
$\displaystyle
y_4 = K_p\left(y_1 + T_d \dot{y}_1\right)
$
\end{minipage}

\bigskip
\subsubsection{P + lim.\ D\hfill type 34}
\begin{minipage}{61mm}
\setlength{\unitlength}{1mm}
\begin{picture}(60,35)
\thicklines
\put( 6,15){\line(10,0){9}}
\put( 5,15){\circle{2}}
\put( 1,14){1}
\put(15, 5){\framebox(30,20){$\displaystyle K_p\:\frac{1+T_d s}{1+T_{fd} s}$}}
\put(54,15){\line(-1,0){9}}
\put(55,15){\circle{2}}
\put(57,14){4}
\put(20,30){\vector(0,-1){5}}
\put(19,31){$K_p$}
\put(30,30){\vector(0,-1){5}}
\put(29,31){$T_d$}
\put(40,30){\vector(0,-1){5}}
\put(39,31){$T_{fd}$}
\put(20, 1){\circle{2}}
\put(20, 2){\line(0,1){3}}
\put(23, 0){2}
\put(40, 1){\circle{2}}
\put(40, 2){\line(0,1){3}}
\put(43, 0){3}
\end{picture}
\end{minipage}\hfill
\begin{minipage}{55mm}
$\displaystyle
\dot{y}_2 = y_1 \qquad T_{fd} < T_d
$\\
$\displaystyle
\dot{y}_3 = y_4
$\\[2mm]
$\displaystyle
y_4 = \frac{1}{T_{fd}}\left(K_p(T_d y_1 + y_2) - y_3\right)
$\\[2mm]
$y_2$ and $y_3$ are internal variables.
\end{minipage}

\bigskip
\subsubsection{PID\hfill type 35}
\begin{minipage}{61mm}
\setlength{\unitlength}{1mm}
\begin{picture}(60,35)
\thicklines
\put( 6,15){\line(10,0){7}}
\put( 5,15){\circle{2}}
\put( 1,14){1}
\put(13, 5){\framebox(34,20){$\displaystyle
K_p\left(1 + \frac{1}{T_i s} + T_d s\right)$}}
\put(54,15){\line(-1,0){7}}
\put(55,15){\circle{2}}
\put(57,14){5}
\put(20,30){\vector(0,-1){5}}
\put(19,31){$K_p$}
\put(30,30){\vector(0,-1){5}}
\put(29,31){$T_i$}
\put(40,30){\vector(0,-1){5}}
\put(39,31){$T_d$}
\put(20, 1){\circle{2}}
\put(20, 2){\line(0,1){3}}
\put(23, 0){2}
\put(30, 1){\circle{2}}
\put(30, 2){\line(0,1){3}}
\put(33, 0){3}
\put(40, 1){\circle{2}}
\put(40, 2){\line(0,1){3}}
\put(43, 0){4}
\end{picture}
\end{minipage}\hfill
\begin{minipage}{55mm}
$\displaystyle
\dot{y}_2 = y_1
$\\
$\displaystyle
\dot{y}_3 = y_2
$\\
$\displaystyle
\dot{y}_4 = y_5
$\\
$\displaystyle
y_4 = K_p\left(T_d y_1 + y_2 + \frac{1}{T_i}y_3\right)
$\\[2mm]
where $y_2$, $y_3$ and $y_4$ \\ are internal variables. \\[3mm]
The resulting output is then \\
$\displaystyle
y_5 = K_p\left(y_1 + \frac{1}{T_i}\int_0^t y_1 d\tau + T_d\dot{y}_1\right)
$
\end{minipage}

\bigskip
\subsubsection{PI + lim.\ D\hfill type 36}
\begin{minipage}{61mm}
\setlength{\unitlength}{1mm}
\begin{picture}(60,35)
\thicklines
\put( 6,15){\line(10,0){5}}
\put( 5,15){\circle{2}}
\put( 1,14){1}
\put(11, 5){\framebox(38,20){$\displaystyle
K_p\:\frac{1 + T_i s}{T_i s}\:\frac{1 + T_d s}{1 + T_{fd} s}$}}
\put(54,15){\line(-1,0){5}}
\put(55,15){\circle{2}}
\put(57,14){5}
\put(20,30){\vector(0,-1){5}}
\put(19,31){$K_p$}
\put(27,30){\vector(0,-1){5}}
\put(26,31){$T_i$}
\put(34,30){\vector(0,-1){5}}
\put(33,31){$T_d$}
\put(41,30){\vector(0,-1){5}}
\put(40,31){$T_{fd}$}
\put(20, 1){\circle{2}}
\put(20, 2){\line(0,1){3}}
\put(23, 0){2}
\put(30, 1){\circle{2}}
\put(30, 2){\line(0,1){3}}
\put(33, 0){3}
\put(40, 1){\circle{2}}
\put(40, 2){\line(0,1){3}}
\put(43, 0){4}
\end{picture}
\end{minipage}\hfill
\begin{minipage}{55mm}
$\displaystyle
\dot{y}_2 = y_1
$\\
$\displaystyle
\dot{y}_3 = K_p\left(y_1 + \frac{1}{T_i}y_2\right)
$\\
$\displaystyle
\dot{y}_4 = y_5
$\\
$\displaystyle
y_5 = \frac{1}{T_{fd}}\left(T_d \dot y_3 + y_3 - y_4\right)
$\\[2mm]
$y_2$, $y_3$ and $y_4$ are internal variables.
\end{minipage}

\bigskip
\subsubsection{P + lim.\ I + lim.\ D\hfill type 37}
\begin{minipage}{61mm}
\setlength{\unitlength}{1mm}
\begin{picture}(60,35)
\thicklines
\put( 6,15){\line(10,0){2}}
\put( 5,15){\circle{2}}
\put( 1,14){1}
\put( 8, 5){\framebox(44,20){$\displaystyle
K_p\:\frac{T_{fi}}{T_i}\:\frac{1 + T_i s}{1 + T_{fi} s}\:\frac{1 + T_d s}{1 + T_{fd} s} $}}
\put(54,15){\line(-1,0){2}}
\put(55,15){\circle{2}}
\put(57,14){5}
\put(14,30){\vector(0,-1){5}}
\put(13,31){$K_p$}
\put(22,30){\vector(0,-1){5}}
\put(21,31){$T_i$}
\put(30,30){\vector(0,-1){5}}
\put(29,31){$T_d$}
\put(38,30){\vector(0,-1){5}}
\put(37,31){$T_{fi}$}
\put(46,30){\vector(0,-1){5}}
\put(45,31){$T_{fd}$}
\put(20, 1){\circle{2}}
\put(20, 2){\line(0,1){3}}
\put(23, 0){2}
\put(30, 1){\circle{2}}
\put(30, 2){\line(0,1){3}}
\put(33, 0){3}
\put(40, 1){\circle{2}}
\put(40, 2){\line(0,1){3}}
\put(43, 0){4}
\end{picture}
\end{minipage}\hfill
\begin{minipage}{55mm}
$\displaystyle
\dot{y}_2 = y_1
$\\
$\displaystyle
\dot{y}_3 = K_p\left(y_1 + \frac{1}{T_i}y_2\right) - \frac{1}{T_{fi}}y_3
$\\
$\displaystyle
\dot{y}_4 = y_5
$\\
$\displaystyle
y_5 = \frac{1}{T_{fd}}\left(T_d \dot y_3 + y_3 - y_4\right)
$\\[2mm]
$y_2$, $y_3$ and $y_4$ are internal variables.
\end{minipage}

%%%%%%%%%%%%%%%%%%%%%%%%%%%%%%%%%%%%%%%%%%%%%%%%%%%%%%%%%%%%%%%%%%%%%%%%%%%%%%%%
\subsection{General Transfer Functions}

\subsubsection{Real pole\hfill type 41}
\begin{minipage}{61mm}
\setlength{\unitlength}{1mm}
\begin{picture}(60,30)
\thicklines
\put( 6,10){\line(1,0){9}}
\put( 5,10){\circle{2}}
\put( 1, 9){1}
\put(15, 0){\framebox(30,20){$\displaystyle \frac{K}{1 + T s}$}}
\put(54,10){\line(-1,0){9}}
\put(55,10){\circle{2}}
\put(57, 9){2}
\put(20,25){\vector(0,-1){5}}
\put(19,26){$K$}
\put(40,25){\vector(0,-1){5}}
\put(39,26){$T$}
\end{picture}
\end{minipage}\hfill
\begin{minipage}{55mm}
$\displaystyle
y_2 = \frac{1}{T}(K y_1 - y_2)
$
\end{minipage}

\bigskip
\subsubsection{Complex Conjugate Pole\hfill type 42}
\begin{minipage}{61mm}
\setlength{\unitlength}{1mm}
\begin{picture}(60,35)
\thicklines
\put( 6,15){\line(10,0){7}}
\put( 5,15){\circle{2}}
\put( 1,14){1}
\put(13, 5){\framebox(34,20){$\displaystyle
\frac{K}{\omega_0 + 2\xi\omega_0 s + s^2}$}}
\put(54,15){\line(-1,0){7}}
\put(55,15){\circle{2}}
\put(57,14){3}
\put(20,30){\vector(0,-1){5}}
\put(19,31){$K$}
\put(30,30){\vector(0,-1){5}}
\put(29,31){$\omega_0$}
\put(40,30){\vector(0,-1){5}}
\put(39,31){$\xi$}
\put(30, 1){\circle{2}}
\put(30, 2){\line(0,1){3}}
\put(33, 0){2}
\end{picture}
\end{minipage}\hfill
\begin{minipage}{55mm}
$\displaystyle
\dot{y}_2 = K y_1 - 2 \xi\omega_0 - \omega_0^2 y_3
$\\
$\displaystyle
\dot{y}_3 = y_2
$\\[2mm]
$y_2$ is an internal variable.
\end{minipage}

\bigskip
\subsubsection{1st Order Element\hfill type 43}
\begin{minipage}{61mm}
\setlength{\unitlength}{1mm}
\begin{picture}(60,35)
\thicklines
\put( 6,15){\line(10,0){7}}
\put( 5,15){\circle{2}}
\put( 1,14){1}
\put(13, 5){\framebox(34,20){$\displaystyle K_p\:\frac{1 + T_2 s}{1 + T_1 s}$}}
\put(54,15){\line(-1,0){7}}
\put(55,15){\circle{2}}
\put(57,14){4}
\put(20,30){\vector(0,-1){5}}
\put(19,31){$K_p$}
\put(30,30){\vector(0,-1){5}}
\put(29,31){$T_1$}
\put(40,30){\vector(0,-1){5}}
\put(39,31){$T_2$}
\put(25, 1){\circle{2}}
\put(25, 2){\line(0,1){3}}
\put(28, 0){2}
\put(35, 1){\circle{2}}
\put(35, 2){\line(0,1){3}}
\put(38, 0){3}
\end{picture}
\end{minipage}\hfill
\begin{minipage}{55mm}
$\displaystyle
\dot{y}_2 = y_1
$\\
$\displaystyle
\dot{y}_3 = y_4
$\\
$\displaystyle
\dot{y}_4 = \frac{1}{T_1}\left( K_p \left( T_2 y_1 + y_2 \right) - y_3 \right)
$\\[2mm]
$y_2$ and $y_3$ are internal variables.
\end{minipage}

\bigskip
\subsubsection{2nd Order Element\hfill type 44}
\begin{minipage}{61mm}
\setlength{\unitlength}{1mm}
\begin{picture}(60,35)
\thicklines
\put( 6,15){\line(10,0){7}}
\put( 5,15){\circle{2}}
\put( 1,14){1}
\put(13, 5){\framebox(34,20){$\displaystyle K_p\:\frac{1 + T_3 s + T_4 s^2}
{1 + T_1 s + T_2 s^2}$}}
\put(54,15){\line(-1,0){7}}
\put(55,15){\circle{2}}
\put(57,14){6}
\put(14,30){\vector(0,-1){5}}
\put(13,31){$K_p$}
\put(22,30){\vector(0,-1){5}}
\put(21,31){$T_1$}
\put(30,30){\vector(0,-1){5}}
\put(29,31){$T_2$}
\put(38,30){\vector(0,-1){5}}
\put(37,31){$T_3$}
\put(46,30){\vector(0,-1){5}}
\put(45,31){$T_4$}
\put(15, 1){\circle{2}}
\put(15, 2){\line(0,1){3}}
\put(18, 0){2}
\put(25, 1){\circle{2}}
\put(25, 2){\line(0,1){3}}
\put(28, 0){3}
\put(35, 1){\circle{2}}
\put(35, 2){\line(0,1){3}}
\put(38, 0){4}
\put(45, 1){\circle{2}}
\put(45, 2){\line(0,1){3}}
\put(48, 0){5}
\end{picture}
\end{minipage}\hfill
\begin{minipage}{55mm}
$\displaystyle
\dot{y}_2 = y_1
$\\
$\displaystyle
\dot{y}_3 = y_2
$\\
$\displaystyle
\dot{y}_4 = y_6
$\\
$\displaystyle
\dot{y}_5 = y_4
$\\[2mm]
$\displaystyle\eqalign{
\dot{y}_6 = \frac{1}{T_2}(& K_p( T_4 y_1 + T_3 y_2 + y_3) \cr
                          & - T_1 y_4 - y_5)}
$\\[2mm]
$y_2$, $y_3$, $y_4$ and $y_5$ are internal \\ variables.
\end{minipage}
