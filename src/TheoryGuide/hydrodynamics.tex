% SPDX-FileCopyrightText: 2023 SAP SE
%
% SPDX-License-Identifier: Apache-2.0
%
% This file is part of FEDEM - https://openfedem.org

%%%%%%%%%%%%%%%%%%%%%%%%%%%%%%%%%%%%%%%%%%%%%%%%%%%%%%%%%%%%%%%%%%%%%%%%%%%%%%%%
%
% FEDEM Theory Guide.
%
%%%%%%%%%%%%%%%%%%%%%%%%%%%%%%%%%%%%%%%%%%%%%%%%%%%%%%%%%%%%%%%%%%%%%%%%%%%%%%%%

\def\rhow{\rho_{\rm w}}
\def\Db{D_{\rm b}}
\def\Dd{D_{\rm d}}
\def\Cd{C_{\rm d}}
\def\Ca{C_{\rm a}}
\def\Cm{C_{\rm m}}
\def\Cda{C_{\rm d,axial}}
\def\Caa{C_{\rm a,axial}}
\def\Cma{C_{\rm m,axial}}

\chapter{Hydrodynamic loads}
\label{chap:Hydrodynamics}

In this Appendix, the additional loads acting on mechanism models due to a
marine environment are described.
These loads are treated as external forces in the dynamics simulation, in the
same manner as described in Section~\ref{subs:External forces}.

\section{Buoyancy of beam elements}

The buoyancy force per unit length of a uniform and linear beam element immersed
in sea water can be computed as
%
\begin{equation}
\label{eq:buoyancy force}
f_{\rm b} = \frac{\pi}{4}\Db^2 \rhow g
\end{equation}
%
where $\Db$ is the effective buoyancy diameter of the beam cross section,
$\rhow$ is the specified mass density of the sea water, and $g=\|{\bf g}\|$
is the gravitation constant, with {\bf g} being the gravitation vector.

For a beam element with length $L_e$ that is totally submerged in water, i.e.,
both of its end points are below the water surface, the total buoyancy force
vector attacking at each end node of the element is
%
\begin{equation}
{\bf F}_{{\rm b}_i}^e = -f_{\rm b} \frac{L_e}{2} \frac{\bf g}{g} =
-\frac{\pi}{8}\Db^2 L \rhow {\bf g} \quad i=1,2
\end{equation}

\section{Drag and added mass}

Forces due to drag and added mass on a beam element immersed in a fluid (water)
can be computed based on the Morison equation, see e.g.~\cite{Chakrabarti},
pages 169--194 for an elaboration.
This is a semi-empirical load model valid only for slender beams with circular
outer cross sections.
The drag and added mass forces per unit length of the beam are here given as,
respectively.
%
\begin{eqnarray}
\label{eq:drag force}
f_{\rm d} &=& \frac{1}{2}\Dd \rhow \Cd \left(u-v\right)|u-v| \\
\label{eq:added mass force}
f_{\rm a} &=& \frac{\pi}{4}\Dd^2 \rhow \left(\Cm\dot{u}-\Ca\dot{v}\right)
\end{eqnarray}
%
where $u$ and $v$ denote the fluid and structure velocities, respectively,
and $\dot{u}$ and $\dot{v}$ are corresponding time derivatives (accelerations),
$\Dd$ is the effective drag diameter of the cross section, $\Cd$ is the drag
coefficient, whereas $\Cm$ and $\Ca$ are inertia and added mass coefficients,
respectively, which normally are related to each other through $\Cm = 1 + \Ca$.

\subsection{Drag forces and associated damping matrix terms}

The drag force, $f_{\rm d}$, is assumed to act normal to the beam axis at any
time, and in the opposite direction of the relative fluid velocity, $u-v$.
It is clear from equation~(\ref{eq:drag force}) that $f_{\rm d}$ is a nonlinear
function of the velocity, and thus we need to calculate the magnitude of the
drag force resultant in the normal plane of the beam, and then transform this
resultant onto the element axes of the beam, rather than applying
equation~(\ref{eq:drag force}) in each coordinate direction.

Let $\bm{u}_i$ and $\bm{v}_i$ denote the global fluid- and structure velocities,
respectively, evaluated at the two nodes, $i=1,2$, of the beam element $e$.
The coordinate vectors of these velocities, referred to the local axes of the
beam element are then calculated as
%
\begin{equation}
\bm{u}_i^e =
\left[\begin{array}{c} u_x \\ u_y \\ u_z \end{array}\right]^e_i =\;
{\bf R}_e^T\bm{u}_i
\quad\mbox{and}\quad
\bm{v}_i^e =
\left[\begin{array}{c} v_x \\ v_y \\ v_z \end{array}\right]^e_i =\;
{\bf R}_e^T\bm{v}_i
\end{equation}
%
where ${\bf R}_e$ is the local-to-global transformation matrix of beam element
$e$, which is equivalent to the rotational part of its updated position matrix.
Since the local $x$-axis of the beam element is always coincident with the axis
passing through the two element nodes, the magnitude of the normal fluid
velocity relative to the structure at an element node is equal to
%
\begin{equation}
\label{eq:u_rel}
u_{\rm rel} = \sqrt{\left(u_y-v_y\right)^2 + \left(u_z-v_z\right)^2}
\end{equation}

The magnitude of the nodal drag force intensity can now be computed as
%
\begin{equation}
\label{eq:drag force magnitude}
f_{{\rm d}_i} = \frac{1}{2}\Dd \rhow \Cd u_{\rm rel}^2
\end{equation}
%
The direction of this force in the local $yz$-plane, defined through the angle
$\theta$ between the local $y$-axis and the force vector, is given by
%
\begin{equation}
\label{eq:drag force direction}
\tan\theta = \frac{u_z - v_z}{u_y - v_y}
\end{equation}
%
such that the drag force vector, expressed in the element coordinate system,
can be computed as
%
\begin{equation}
\label{eq:drag force vector}
\bm{f}_{{\rm d}_i}^e = \left[\begin{array}{c}
0 \\ f_{{\rm d}y} \\ f_{{\rm d}z} \end{array}\right]_i^e =
f_{{\rm d}_i} \left[\begin{array}{c}
0 \\ \cos\theta \\ \sin\theta \end{array}\right]
\end{equation}
%
Using the identities $\cos(\arctan x) = \frac{1}{\sqrt{1+x^2}}$
and $\sin(\arctan x) = \frac{x}{\sqrt{1+x^2}}$, the combination of
equations~(\ref{eq:u_rel}--\ref{eq:drag force direction}) with
equation~(\ref{eq:drag force vector}) results in
%
\begin{equation}
\left[\begin{array}{c}
f_{{\rm d}y} \\ f_{{\rm d}z} \end{array}\right]_i^e =
\frac{1}{2}\Dd \rhow \Cd u_{\rm rel}
\left[\begin{array}{c} u_y - v_y \\ u_z - v_z \end{array}\right]
\end{equation}
%
In addition, we may account for drag in the axial direction (skin friction),
through a separate drag coefficient $\Cda$ for that direction.
We then use equation~(\ref{eq:drag force}) directly, viz.
%
\begin{equation}
f_{{{\rm d}x}_i}^e = \frac{1}{2}\Dd\rhow\Cda|u_x-v_x|\left(u_x-v_x\right)
\end{equation}
%
such that the total drag force vector at node $i$ of element $e$ in its local
coordinate system becomes
%
\begin{equation}
\label{eq:local drag force vector}
\bm{f}_{{\rm d}_i}^e = \frac{1}{2}\Dd\rhow \left[\begin{array}{r}
\Cda\, |u_x-v_x|\,\left(u_x-v_x\right) \\
\Cd\, u_{\rm rel}\, \left(u_y-v_y\right) \\
\Cd\, u_{\rm rel}\, \left(u_z-v_z\right) \end{array}\right]
\end{equation}

The tangent matrix terms associated with this drag force are obtained by taking
the variation of equation~(\ref{eq:local drag force vector}) with respect to the
local structural velocities, $\bm{v}_i^e$, i.e.,
(omitting the indices $e$ and $i$)
%
\begin{equation}
\delta\bm{f}_{\rm d} =
\frac{\partial\bm{f}_{\rm d}}{\partial\bm{v}} \delta\bm{v} =
-{\bf c}_{{\rm d}_i}^e \delta\bm{v}
\end{equation}
%
where ${\bf c}_{{\rm d}_i}^e$ is a $3\times3$ damping matrix with the non-zero
terms as follows:
%
\begin{eqnarray}
c_{{\rm d}11} = -\frac{\partial f_{{\rm d}x}}{\partial v_x} &=&
\Dd\rhow \Cda |u_x - v_x| \\
%
c_{{\rm d}22} = -\frac{\partial f_{{\rm d}y}}{\partial v_y} &=&
-\frac{1}{2}\Dd\rhow \Cd \left(
\frac{\partial u_{\rm rel}}{\partial v_y}\left(u_y-v_y\right) - u_{\rm rel}
\right) \\
%
c_{{\rm d}33} = -\frac{\partial f_{{\rm d}z}}{\partial v_z} &=&
-\frac{1}{2}\Dd\rhow \Cd \left(
\frac{\partial u_{\rm rel}}{\partial v_z}\left(u_z-v_z\right) - u_{\rm rel}
\right) \\
%
c_{{\rm d}32} = -\frac{\partial f_{{\rm d}z}}{\partial v_y} &=&
-\frac{1}{2}\Dd\rhow \Cd
\frac{\partial u_{\rm rel}}{\partial v_y}\left(u_z-v_z\right) \\
%
c_{{\rm d}23} = -\frac{\partial f_{{\rm d}y}}{\partial v_z} &=&
-\frac{1}{2}\Dd\rhow \Cd
\frac{\partial u_{\rm rel}}{\partial v_z}\left(u_y-v_y\right)
\end{eqnarray}
%
From equation~(\ref{eq:u_rel}) it follows that
%
\begin{equation}
\frac{\partial u_{\rm rel}}{\partial v_i} = -\frac{u_i-v_i}{u_{\rm rel}}
\;,\quad i=y,z
\end{equation}
%
which results in
%
\begin{eqnarray}
c_{{\rm d}11} &=& \Dd\rhow\Cda |u_x - v_x| \\
%
c_{{\rm d}22} &=& \frac{1}{2}\Dd\rhow\Cd \left(
\frac{\left(u_y-v_y\right)^2}{u_{\rm rel}} + u_{\rm rel}
\right) \\
%
c_{{\rm d}33} &=& \frac{1}{2}\Dd\rhow\Cd \left(
\frac{\left(u_z-v_z\right)^2}{u_{\rm rel}} + u_{\rm rel}
\right) \\
%
c_{{\rm d}23} = c_{{\rm d}32} &=& \frac{1}{2}\Dd\rhow\Cd
\frac{\left(u_y-v_y\right)\left(u_z-v_z\right)}{u_{\rm rel}}
\end{eqnarray}

\subsection{Added mass forces and associated mass matrix terms}

The added mass force, $f_{\rm a}$, is also assumed to act normal to the beam
axis at any time.
However, it is clear the $f_{\rm a}$ is a linear function of the acceleration,
and thus we may compute the added mass force resultant by using
equation~(\ref{eq:added mass force}) in each coordinate direction, as follows:
%
\begin{equation}
\label{eq:local added mass force vector}
\bm{f}_{{\rm a}_i}^e = \frac{\pi}{4}\Dd^2\rhow \left(\left[\begin{array}{r}
C_{\rm m,axial}\dot{u}_x \\ \Cm\dot{u}_y \\ \Cm\dot{u}_z
\end{array}\right] - \left[\begin{array}{r}
\Caa\dot{v}_x \\ \Ca\dot{v}_y \\ \Ca\dot{v}_z
\end{array}\right]\right)
\end{equation}
%
where $\Cma$ and $\Caa$ are inertia and added mass coefficients for motion in
the axial direction of the beam element.
Normally these coefficients are both zero.

The tangent matrix terms associated with this added mass force are obtained by
taking the variation of equation~(\ref{eq:local added mass force vector}) with
respect to the local structural accelerations, $\dot{\bm{v}}_i^e$, i.e.,
(omitting the indices $e$ and $i$)
%
\begin{equation}
\delta\bm{f}_{\rm a} =
\frac{\partial\bm{f}_{\rm a}}{\partial\dot{\bm{v}}} \delta\dot{\bm{v}} =
-{\bf m}_{{\rm a}_i}^e \delta\dot{\bm{v}}
\end{equation}
%
where ${\bf m}_{{\rm a}_i}^e$ is a $3\times3$ diagonal mass matrix with the
non-zero terms as follows:
%
\begin{eqnarray}
m_{{\rm a}11} = -\frac{\partial f_{{\rm a}x}}{\partial\dot{v}_x} &=&
\frac{\pi}{4}\Dd^2\rhow\Caa \\
%
m_{{\rm a}22} = -\frac{\partial f_{{\rm a}y}}{\partial\dot{v}_y} &=&
\frac{\pi}{4}\Dd^2\rhow\Ca \\
%
m_{{\rm a}33} = -\frac{\partial f_{{\rm a}z}}{\partial\dot{v}_z} &=&
\frac{\pi}{4}\Dd^2\rhow\Ca
\end{eqnarray}

\section{Hydrodynamic element load vectors}

The above expressions for drag and added mass forces denote forces per unit
length of the beam element, evaluated at each end node of the element.
To get the total force vector for a given element we must integrate these forces
along the element.
We do that by assuming a linear variation of the hydrodynamic forces along each
element between the two end point values, i.e.
%
\begin{equation}
\bm{f}_{\rm hd}^e(\xi) := (1-\xi)(\bm{f}_{{\rm d}_1}^e+\bm{f}_{{\rm a}_1}^e) +
\xi(\bm{f}_{{\rm d}_2}^e+\bm{f}_{{\rm a}_2}^e)\quad\forall\;\xi\in[0,1]
\end{equation}
%
The consistent hydrodynamic element load vectors may then be calculated as
%
\begin{equation}
\eqalign{
{\bf F}^e =
\left[\begin{array}{c} {\bf F}_1^e \\ {\bf F}_2^e \end{array}\right] = &
\int\limits_0^1 \left[\begin{array}{l}
\bm{f}_{\rm hd}^e(\xi) (1-\xi) \\
\bm{f}_{\rm hd}^e(\xi) \xi
\end{array}\right] L_e{\rm d}\xi \cr = & \left[\begin{array}{l}
\frac{1}{3}(\bm{f}_{{\rm d}_1}^e+\bm{f}_{{\rm a}_1}^e) +
\frac{1}{6}(\bm{f}_{{\rm d}_2}^e+\bm{f}_{{\rm a}_2}^e) \\[1mm]
\frac{1}{6}(\bm{f}_{{\rm d}_1}^e+\bm{f}_{{\rm a}_1}^e) +
\frac{1}{3}(\bm{f}_{{\rm d}_2}^e+\bm{f}_{{\rm a}_2}^e)
\end{array}\right] L_e}
\end{equation}

Similarly, the element drag damping matrix becomes
%
\begin{equation}
{\bf C}_{\rm d}^e \;= \left[\begin{array}{cc}
{\bf C}_{{\rm d}_1}^e & {\bf 0} \\ {\bf 0} & {\bf C}_{{\rm d}_2}^e
\end{array}\right] = \left[\begin{array}{cc}
\frac{1}{3}{\bf c}_{{\rm d}_1}^e +
\frac{1}{6}{\bf c}_{{\rm d}_2}^e & {\bf 0} \\[1mm] {\bf 0} &
\frac{1}{6}{\bf c}_{{\rm d}_1}^e +
\frac{1}{3}{\bf c}_{{\rm d}_2}^e
\end{array}\right] L_e
\end{equation}
%
and the same for the added mass matrix, ${\bf M}_{\rm a}^e$.
However, since ${\bf m}_{{\rm a}_1}^e = {\bf m}_{{\rm a}_2}^e$,
it follows that the element added mass matrix can be written
%
\begin{equation}
{\bf M}_{\rm a}^e \;= \left[\begin{array}{cc}
{\bf M}_{{\rm a}_1}^e & {\bf 0} \\ {\bf 0} & {\bf M}_{{\rm a}_2}^e
\end{array}\right] = \left[\begin{array}{cc}
{\bf m}_{{\rm a}_1}^e & {\bf 0} \\[1mm] {\bf 0} & {\bf m}_{{\rm a}_2}^e
\end{array}\right] \frac{L_e}{2}
\end{equation}
