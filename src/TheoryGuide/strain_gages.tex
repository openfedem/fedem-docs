% SPDX-FileCopyrightText: 2023 SAP SE
%
% SPDX-License-Identifier: Apache-2.0
%
% This file is part of FEDEM - https://openfedem.org

%%%%%%%%%%%%%%%%%%%%%%%%%%%%%%%%%%%%%%%%%%%%%%%%%%%%%%%%%%%%%%%%%%%%%%%%%%%%%%%%
%
% FEDEM Theory Guide.
%
%%%%%%%%%%%%%%%%%%%%%%%%%%%%%%%%%%%%%%%%%%%%%%%%%%%%%%%%%%%%%%%%%%%%%%%%%%%%%%%%

\section{Virtual strain gauges}

The strains computed from a finite element provide the basis for computing the
strain-rosette strains.
Both three- and four-node strain-measuring elements can be defined,
with the strain computations at the centroid of both elements.
The membrane formulation of the elements is a constant strain triangle and a
bi-linear quadrilateral element with a $C^0$ bending approach.
The strains of these elements can be expressed as
%
\begin{equation}
\tilde{\tf\varepsilon} \;=\; \tilde{\mf B} \tilde{\mf v}
\end{equation}
%
where $\tilde{\mf B}$ is the strain-displacement matrix in the local element
coordinate system.

The strain-rosette element is defined by the user and overlaid on the
actual FE mesh.
It does not, however, contribute any stiffness to the structure.
It exists only as a post-processing device for the purpose of calculating
the strains at that location.

The Cartesian strains in the rosette coordinate system can be computed as
%
\begin{equation}
\eqalign{
{\tf\varepsilon}_r \;=\; {\mf T}_{\varepsilon_r} \tilde{\tf\varepsilon}
\quad \text{where} & \quad
{\mf T}_{\varepsilon_r} = \left[ \begin{array}{*{20}c}
c_{11}^2  & c_{12}^2 & c_{11} c_{12} \\
c_{21}^2  & c_{22}^2 & c_{21} c_{22} \\
2c_{11} c_{21} & 2c_{22} c_{12} & (c_{11} c_{22} + c_{12} c_{21})
\end{array} \right] \cr
\text{and} & \quad c_{ij} \;=\; {\tf i}_i^r \cdot {\tf i}_j
}
\end{equation}
%
where ${\tf i}_i^r$ are base vectors for the rosette coordinate system, and
${\tf i}_j$ are base vectors for the element coordinate system.

Once the Cartesian strains ${\tf \varepsilon}_r$ are established, the user can
compute the strain for each leg of the rosette ${\tf\varepsilon}_l$, defined by
the unit direction vector ${\tf i}^l$ of the leg
%
\begin{equation}
\eqalign{
{\varepsilon}_l \;=\; {\mf t}_l {\tf\varepsilon}_r \quad \text{where} & \quad
{\mf t}_l \;=\; \left[ \begin{array}{ccc}
c_{l1}^2 & c_{l2}^2 & c_{l1}\:c_{l2}
\end{array} \right] \cr
\text{and} & \quad c_{lj} = {\tf i}_l \cdot {\tf i}_j^r
}
\end{equation}

\subsection{Recovering the element displacement vectors}

To compute the element strains and the strain gauge readings,
the element displacement vector $\tilde{\mf v}$ must be recovered from the
superelement displacement vector ${\mf v}_s$.

The free (unconstrained) DOFs for a superelement can be recovered from the
external triad DOFs ${\mf v}_{se}$, and the external DOFs ${\mf y}$ can be
recovered by means of the inverse CMS-transformation of Chapter~\ref{chap:CMS},
(see \eqnref{eqCMS:Hdef}):
%
\begin{equation}
{\mf v}_{\text{free}} \;=\; {\mf H v}_s \quad\text{or}\quad
\left[\begin{array}{c} {\mf v}_i \\ {\mf v}_e \end{array}\right] \;=\;
\left[\begin{array}{cc} {\mf B} & {\tf\Phi} \\ {\mf I} & {\mf 0}
\end{array}\right]
\left[\begin{array}{c} {\mf v}_{se} \\ {\mf y}
\end{array}\right]
\end{equation}
%
The columns of matrix $\mf B$ are displacement vectors; these vectors represent
the static displacements that occur when each of the external triad DOFs is
subjected to a unit displacement, one at a time.
The eigenvectors of the superelement are contained by $\tf\Phi$ when all the
external triad DOFs are fixed.

In addition, the full set of DOFs has to be recovered from possible linear
couplings by means of the matrix relationship\footnote{\label{effNote}
Numerically, this is handled in a much more efficient manner.}
%
\begin{equation}
{\mf v}_\text{full} \;=\; {\mf L v}_\text{free}
\end{equation}
%
where it is assumed that no prescribed displacements are allowed on the
internal DOFs, i.e., prescribed displacements can exist on external DOFs only.

The element displacement vector can now be extracted from the full system
displacement vector as$^\text{\ref{effNote}}$
%
\begin{equation}
{\mf v} \;=\; {\mf A v}_\text{full}
\end{equation}
%
where $\mf A$ is the incidence matrix containing only $0$s and $1$s,
collecting the correct DOFs from the full superelement vector.

Transformation to the local element coordinate system is handled by
%
\begin{equation}
\eqalign{
\tilde{\mf v} \;=\; {\mf T v} \quad \text{where} & \quad
{\mf T} \;=\; \left\lceil\begin{array}{ccc}
{\mf t} & \cdots & {\mf t}
\end{array}\right\rfloor \cr \text{and} & \quad
{\mf t} \;=\; \left[\begin{array}{ccc}
{\tf i}_1\cdot{\tf I}_1 & {\tf i}_1\cdot{\tf I}_2 & {\tf i}_1\cdot{\tf I}_3 \\
{\tf i}_2\cdot{\tf I}_1 & {\tf i}_2\cdot{\tf I}_2 & {\tf i}_2\cdot{\tf I}_3 \\
{\tf i}_3\cdot{\tf I}_1 & {\tf i}_3\cdot{\tf I}_2 & {\tf i}_3\cdot{\tf I}_3
\end{array}\right]
}
\end{equation}
%
using $\lceil\;\rfloor$ to signify the diagonal matrix.

The superelement-to-element displacement relationship can now be written:

\begin{equation}
\tilde{\mf v} \;=\; {\mf E}{\mf v}_s
\quad\text{where}\quad
{\mf E} = {\mf T A L H}
\end{equation}

\subsection{Strain displacement relationship for the strain-rosette}

The strain displacement matrix for the Cartesian strains in the rosette
coordinate system can now be established at the element location as a function
of the superelement DOFs, expressed as:
%
\begin{equation}
\label{eq:rosette strain}
{\tf \varepsilon}_r = {\mf B}{\mf v}_s
\quad\text{where}\quad
{\mf B} \;=\; {\mf T}_{\varepsilon r} \tilde {\mf B E}
\end{equation}

\remark{The strain-displacement matrix ${\mf B}$ of \eqnref{eq:rosette strain}
is computed for each rosette during initialization and stored in memory while
computing the strains for a time series.
${\mf B} = {\mf T}_{\varepsilon r} \tilde{\mf B}{\mf E}$ is thus computed once,
whereas ${\varepsilon}_r = {\mf B}{\mf v}_s$ is computed for each time step.}

The strain-gauge readings for each leg are then obtained from the Cartesian
strains as
%
\begin{equation}
\label{eq:gage strain}
\varepsilon_l \;=\; {\mf t}_l {\tf\varepsilon}_r
\quad\text{and}\quad
\sigma_l \;=\; {\mf t}_l {\mf C} {\tf\varepsilon}_r
\end{equation}
%
where $\mf C$ is the constitutive matrix from \eqnref{eqK43:18}.
